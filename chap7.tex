\secrel{Chapter 7\ --- Measuring and measurement tools}\secdown

\secrel{Measurement throughout history}

No one knows who was the first to actually measure something, but at some point
in history, craftsmanship reached a point where doing things "by eye"\ just
wasn't good enough any more. For example, the builders of the pyramids would
start the job out with a gage carved in stone. From this, each worker would mark
his own measuring stick. The punishment for having a stick that wasn't correctly
calibrated was \emph{death}. Common sense tells me that they quickly arrived at
only one stone for a standard and that the craftsmen were pretty careful about
marking their measuring sticks. It could ruin your whole day to have to kill
your best stone mason because he got his sticks mixed up. They must have had a
number of jobs going on at the same time, and I'm sure some enterprising
Egyptian did good business selling calibrated sticks, saving everyone in the
building trades a lot of trouble.

\secrel{The modem stick and stone}

Things today really haven't changed all that much. We still need standards for
measurement and we keep coming up with ways to divide the "stick"\ into smaller
and smaller parts. The "stone"\ is now a gas that is measured to the atom
electronically, whether you use the inch or the metric system. The "sticks"\ of
today are the rulers, tapes, scales and electronic readouts we use daily.

\secrel{Measurement increments had a human touch}

You might find it interesting to note that most of the dimensions we are
familiar with in the inch system came from parts of the human body. The inch was
once based on the length of the last joint of the thumb. A yard was the distance
from the tip of the nose to the end of the fingers with the arm outstretched. (A
seamstress measures "yards of cloth"\ by pulling from nose to fingertips to this
day.) A fathom is two yards, or the distance from outstretched fingertip to
fingertip. The ancient Egyptians also used cubits (elbow to fingertips), digits
(one finger width), palms (four digits) and hands (five digits). The height of a
horse's shoulder is still often measured in "hands".

Time also figured into some early measurements of distance as in the acre which
was the amount of land a team of oxen could plow in a day. The length of the
furrow in that acre was called a furlong. A journey was measured in hours (much
as it is on the L.A. freeway system today), days or moons.

\bigskip
\textit{The tools used for measuring and checking parts in a machine shop stand
ready for use on a ground granite surface plate. The parts you make will be no
more accurate than your ability to measure them.}
\bigskip

The foot is also a common unit of measurement, although deciding whose fool was
to be the standard was a problem that plagued standardization throughout the
ages. It would vary from place to place and time to time as each ruler declared
his foot to be the standard for all. The earliest preserved standard for length
comes from 2575 B.C. and is the length of the foot of a statue of Gudea, the
governor of Lagash in Mesopotamia. It was about 10.41 modern inches long (they
had smaller feet back then) and was divided into 16 parts. Later the Romans
subdivided their version of the foot into 12 unciae, hence our inches. A Roman
pace was 5 feet, and 1000 paces made up a Roman mile.

Despite the current movement to standardize the world on the metric system, my
own preference is the inch system. Mere's why. When working in metal, .001"\ (one
thousandth of an inch) is a tolerance that can be achieved with cutting tools,
and .0001"\ (one ten thousandth of an inch) is a tolerance that can be achieved
by grinding. The numbers don't come out quite so neatly in the metric system.
1\,mm equals .03937", .1\,mm equals .0039"\ and .01\,mm equals .0004". The
tolerance of $\pm$.1\,mm (.004") is too coarse for most work, and $\pm$.01\,mm
(.0004") is too fine. Therefore, you end up with tolerances too tight or too
loose because the draftsman usually calls out a tolerance of $\pm$.01 mm when it
should be $\pm$.025mm. In addition, the basic unit of distance measure, the meter,
is unrelated to any human increment, unless you are a basketball player who can
stretch his fingertips out to 39.37 inches from his nose. On the plus side for
the metric system, of course, is the fact that it is based on units of 10, and
many conversions can be done in your head. In the end it will be interesting to
see if the pure logic of the metric system can finally erase the human side of
measurement in the inch system.

\secrel{Then or now, skill with your tools is still a part of accurate
measurement}

Egyptian builders had only simple plumb lines, wood squares and rulers, but they
obviously used them with great skill, because their measurements were amazingly
accurate. The dimensions of the Great Pyramid of Gizeh, built by thousands of
workers, boasts sides that vary no more than 0.05 percent from the mean length,
That means a deviation of only 4.5 inches over a span of 755 feet. Some
construction workers of today might find it hard to duplicate that accuracy.
When Sherline's factory was being built, a laser level was set up in the middle
of the foundation to lay out the building. I noticed a workman accidentally kick
one of the legs of the level and stick it back in place by hand. If I hadn't
brought it to the attention of the foreman who releveled the laser sight, our
building might have resembled the leaning tower of Pisa today. This is a good
lesson that skill and technique can overcome some of the faults of poor
measurement tools, but carelessness or improper technique can render even the
most sophisticated tools useless.

\bigskip
\textit{A large coordinate measuring machine makes accurate part measurement
easier for the factory that can afford one. Unfortunately their price takes them
out of the reach of the home shop machinist.}
\bigskip

\secrel{Machine shop measurement today}

Until a few years ago, most factory and machine shop measuring was done with a
micrometer, height gage, squares and surface plates. In the past few years many
new methods to measure your work have been added that take advantage of the
calculating power of the computer. The most important of these new measuring
systems would be the "coordinate measuring machine". It is built on a surface
plate and can read all three axes of movement: "X", "Y" and "Z". What is
interesting about them is the methods used to come up with dimensions. They have
a small probe that very accurately informs the computer when it touches the
part. To read the diameter of a hole, the probe has to touch three points in
that hole which don't have to be equally spaced. The computer will triangulate
these points and produce a diameter within .0001" (.0025mm) and note where the
center of the hole is located on the part. The part doesn't even have to be
located square with the machine, yet it can be checked accurately. The programs
that run these machines can be as complicated as the programs that make the
part. The cost of these marvelous machines can be several hundred thousand
dollars, but they are well worth it to companies that purchase and manufacture
millions of dollars worth of precision machined parts.

\secrel{Modern scrap is more costly than ever}

This technology allows aerospace companies to build more perfect products, but
it has also filled salvage yards across the world with some very expensive scrap
parts. The problem in many cases is that manufacturer of these parts doesn't
have measuring equipment that is as good as their customer's. Who can afford
\$100,000 measuring tools to measure \$10 parts? Contract machine shops can go
broke overnight when they find out an expensive, high quantity part has been
rejected. When making machined parts for other people you have to be able to
prove your parts are correct.

When making parts for yourself, you can fit one part to another and lighten the
tolerances in the process. This is a very important difference, and if you work
at fitting and matching parts together you become a modelmaker. Modelmakers
develop a special skill to make and fit parts together so the assembly works in
unison, while the machinist will produce close tolerance parts that are within
the tolerance of the part drawing. In most cases, he doesn't have the luxury of
test fitting it to the part it will eventually be assembled to. When you make
parts for other people the drawing is "king". A machinist shouldn't have to
worry about problems someone may encounter assembling parts they manufacture.
They have enough problems making a part within the tolerances given. Toolmakers
are the best at both skills because they usually make the parts they fit
together. If you build and fit parts together you are closer to a toolmaker than
a machinist. Even though your tabletop machines are small, the processes you
will use in making your parts are the same as any machinist must deal with.

\secrel{Watch those divisions}

If you are making parts for your own enjoyment there is nothing wrong with
changing a dimension here and there to save a part that may be a few thousandths
(.lmm) undersize. Of course, if it affects the strength or integrity of the
assembly, I would not do it. When building small metal parts they have yet to
come up with a good "putting on tool"* so you have to get used to the idea of
starting over. You can't work with dimensions all day without making an
occasional error. The only people who don't make mistakes are those who do
nothing. I've

\bigskip
* NOTE: A long-standing joke in machining, a"pulting-on"lool is what you send
the new apprentice to the tool crib for when he makes his first undersize part.
\bigskip

found many mistakes are accurate to a very close tolerance, but off by a
division. When taking a reading, always read every dimension, not just the
dimension you are correcting. A good example would be when you are using a lathe
and getting close to the final cut. The diameter you are reading may need a very
close tolerance and you are concentrating on the thousandths of an inch or
tenths of a millimeter located on the barrel of a micrometer. When you start
looking only at this part of your micrometer you could screw up and get the
diameter off by one full turn which would be twenty-five thousandths (.5mm).
Mistakes of this magnitude will usually create scrap.

\secrel{Good measurement tools are one of the pleasures of machining}

I enjoy buying measuring tools. Most people treat these tools with great respect
and take pride in them. A good toolbox full of high quality measuring tools in
your workshop has the same owner's sense of satisfaction as a china cabinet full
of figurines has for your wife. I don't know if you will convince her of this
fact, but it's worth trying. I have always been a fan of the Starrett company
located in Athol, Massachusetts. Their measuring tools have set the example for
the rest of the world to follow, and they are still reasonably priced. The truth
is that you can get by with far fewer measuring tools than you will buy. Again,
I want you to understand that I'm referring to people working at this trade to
please themselves, not to please a customer.

\bigskip
\textit{Every machinist's tool box eventually collects a number of favorite
measuring tools.}
\bigskip

Measuring tools don't have to be expensive to work well. Due to the number of
imports on the market there is a great selection of measuring tools available
that all work quite well. People seem to think tools like this are a lot more
valuable than they are. I have seen people trying to sell a beat up one-inch
micrometer at a swap meet that I wouldn't even use as a welding clamp. They may
be asking more money than you could buy a set of three new ones at today's
prices.

The cost of measuring does go up considerably when you buy specialized tools. A
good one-inch micrometer could be purchased for around thirty dollars (1997),
but a thread micrometer could cost three hundred dollars. As soon as you get
away from the basic "mikes" you are in the hundred-dollar range.

\secrel{Working with tolerances}

Before getting into measuring, there is one more concept I want you to
understand and that is the concept of tolerances. Tolerances are the limits
placed on any dimension by the designer of the part. They are designed to make
the part price as low as possible while still making the part useable in every
combination with other parts of a given assembly. Almost any engineer can design
parts that will work if tight tolerances are held. These parts are almost
impossible to build and can increase the cost by a factor often. It takes a good
engineer to design parts within practical limits. The tolerances are always
given on commercial drawings, in fact, we would refuse to bid a job unless every
dimension had a tolerance when we did contract machining. You can't be arguing
with a customer about how they assumed you would know that the part you made for
them had to fit another part for which you had never seen the drawing. Making
parts that don't work can be a financial tragedy when the quantities are high
and costs are in the thousands of dollars. Before starting on any part, review
the drawing and convince yourself the part will work. When in doubt, tighten the
tolerance so you know it will work.

Tolerances should be thought of as a percentage of the dimension. For example,
.001" (.025mm) may seem like a very tight tolerance when you are turning or
boring a diameter of three inches, but it would be a two percent error on a
diameter of .050" (1.27mm). If you had the same limits for a three inch part, a
two percent tolerance would allow you a range of .013" (.33mm). These are
boilermaker's tolerances. Obviously, small parts can't be manufactured to the
same tolerances as large parts and still work. Most hobby drawings don't have
tolerances listed, and you have to decide which parts are good or bad. Don't
forget that a dimension taken on a part that doesn't have the proper surface
finish will come out undersize after it is polished. Allow for this fact.

\secrel{Technique and "feel" in using measuring tools}

A delicate "feel" must be developed to measure the small parts manufactured on
tabletop machine tools. Just because your micrometer has lines on it that
represent one ten thousandth of an inch (.0025mm) it does not mean you can
measure to that degree of accurately without that special "feel", To develop
this feel you should start measuring things of known dimension to see if you
come up with the same answer. If you have a friend that is skilled in taking
measurements, have them give you a lesson. Machinists usually own their own
measuring tools. They develop confidence in these instruments and, along with
that "feel", they are sure their dimensions are correct. You can't hold your
head high in the metal cutting trades until you can make parts to size.

\secrel{Gage readouts}

The actual readout on a measuring gage uses one of three methods to come up with
an accurate dimension: The screw thread on a micrometer, the Vernier scale on
height gages, calipers and angle reading devices or the new digital readouts
found on all types of modem measuring tools.

\bigskip
\textit{The digital readout of a modern caliper makes life easy for a machinist.
Readings in inches or millimeters can be taken and translated back and forth
eliminating many of the math errors associated with dimension conversions.}
\bigskip

The screw thread on a micrometer is forty threads per inch or has a pitch of
.025" and a calibrated movement of one inch. The metric micrometer has a .5mm
pitch and 25mm total calibrated movement. One of the reasons I don't like the
metric system is the way micrometers are calibrated. When you're using a inch
model micrometer, all you have to do to get a reading is add the reading to the
size of the gage. With metric models you have to add the range of the micrometer
to the reading. The problem is the range isn't a simple number like the inch
system; It is 25 millimeters. If you are using a micrometer with a range of 75mm
to 100mm you can't add your reading to the number one. You have to add it to 75.
This gives you an excellent way to introduce mistakes into your calculations.

Pitch is the amount of movement the screw or nut will move in one revolution.
These threads have to be precise and are usually made with a precision thread
roller.\note{A paint of interest: Although the lead screws on Sherline tools are
not ground, a precision thread roller is used to make them. They are accurate to
within 99.97\%.} Expensive micrometers will also have a screw that has been
ground. The same type of measuring assembly is used on all micrometers. They are
mounted on different gages to make a very large range of sizes. Micrometers are
available that have a range that can be expanded by changing a spacer at the
anvil end. They can be somewhat awkward to use because they feel out of balance.

\secrel{Getting consistent readings with a micrometer}

Micrometers also have a locking lever and a ratchet thimble at the end of the
barrel to allow you to apply the same pressure each time when measuring. It is
still up to the user to determine if the reading is correct. You can misread a
micrometer by trying to measure the surface of a part that isn't in line and
parallel with the anvil. If your mike is twisted on the part or if the part is
twisted, you can't get an accurate measurement. Surface finish has to be
considered when measuring. To read to four places you can mentally interpolate
between the lines or use the vernier scale that is engraved opposite the barrel,
assuming your micrometer has this scale. I have never understood why they don't
manufacture a micrometer with two inches of movement (50\,mm), particularly in
larger sizes. This would certainly cut down on the number of micrometers needed.
(Maybe I just answered my own question\ldots)

Sherline machinists will need at least a zero-to-one-inch mike (0-25mm) to
start. Calipers can take care of the larger dimensions you will come up with. As
you get more interested in machining, you will find needs for these lovely
tools, but I wouldn't purchase too many until I had decided that cutting metal
can be fun.

\secrel{Reading Vernier scales}

The Vernier scale, named after its inventor, is used on all kinds of products
that measure lengths and angles. They are quite simple to use and can be read to
.001" (.025mm). The method they use to break a scale into so many divisions
without having a line for each division is clever. For this example, let's use a
height gage that reads to one thousandth of one inch (.025mm). The basic scale
is fixed and divided into 20 parts per inch (25mm). They can be manufactured as
long as needed. Eighteen inches or 400mm would be common scales used on height
gages. Opposite this scale on the inch model is a Vernier scale that has been
divided into fifty parts mounted on the movable slide. The reason 50 divisions
are used is because 1/20" (the units into which the basic scale has been
divided) equals fifty thousandths of an inch. This provides an accuracy of one
thousandth of an inch. The Vernier scale will have only one line on its scale
that lines up with a line on the basic scale for each one thousandth inch
movement unless the setting is zero. If you set the Vernier scale at zero, you
will find that the length of fifty equally spaced marks on the Vernier scale is
equal to exactly one division less than fifty divisions on the basic scale. On
the inch model used in this example it would work out to be two inches and four
hundred fifty thousandths of an inch on

\bigskip
\textit{Vernier scales use a clever method to come up with a scale that reads to
high levels of accuracy without requiring a division on the main scale for every
increment. The reading on this scale is .867}
\bigskip

the basic scale. As the slide is raised, the two lines that line up with each
other keep getting higher in value until you get to fifty. At this time, the
zero line will also be lined up with a line on the basic scale to start the
process again. This is the only time two sets of lines will line up at the same
time.

The rules for reading Vernier scales are as follows: First you read the basic
scale where the zero line on the Vernier lies on the basic scale in whole
divisions. Then add the amount on the Vernier, which is the line on the Vernier
scale that lines up with a line on the basic scale. Remember the lines on the
Vernier scale only use the basic scales lines to line up with. You don't
consider what these lines represent on the basic scale. To accurately read a
Vernier scale you should view the lines at an angle that clearly

\bigskip
MAIN SCALE

VERNIER SCALE
\bigskip

\textit{EXAMPLE 1\ --- Here is a very simple Vernier scale and the key to how it
works. In this example, divsions on the main scale correspond to tenths,
divisions on the Vernier scale indicate hundredths. Ten divisions on the Vernier
scale are the same length as nine divisions on the main scale, so that at any
position other than zero, only one of the pairs of lines will line up exactly.
The zero of the Vernier scale is past . 2 on the main scale. The seventh line on
the Vernier scale (.07) lines up with a line on the main scale. Therefore the
reading is .2 + .07 or .27.}
\bigskip

\textit{EXAMPLE 2\ --- Above is an example like the one mentioned in the text.
Each division on the main scale represents 1/20" or .050". The small divisions
on the Vernier scale each represent. 001". The zero is just past. 45 " on the
main scale. The 22nd line of the Vernier scale or. 022 " most exactly lines up
with a line on the main scale. Therefore the reading is .450 + .022 or .472".}
\bigskip

shows which single line falls on the basic scale. I start by looking where the
zero on the Vernier scale is and make a guess where I should start. For example,
if the zero falls about half way between the divisions, I start with the
twenty-five on the Vernier. I try to look at more than one line at a time. If
you stare at a scale too long, your eyes may start playing tricks on you. This
usually happens right after you have found the part you have been working on for
the last two hours is undersize.

\secrel{Thinking about measurement in the planning stages of your project}

Before starting any complex part you should consider how you will measure it to
assure its accuracy. Sometimes the calibrated machine you are using will be more
accurate than your measuring tools, but few of us have that much faith. We know
that the only thing worse than making a bad part is sending the defective part
out or using it ourselves, so it must be checked. Once the part has been
produced it is too late if it doesn't pass inspection.

\bigskip
SHAFT

BARREL
\bigskip

\textit{EXAMPLE 3\ --- A simple micrometer barrel. Each whole number on the
shaft is .100 (inches or millimeters). Each increment between the whole numbers
on the shaft is . 025. Each number on the rotating barrel is .001. This example
reads .2 + .025 + .014 which equals a reading of .239.}
\bigskip

\textit{EXAMPLE 4\ --- A Vernier micrometer barrel. Each whole number on the
shaft is .100 (inches or millimeters). Each increment between the whole numbers
on the shaft is .025. Each number on the rotating barrel is .001. Each number on
the Vernier scale is .0001. The reading shown here is .3 + .075 + .008 + . 0008
which equals a reading of 0.3838.}
\bigskip



\secup
