\secrel{Chapter 7\ --- Measuring and measurement tools}\secdown

\secrel{Measurement throughout history}

No one knows who was the first to actually measure something, but at some point
in history, craftsmanship reached a point where doing things "by eye"\ just
wasn't good enough any more. For example, the builders of the pyramids would
start the job out with a gage carved in stone. From this, each worker would mark
his own measuring stick. The punishment for having a stick that wasn't correctly
calibrated was \emph{death}. Common sense tells me that they quickly arrived at
only one stone for a standard and that the craftsmen were pretty careful about
marking their measuring sticks. It could ruin your whole day to have to kill
your best stone mason because he got his sticks mixed up. They must have had a
number of jobs going on at the same time, and I'm sure some enterprising
Egyptian did good business selling calibrated sticks, saving everyone in the
building trades a lot of trouble.

\secrel{The modem stick and stone}

Things today really haven't changed all that much. We still need standards for
measurement and we keep coming up with ways to divide the "stick"\ into smaller
and smaller parts. The "stone"\ is now a gas that is measured to the atom
electronically, whether you use the inch or the metric system. The "sticks"\ of
today are the rulers, tapes, scales and electronic readouts we use daily.

\secrel{Measurement increments had a human touch}

You might find it interesting to note that most of the dimensions we are
familiar with in the inch system came from parts of the human body. The inch was
once based on the length of the last joint of the thumb. A yard was the distance
from the tip of the nose to the end of the fingers with the arm outstretched. (A
seamstress measures "yards of cloth"\ by pulling from nose to fingertips to this
day.) A fathom is two yards, or the distance from outstretched fingertip to
fingertip. The ancient Egyptians also used cubits (elbow to fingertips), digits
(one finger width), palms (four digits) and hands (five digits). The height of a
horse's shoulder is still often measured in "hands".

Time also figured into some early measurements of distance as in the acre which
was the amount of land a team of oxen could plow in a day. The length of the
furrow in that acre was called a furlong. A journey was measured in hours (much
as it is on the L.A. freeway system today), days or moons.

\bigskip
\textit{The tools used for measuring and checking parts in a machine shop stand
ready for use on a ground granite surface plate. The parts you make will be no
more accurate than your ability to measure them.}
\bigskip

The foot is also a common unit of measurement, although deciding whose fool was
to be the standard was a problem that plagued standardization throughout the
ages. It would vary from place to place and time to time as each ruler declared
his foot to be the standard for all. The earliest preserved standard for length
comes from 2575 B.C. and is the length of the foot of a statue of Gudea, the
governor of Lagash in Mesopotamia. It was about 10.41 modern inches long (they
had smaller feet back then) and was divided into 16 parts. Later the Romans
subdivided their version of the foot into 12 unciae, hence our inches. A Roman
pace was 5 feet, and 1000 paces made up a Roman mile.

Despite the current movement to standardize the world on the metric system, my
own preference is the inch system. Mere's why. When working in metal, .001"\ (one
thousandth of an inch) is a tolerance that can be achieved with cutting tools,
and .0001"\ (one ten thousandth of an inch) is a tolerance that can be achieved
by grinding. The numbers don't come out quite so neatly in the metric system.
1\,mm equals .03937", .1\,mm equals .0039"\ and .01\,mm equals .0004". The
tolerance of $\pm$.1\,mm (.004") is too coarse for most work, and $\pm$.01\,mm
(.0004") is too fine. Therefore, you end up with tolerances too tight or too
loose because the draftsman usually calls out a tolerance of $\pm$.01 mm when it
should be $\pm$.025mm. In addition, the basic unit of distance measure, the meter,
is unrelated to any human increment, unless you are a basketball player who can
stretch his fingertips out to 39.37 inches from his nose. On the plus side for
the metric system, of course, is the fact that it is based on units of 10, and
many conversions can be done in your head. In the end it will be interesting to
see if the pure logic of the metric system can finally erase the human side of
measurement in the inch system.

\secrel{Then or now, skill with your tools is still a part of accurate
measurement}

Egyptian builders had only simple plumb lines, wood squares and rulers, but they
obviously used them with great skill, because their measurements were amazingly
accurate. The dimensions of the Great Pyramid of Gizeh, built by thousands of
workers, boasts sides that vary no more than 0.05 percent from the mean length,
That means a deviation of only 4.5 inches over a span of 755 feet. Some
construction workers of today might find it hard to duplicate that accuracy.
When Sherline's factory was being built, a laser level was set up in the middle
of the foundation to lay out the building. I noticed a workman accidentally kick
one of the legs of the level and stick it back in place by hand. If I hadn't
brought it to the attention of the foreman who releveled the laser sight, our
building might have resembled the leaning tower of Pisa today. This is a good
lesson that skill and technique can overcome some of the faults of poor
measurement tools, but carelessness or improper technique can render even the
most sophisticated tools useless.

\bigskip
\textit{A large coordinate measuring machine makes accurate part measurement
easier for the factory that can afford one. Unfortunately their price takes them
out of the reach of the home shop machinist.}
\bigskip

\secrel{Machine shop measurement today}

Until a few years ago, most factory and machine shop measuring was done with a
micrometer, height gage, squares and surface plates. In the past few years many
new methods to measure your work have been added that take advantage of the
calculating power of the computer. The most important of these new measuring
systems would be the "coordinate measuring machine". It is built on a surface
plate and can read all three axes of movement: "X", "Y"\ and "Z". What is
interesting about them is the methods used to come up with dimensions. They have
a small probe that very accurately informs the computer when it touches the
part. To read the diameter of a hole, the probe has to touch three points in
that hole which don't have to be equally spaced. The computer will triangulate
these points and produce a diameter within .0001"\ (.0025mm) and note where the
center of the hole is located on the part. The part doesn't even have to be
located square with the machine, yet it can be checked accurately. The programs
that run these machines can be as complicated as the programs that make the
part. The cost of these marvelous machines can be several hundred thousand
dollars, but they are well worth it to companies that purchase and manufacture
millions of dollars worth of precision machined parts.

\secrel{Modern scrap is more costly than ever}

This technology allows aerospace companies to build more perfect products, but
it has also filled salvage yards across the world with some very expensive scrap
parts. The problem in many cases is that manufacturer of these parts doesn't
have measuring equipment that is as good as their customer's. Who can afford
\$100,000 measuring tools to measure \$10 parts? Contract machine shops can go
broke overnight when they find out an expensive, high quantity part has been
rejected. When making machined parts for other people you have to be able to
prove your parts are correct.

When making parts for yourself, you can fit one part to another and lighten the
tolerances in the process. This is a very important difference, and if you work
at fitting and matching parts together you become a modelmaker. Modelmakers
develop a special skill to make and fit parts together so the assembly works in
unison, while the machinist will produce close tolerance parts that are within
the tolerance of the part drawing. In most cases, he doesn't have the luxury of
test fitting it to the part it will eventually be assembled to. When you make
parts for other people the drawing is "king". A machinist shouldn't have to
worry about problems someone may encounter assembling parts they manufacture.
They have enough problems making a part within the tolerances given. Toolmakers
are the best at both skills because they usually make the parts they fit
together. If you build and fit parts together you are closer to a toolmaker than
a machinist. Even though your tabletop machines are small, the processes you
will use in making your parts are the same as any machinist must deal with.

\secrel{Watch those divisions}

If you are making parts for your own enjoyment there is nothing wrong with
changing a dimension here and there to save a part that may be a few thousandths
(.lmm) undersize. Of course, if it affects the strength or integrity of the
assembly, I would not do it. When building small metal parts they have yet to
come up with a good "putting on tool"* so you have to get used to the idea of
starting over. You can't work with dimensions all day without making an
occasional error. The only people who don't make mistakes are those who do
nothing. I've

\bigskip
* NOTE: A long-standing joke in machining, a"pulting-on"lool is what you send
the new apprentice to the tool crib for when he makes his first undersize part.
\bigskip

found many mistakes are accurate to a very close tolerance, but off by a
division. When taking a reading, always read every dimension, not just the
dimension you are correcting. A good example would be when you are using a lathe
and getting close to the final cut. The diameter you are reading may need a very
close tolerance and you are concentrating on the thousandths of an inch or
tenths of a millimeter located on the barrel of a micrometer. When you start
looking only at this part of your micrometer you could screw up and get the
diameter off by one full turn which would be twenty-five thousandths (.5mm).
Mistakes of this magnitude will usually create scrap.

\secrel{Good measurement tools are one of the pleasures of machining}

I enjoy buying measuring tools. Most people treat these tools with great respect
and take pride in them. A good toolbox full of high quality measuring tools in
your workshop has the same owner's sense of satisfaction as a china cabinet full
of figurines has for your wife. I don't know if you will convince her of this
fact, but it's worth trying. I have always been a fan of the Starrett company
located in Athol, Massachusetts. Their measuring tools have set the example for
the rest of the world to follow, and they are still reasonably priced. The truth
is that you can get by with far fewer measuring tools than you will buy. Again,
I want you to understand that I'm referring to people working at this trade to
please themselves, not to please a customer.

\bigskip
\textit{Every machinist's tool box eventually collects a number of favorite
measuring tools.}
\bigskip

Measuring tools don't have to be expensive to work well. Due to the number of
imports on the market there is a great selection of measuring tools available
that all work quite well. People seem to think tools like this are a lot more
valuable than they are. I have seen people trying to sell a beat up one-inch
micrometer at a swap meet that I wouldn't even use as a welding clamp. They may
be asking more money than you could buy a set of three new ones at today's
prices.

The cost of measuring does go up considerably when you buy specialized tools. A
good one-inch micrometer could be purchased for around thirty dollars (1997),
but a thread micrometer could cost three hundred dollars. As soon as you get
away from the basic "mikes"\ you are in the hundred-dollar range.

\secrel{Working with tolerances}

Before getting into measuring, there is one more concept I want you to
understand and that is the concept of tolerances. Tolerances are the limits
placed on any dimension by the designer of the part. They are designed to make
the part price as low as possible while still making the part useable in every
combination with other parts of a given assembly. Almost any engineer can design
parts that will work if tight tolerances are held. These parts are almost
impossible to build and can increase the cost by a factor often. It takes a good
engineer to design parts within practical limits. The tolerances are always
given on commercial drawings, in fact, we would refuse to bid a job unless every
dimension had a tolerance when we did contract machining. You can't be arguing
with a customer about how they assumed you would know that the part you made for
them had to fit another part for which you had never seen the drawing. Making
parts that don't work can be a financial tragedy when the quantities are high
and costs are in the thousands of dollars. Before starting on any part, review
the drawing and convince yourself the part will work. When in doubt, tighten the
tolerance so you know it will work.

Tolerances should be thought of as a percentage of the dimension. For example,
.001"\ (.025mm) may seem like a very tight tolerance when you are turning or
boring a diameter of three inches, but it would be a two percent error on a
diameter of .050"\ (1.27mm). If you had the same limits for a three inch part, a
two percent tolerance would allow you a range of .013"\ (.33mm). These are
boilermaker's tolerances. Obviously, small parts can't be manufactured to the
same tolerances as large parts and still work. Most hobby drawings don't have
tolerances listed, and you have to decide which parts are good or bad. Don't
forget that a dimension taken on a part that doesn't have the proper surface
finish will come out undersize after it is polished. Allow for this fact.

\secrel{Technique and "feel"\ in using measuring tools}

A delicate "feel"\ must be developed to measure the small parts manufactured on
tabletop machine tools. Just because your micrometer has lines on it that
represent one ten thousandth of an inch (.0025mm) it does not mean you can
measure to that degree of accurately without that special "feel", To develop
this feel you should start measuring things of known dimension to see if you
come up with the same answer. If you have a friend that is skilled in taking
measurements, have them give you a lesson. Machinists usually own their own
measuring tools. They develop confidence in these instruments and, along with
that "feel", they are sure their dimensions are correct. You can't hold your
head high in the metal cutting trades until you can make parts to size.

\secrel{Gage readouts}

The actual readout on a measuring gage uses one of three methods to come up with
an accurate dimension: The screw thread on a micrometer, the Vernier scale on
height gages, calipers and angle reading devices or the new digital readouts
found on all types of modem measuring tools.

\bigskip
\textit{The digital readout of a modern caliper makes life easy for a machinist.
Readings in inches or millimeters can be taken and translated back and forth
eliminating many of the math errors associated with dimension conversions.}
\bigskip

The screw thread on a micrometer is forty threads per inch or has a pitch of
.025"\ and a calibrated movement of one inch. The metric micrometer has a .5mm
pitch and 25mm total calibrated movement. One of the reasons I don't like the
metric system is the way micrometers are calibrated. When you're using a inch
model micrometer, all you have to do to get a reading is add the reading to the
size of the gage. With metric models you have to add the range of the micrometer
to the reading. The problem is the range isn't a simple number like the inch
system; It is 25 millimeters. If you are using a micrometer with a range of 75mm
to 100mm you can't add your reading to the number one. You have to add it to 75.
This gives you an excellent way to introduce mistakes into your calculations.

Pitch is the amount of movement the screw or nut will move in one revolution.
These threads have to be precise and are usually made with a precision thread
roller.\note{A paint of interest: Although the lead screws on Sherline tools are
not ground, a precision thread roller is used to make them. They are accurate to
within 99.97\%.} Expensive micrometers will also have a screw that has been
ground. The same type of measuring assembly is used on all micrometers. They are
mounted on different gages to make a very large range of sizes. Micrometers are
available that have a range that can be expanded by changing a spacer at the
anvil end. They can be somewhat awkward to use because they feel out of balance.

\secrel{Getting consistent readings with a micrometer}

Micrometers also have a locking lever and a ratchet thimble at the end of the
barrel to allow you to apply the same pressure each time when measuring. It is
still up to the user to determine if the reading is correct. You can misread a
micrometer by trying to measure the surface of a part that isn't in line and
parallel with the anvil. If your mike is twisted on the part or if the part is
twisted, you can't get an accurate measurement. Surface finish has to be
considered when measuring. To read to four places you can mentally interpolate
between the lines or use the vernier scale that is engraved opposite the barrel,
assuming your micrometer has this scale. I have never understood why they don't
manufacture a micrometer with two inches of movement (50\,mm), particularly in
larger sizes. This would certainly cut down on the number of micrometers needed.
(Maybe I just answered my own question\ldots)

Sherline machinists will need at least a zero-to-one-inch mike (0-25mm) to
start. Calipers can take care of the larger dimensions you will come up with. As
you get more interested in machining, you will find needs for these lovely
tools, but I wouldn't purchase too many until I had decided that cutting metal
can be fun.

\secrel{Reading Vernier scales}

The Vernier scale, named after its inventor, is used on all kinds of products
that measure lengths and angles. They are quite simple to use and can be read to
.001"\ (.025mm). The method they use to break a scale into so many divisions
without having a line for each division is clever. For this example, let's use a
height gage that reads to one thousandth of one inch (.025mm). The basic scale
is fixed and divided into 20 parts per inch (25mm). They can be manufactured as
long as needed. Eighteen inches or 400mm would be common scales used on height
gages. Opposite this scale on the inch model is a Vernier scale that has been
divided into fifty parts mounted on the movable slide. The reason 50 divisions
are used is because 1/20"\ (the units into which the basic scale has been
divided) equals fifty thousandths of an inch. This provides an accuracy of one
thousandth of an inch. The Vernier scale will have only one line on its scale
that lines up with a line on the basic scale for each one thousandth inch
movement unless the setting is zero. If you set the Vernier scale at zero, you
will find that the length of fifty equally spaced marks on the Vernier scale is
equal to exactly one division less than fifty divisions on the basic scale. On
the inch model used in this example it would work out to be two inches and four
hundred fifty thousandths of an inch on

\bigskip
\textit{Vernier scales use a clever method to come up with a scale that reads to
high levels of accuracy without requiring a division on the main scale for every
increment. The reading on this scale is .867}
\bigskip

the basic scale. As the slide is raised, the two lines that line up with each
other keep getting higher in value until you get to fifty. At this time, the
zero line will also be lined up with a line on the basic scale to start the
process again. This is the only time two sets of lines will line up at the same
time.

The rules for reading Vernier scales are as follows: First you read the basic
scale where the zero line on the Vernier lies on the basic scale in whole
divisions. Then add the amount on the Vernier, which is the line on the Vernier
scale that lines up with a line on the basic scale. Remember the lines on the
Vernier scale only use the basic scales lines to line up with. You don't
consider what these lines represent on the basic scale. To accurately read a
Vernier scale you should view the lines at an angle that clearly

\bigskip
MAIN SCALE

VERNIER SCALE
\bigskip

\textit{EXAMPLE 1\ --- Here is a very simple Vernier scale and the key to how it
works. In this example, divsions on the main scale correspond to tenths,
divisions on the Vernier scale indicate hundredths. Ten divisions on the Vernier
scale are the same length as nine divisions on the main scale, so that at any
position other than zero, only one of the pairs of lines will line up exactly.
The zero of the Vernier scale is past . 2 on the main scale. The seventh line on
the Vernier scale (.07) lines up with a line on the main scale. Therefore the
reading is .2 + .07 or .27.}
\bigskip

\textit{EXAMPLE 2\ --- Above is an example like the one mentioned in the text.
Each division on the main scale represents 1/20"\ or .050". The small divisions
on the Vernier scale each represent. 001". The zero is just past. 45 "\ on the
main scale. The 22nd line of the Vernier scale or. 022 "\ most exactly lines up
with a line on the main scale. Therefore the reading is .450 + .022 or .472".}
\bigskip

shows which single line falls on the basic scale. I start by looking where the
zero on the Vernier scale is and make a guess where I should start. For example,
if the zero falls about half way between the divisions, I start with the
twenty-five on the Vernier. I try to look at more than one line at a time. If
you stare at a scale too long, your eyes may start playing tricks on you. This
usually happens right after you have found the part you have been working on for
the last two hours is undersize.

\secrel{Thinking about measurement in the planning stages of your project}

Before starting any complex part you should consider how you will measure it to
assure its accuracy. Sometimes the calibrated machine you are using will be more
accurate than your measuring tools, but few of us have that much faith. We know
that the only thing worse than making a bad part is sending the defective part
out or using it ourselves, so it must be checked. Once the part has been
produced it is too late if it doesn't pass inspection.

\bigskip
SHAFT

BARREL
\bigskip

\textit{EXAMPLE 3\ --- A simple micrometer barrel. Each whole number on the
shaft is .100 (inches or millimeters). Each increment between the whole numbers
on the shaft is . 025. Each number on the rotating barrel is .001. This example
reads .2 + .025 + .014 which equals a reading of .239.}
\bigskip

\textit{EXAMPLE 4\ --- A Vernier micrometer barrel. Each whole number on the
shaft is .100 (inches or millimeters). Each increment between the whole numbers
on the shaft is .025. Each number on the rotating barrel is .001. Each number on
the Vernier scale is .0001. The reading shown here is .3 + .075 + .008 + . 0008
which equals a reading of 0.3838.}
\bigskip

You have to build parts that are known to be good before they get to inspection.
This is what machining is all about. You need a plan that doesn't allow failure
and the first line of defense is at the machine you are using to make the part.
A lead screw is very accurate, and for a short movements it would be hard to
measure the error, but the mechanical load of cutting metal makes cutters bend
and machine tools move. You can't be taking your part into inspection after
every cut because it is so difficult to mount it to the machine again exactly
the same way. I have no way of solving this problem other than making you aware
of it. If I wrote a book about the errors I have made in the machine trades you
would have trouble picking it up it would be so heavy. The errors we encounter
are very seldom the errors we expected. We have thought these through and
eliminated the possibility of these errors happening. Errors sneak up on you
from the blind side and you don't see them coming. This usually happens in
inspection where it is too late. Lay out the job so you have a plan to keep
mistakes from happening. The job of a machinist is to make accurate parts. The
inspection department will prove a part good or bad, but will never make a good
part. Again, have as much of a plan on how to measure your part as you have to
make it.

\secrel{Measuring from surface plates}

To make accurate measurements you need an accurate surface to measure from. The
surface can be the part itself. If this isn't possible, a surface other than the
part is used. They call this a "surface plate".

\bigskip
\textit{Measuring a part using a height gage on a surface plate.}
\bigskip

Surface plates are usually made of a high quality granite because the material
is very stable. The surface is ground to a fine, flat finish that doesn't
require lubrication. I usually keep the surface clean with a spray bathroom
cleaner. When the surface gets dirty with oil and grit, you increase wear
between the instrument you are using and the surface plate. Worse than that, you
start losing the feel it takes to get accurate measurements because the
instrument doesn't slide well and feels sticky. A height gage should slide
effortlessly across the surface. For the home shop I would buy a "shop grade"
surface plate that is around twelve inches by eighteen inches. The extra
accuracy of an "Inspection grade"\ plate would be hard to justify against the
extra cost for home use.

The accuracy of your surface plate may be checked with a height gage and a dial
indicator. Clamp a finger type dial test indicator in place of the scribe. A
rigid extension will improve this process because the farther the test indicator
is from the base, the more it will exaggerate the error. Set the indicator to
zero by raising or lowering the slide. Move the height gage and watch the
indicator. If the surface plate is perfect, the reading will always be zero.
Imagine a boat that is going through waves. In a calm sea the boat is always
aimed at the horizon, but in a storm the boat may be pointed above and below the
horizon. You need to have room to use a height gage and square around the part
you are checking. Diameters are easy to check with micrometers and calipers, but
checking parts on a surface plate offers a different set of problems. Now you
will be measuring the "X", "Y"\ and "Z"\ axes. Always measure the part the same
way it is dimensioned on the drawing. Don't let tolerance build-up catch you
sleeping. This happens when you are working with a liberal tolerance that may
control a tight tolerance in another location of the part. "THINK FIRST", cut
later.

\secrel{Indicators}

Indicators are neat little gadgets that come in a variety of sizes. There are
two types. One uses a rack gear which engages a gear attached to the dial
pointer. This type has a wide range of movements available; as much as four
inches (100mm), and they are accurate to less than .001"\ (.025mm). Their
accuracy is governed by the accuracy of the rack gear. As with any piece of
measuring equipment, you should determine its accuracy before using it.

\bigskip
\textit{A dial indicator like this Starrest 'Last Word"\ indicator can be used in
many ways to check both the accuracy of the alignment of your machine or the
size of your parts.}
\bigskip

Dial indicators can be set to zero by rotating a locking ring that moves the
scale rather than the pointer. They also have a wide variety of tips available.
Many have a magnetic base\note{Remember that many of the components of Sherline
tools are made from aluminum. For example, magnets will not stick to the slides,
but the lathe and mill column bed are both steel. Sherline users have also
attached steel plates to the wooden base boards to which their machines arc
mounted so that magnetic bases can be used.} attached to them so you can measure
the movements of slides or similar items that don't have calibrations. This type
of indicator may be inexpensive, but they are not needed as often as the
"finger"\ type.

The finger type is called a "dial test indicator", and the dial moves by the
rotational movement of the finger. You can get errors in the distance indicated
according to the angle of the finger in relation to the body, but the way these
indicators are used, it isn't normally a problem. These indicators are used to
square and locate. Their exact use is described in appropriate places in this
book, but here are a few tips to be aware of: Don't buy an indicator of this
type that is too accurate for the job. A "tenth"\ indicator, with one
ten-thousandth of an inch graduations, has too limited a throw to be useful.
They also bounce around so much they are difficult to read. You need a small
indicator for working with Sherline tools, and I would recommend the Starred
"Last Word"\ indicator. It is small and very stable. The examples drawn or
photographed in this book will usually show this indicator.

\bigskip
\textit{Height gages come in a wide variety of  sizes and readouts. Shown is a
dial readout, but digital readouts are also available.}
\bigskip

\secrel{Height gages}

A height gage is a vertical scale attached to a precision base and used mainly
on surface plates. It usually has a Vernier scale, but is now available with a
digital readout as well. (By the way, just because a readout is digital, it
doesn't mean it is more accurate. Know what the accuracy of digital readouts are
before you put a lot of faith in them.) A removable "scribe"\ is attached to a
slide that travels along a vertical scale. On better Vernier models the basic
scale can be adjusted a small amount to zero different probes. Reading this
scale has been previously discussed. The scribe is not pointed as we normally
think of scribes. In most cases, the point is carbide and has only been ground
from the top down. This allows the user to use the bottom surface of the scribe
to measure heights. When these measurements are taken, either the part or gage
should be moved to get the "feel"\ of how much pressure is being applied at the
point of contact. On very close tolerance dimensions, the scribe is replaced
with a small finger type dial indicator. This allows you to "zero"\ the indicator
on a known height, usually gage blocks. The reading of your height gage is taken
at zero and has to be accounted for when your measurement is taken. The
indicator should be set to zero with very little movement or pressure at the
tip. The height gage should be used in a delicate fashion in this configuration,
because a sudden jolt could move the indicator and give you a false reading.
When in doubt, check it again.

\bigskip
\textit{Simple calipers come in many shapes and sizes for measuring either
inside or outside dimensions. These have no readouts and are designed to
transfer dimensions from the part to the gage}
\bigskip

\secrel{Dial, Vernier and digital calipers}

With calipers, you have a choice of dial, vernier, and digital. They are used
for both lathe and mill work. The shape of a micrometer allows you to measure
the diameter for which it was designed. On the other hand, measuring large
diameters with calipers may be impossible because the jaws are not long enough.
Calipers are a must for your tool box. They can read either inside or outside
dimensions. Unless you are working in a very dirty environment, you will be
making a choice between dial and digital calipers. Each has its own advantage.
Dial calipers read directly as the slide is moved. There is a slight lag in
digital calipers making them difficult to preset. Digital calipers can be reset
to zero at any place on the slide, making the readings simple. They also read
both metric and inch dimensions with the push of a button. On the other hand, I
believe I can get more accurate readings with dial calipers because they have a
belter "feel".

\secrel{Re-zeroing a dial caliper}

\bigskip
\textit{When using dial calipers, make sure to clear all chips from the
measuring surfaces. Here a part is checked for size as it is turned on the
lathe. The part is held in a collet pot chuck which was machined to size to hold
it.}
\bigskip

Dial calipers have more problems with small pieces of debris than any other
type. They have a delicate rack gear driving the dial gear and will skip a tooth
if they get contaminated. This shows up when you close them completely and the
dial doesn't read zero. When a dial caliper is purchased, a small, thin piece of
shim stock shaped like the letter "b"\ is included with them to correct this
problem. You purposely put this shim into the gear train and allow the dial gear
to climb up on the shim, then move the shim and try to get the dial gear to drop
into the proper tooth on the rack gear. It is easier to keep these tools clean
so you will not have to waste time resetting them. The inside jaws can be
checked by setting a micrometer to an amount and reading the gap with your
calipers. You should get the same reading. If you don't, you will have to
consider this error when measuring. This is also a good way to develop your
skill reading calipers. Even if you are working with your own measuring tools
you should always check the tool against a standard before working on an
expensive part. This can be easily accomplished by checking the zero reading on
calipers.

\bigskip
\textit{Inside dimensions on large cylinders or between parallel surfaces are
taken with telescoping gages. This set allows spacer rods of various lengths to
be added to the micrometer head to measure distances of from 2 to 8 inches.}
\bigskip

\secrel{Inside micrometers}

Inside micrometers are very seldom used by individuals and are usually only
found in professional inspection rooms. Accurate readings can be taken with a
device called "small hole gages"\ and "telescoping gages."\ They both offer a
method of transposing a dimension to your micrometer that can then be read. You
put these devices in the hole you are checking and expand them until they are
touching both sides. On hole gages a tapered shaft expands a split ball. On the
telescoping gage a compression spring expands the gage to the maximum allowed by
the hole diameter. The telescoping gage can then be locked while the hole gage
relies on friction to maintain its size. It is very important that you develop
the "feel"\ using these tools. Rock the handle end to judge the amount of
pressure the contact points are subjected to. If it clicks past center, the gage
should be reset. When transcribing this setting to a micrometer, the same "feel"
is important to get the reading the way it was obtained at the part. You should
be able to measure hole diameters with less than .001"\ (.025mm) of error. Be
careful of tapered parts and particularly with milled cavities. End mills
deflect according to the way they are being used; that is, conventional milling
or climb milling, and it is possible to cut a "pocket"\ (square or rectangle
shape cut into a flat surface) that is bigger at the bottom than the top. By
dragging the gages up and down the measured surface, they can be checked for
taper.

\bigskip
\textit{On a depth micrometer the numbers read opposite a regular micrometer;
that is, they get bigger as the barrel is screwed in.}
\bigskip

\secrel{Depth micrometers}

Depth micrometers are handy, but to save money most home machinists will usually
use the depth rod on calipers to get these dimensions. With calipers you should
check the depth reading at zero on a smooth flat surface and compensate for any
error. It is imperative that depth measuring instruments be held square to the
surface when being used. Depth mikes are sold in sets that have interchangeable
rods for each range of the micrometer, Each rod is adjustable, but they are
usually set perfectly when they are purchased. They do not come with standards,
so you will need to come up with a method to check them. Depth micrometers read
backwards in comparison with a regular micrometer. The number gels bigger as the
barrel is screwed in.

\secrel{Optical comparators}

\bigskip
\textit{The optical comparator in Sherline's measurement room determines
dimensions that would be difficult or impossible using other methods.}
\bigskip

Comparators are optical devices and work by accurately making a shadow of the
part and portraying it on a calibrated screen, usually ground glass. The slide
movement can be measured, allowing the operator to measure the part. In many
cases the part is too small to be measured with conventional measuring tools.
You can also use them to check curves. On better comparators there are a variety
of lenses to work with. These optics must be ground correctly to eliminate any
distortion that would create errors. The operator must focus a comparator
sharply for accurate readings. These devices are not usually found in the home
workshop, but you should be aware that they exist.

\bigskip
\textit{A selection of gage pins and gage blocks make measuring a specific
dimension easier. In many cases you can make your own as you need them.}
\bigskip

\secrel{Gage pins and gage blocks}

Gage pins and blocks are another way of checking fits and checking the gages you
work with. Small holes can be very difficult to measure. The best way is to have
a pin of a known size and see how it fits. This is an item that has a spot in
all inspection rooms, although they are a little pricey for the home workshop.
You can buy these gage pins in an amazing number of sizes. They are usually
bought by the set. A set that could be useful to the miniature machinist would
be one that goes to .250"\ (6mm) and will cost around \$100 (1998). A gage pin is
included for each .001"\ starting at .050"\ (1mm). These pins are hardened and
ground to precision.

Gage blocks are sometimes called "Jo blocks"\ after the person who invented them.
They are sold in different grades and the "shop grade"\ should take care of the
needs of a home machinist because they are amazingly accurate.

When working at home, don't think you need every device I have mentioned. Before
you start a job you can make many of the gages you may need. It only takes a
moment to turn and polish a diameter to size.

\secrel{Using an edge finder to locate the edge of a part}

There are two quick methods of "picking up an edge"\ of a part on a mill. The
first is to put a shaft of known diameter in the spindle and see that it runs
perfectly true. Using a depth micrometer against the edge of the part, measure
the distance to the outside diameter of the shaft. To that dimension add 1/2 the
known shaft diameter. You now have the distance from the edge of the part to the
centerline of the spindle. Rotate the handwheel on the axis being set exactly
this distance and you will have the centerline of the spindle lined up with the
edge of the part from which you measured.

\bigskip
SPINDLE

3/8"\ SHAFT

PART

200"\ DIA. SHAFT
\bigskip

\textit{Using an "edge finder"\ to accurately locate the edge of a part.}
\bigskip

\textit{The Starred 827A edge finder has a 3/8"\ shaft which will fit in a
Sherline 3/8"\ end mill holder.}
\bigskip

The second method is much easier. It involves the use of a clever tool called an
"edge finder". These labor saving devices have been around for years and have
two lapped surfaces held together by a spring. One surface is on the end of a
shaft which fits in a 3/8"\ end mill holder and is held in the spindle. The other
is a .200"\ diameter shaft held to the larger shaft with a spring so it is free
to slide around. With the spindle running at approximately 2000 RPM the shorter
shaft will be running way off center. As this shaft is brought into contact with
the edge you are trying to locate in relation to the spindle, the .200"\ shaft
will be tapped to the center as the spindle rotates. This keeps making the .200"
shaft run continually truer. When the shaft runs perfectly true it makes contact
with the part 100\% of the time. This creates a drag on the surface of the shaft
that will "kick"\ it off center. At this point you know the part is exactly .100"
(half the diameter) from the centerline of the spindle. Advancing the handwheel
.100"\ (on a Sherline mill two revolutions at .050"\ per revolution) will bring
the edge of the part into alignment with the spindle.

It is important to use a high quality edge finder such as the Starrett 827A
shown in the previous column It must have a 3/8"\ shaft to fit the end mill
holder on the Sherline milk Metric sized edge finders are also available which
work in the same manner.

\secrel{The sine bar}

The "sine bar"\ gage is worth mentioning. It was given this name because it uses
the sine of an angle to determine how much the bar, or plate, has to be tipped
up to represent that angle. (The sine of a number and other trigonometric values
can be found in trig tables or determined with the push of a button on
scientific calculators.) The sine of the angle is then multiplied by the length
of the bar and this gives you the amount to raise the end. The base of a sine
bar is two round bars that have their centers exactly spaced apart as stated. As
one end is tipped up, the round bar rolls on the surface plate. The bar end you
are placing the calculated amount under, usually with "jo blocks", will always
rest at a point that keeps the stated distance constant, because each end will
roll the same amount. The length of the sine bar is actually the hypotenuse of a
right triangle. This distance has to be very accurate because a thirty-second
(1/120$^o$) reading on a five-inch sine bar would be less than .001". They are
used for accurately checking angles.

\bigskip
\textit{A sine bar uses the sine function from the trig tables and the length of
the bar to accurately establish an angle for milling.}
\bigskip

For example, suppose you had to measure an angle to degrees and minutes with a
tolerance of plus or minus ten minutes of a degree. Seconds are very seldom used
because a second is 1/2,296,000 of a revolution. This a very small amount and
will never be called out exactly except by astronomers or surveyors. Using a
five-inch sine bar, you look up the sine of the angle you need to set the bar to
and multiply it times five. If you don't have a Jo block set you can machine a
block to the right length and get it as accurate as you need. A set of sliding
parallels could also be used. The plates on sine bars have holes drilled and
tapped to clamp the work down with. It is imperative that the work is clamped
down square with the sine bar. A second right angle plate is used to accomplish
this. The angle can then be checked with dial test indicator mounted to a height
gage. A zero runout would be a perfect angle.

\secrel{Working to scribed layout lines}

The most common practice when working with a mill is to lay out the hole centers
and other key locations using a height gage and a surface plate. A coloring
(usually deep blue) called layout fluid or "Dykem"\ is brushed or sprayed on a
clean surface of the part. A thin layer is best because it dries quicker and
won't chip when a line is scribed. The purpose of this fluid is to highlight the
scribed line.

Don't prick punch the scribed crossed lines representing a hole center. Using a
center drill in the mill spindle and a magnifying glass, bring the headstock
down until the center drill just barely touches the scribed cross. Examine the
mark left with a magnifying glass and make any corrections needed to get it
perfectly on center. The average person should be able to locate the spindle
within .002"\ to .003"\ (.05\,mm to .075\,mm) of the center using this method.

Once the first hole is located in this manner. additional holes can be located
using the handwheels. (This is where the optional resettable "zero"\ handwheels
are handy.) Now the scribed marks are used as a double check and the handwheels
take care of the accuracy. Don't forget to account for the backlash of the lead
screw by always turning the handwheels in the same direction as you go from one
point to the next. More on backlash can be found on page 258, and an example of
laying out a simple hole pattern using trigonometry and your mill's handwheels
is included on page 323.

\bigskip
\textit{Machining to layout lines on your part is a good way to get close to
your desired dimension. Final dimensions are achieved through keeping track of
your handwheel inputs and double checking with accurate measurement toots.}
\bigskip

\textit{A digital readout makes a machinist's life easier. Much like air
conditioning on a car or a microwave oven in the kitchen, once you 've gotten
used to it you wonder how you ever got by without it. Now tabletop machinists
using Sherline mills can benefit from the same hardware that has become so
popular on full size mills by adding a D. R. O. to their mill. It reads out to
three decimal places and resets to zero with the touch of a button. The readout
also indicates spindle RPM at all times. Backlash can be compensated for to a
half a thousandth of an inch. No more counting handwheel rotations. Just set the
axes to zero and crank in the desired table position.}
\bigskip

\secup
