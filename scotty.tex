
\secrel{Scotty Hewitt\ldots as much an artist as an artisan}

\textit{One of Scatty Hewitt's fun projects\ --- a 1930's air powered race car.
For more on Scatty, see the next page.}
\bigskip

Scotty Hewitt of Van Nuys, California is a relative newcomer to miniature
machining. Scotty spent his life as a race car driver, but miniature tools
always held a fascination for him. After seeing a Sherline display at a hobby
show he decided to take up the hobby. Joe traded him a Sherline mill in exchange
for some lessons in driving a race car. Since then his skills as a machinist
have improved much faster than Joe's have as a race car driver, but a chance
meeting took each of their lives in a new and fun direction.

Scotty's work is the perfect example of a craftsman developing an artistic style
of his own. The toy-like quality of his models combines finely machined metal
parts with brightly colored, hand carved wooden bodywork on his race cars. His
marine engines are also nicely displayed on wooden bases or in models of skiffs
and tugboats. The appeal of his work has been put to the test in Sherline's
Machinist's Challenge contest at the North American Model Engineering Society's
show in Wyandotte, Michigan. As judged by the show's spectators, Scotty's work
won first place for three years in a row.

The projects Scotty produces are not copies of other people's work nor are they
built from standard kits or plans. They are uniquely his own. Scotty builds not
just with his hands or with his brain but with his heart as well. He is a
machinist, but more, he is also an artist.

\bigskip
\textit{This bright red racer won 1st place in the 1996 Machinist's Challenge.
Probably not the best machined or most complicated entry in the contest that
year, it did have a special appeal that spanned the range of spectators who do
the voting on the contest entries. Just about everybody who voted gave one of
their five votes to this car.}

\textit{Building things that are fun to display is part of the joy of tabletop
machining. You don't have to be a machinist to appreciate a project like this.
When friends or grandkids see it on your display shelf you get the added
satisfaction of being able to say, "I made it myself."}
\bigskip

\bigskip
\textbf{Project: 4-cyl. oscillating marine engine}

\bigskip
\textit{(Left) One of Scotty's first winning projects was this marine engine
which won first place at the N.A.M.E.S. contest in 1994. It features a throttle,
lubricator, drain valve and forward and reverse mechanism which are all fully
functional.}

\bigskip\textit{(Above) In 1995, Scotty made the engine in a much smaller size,
displayed it in a model of a wooden skiff and took first place again.}

\bigskip
\textbf{Project: 5"\ Steam tugboat}

\bigskip\textit{This little tugboat is another good example of effective display
and features a fully functional marine engine and boiler made from brass. It
finished 3rd place in the 1996 N.A.M.E.S. contest.}

\bigskip
\textbf{Project: 5"\ $CO_{2}$ powered vintage racer}

\bigskip\textit{Under the hand carved wooden body is a complete frame and
$CO_{2}$ engine that drives the rear wheels. The hand carved driver also adds a
lot of character to the project and shows another facet of Scotty's skills as a
modelmaker. The detailed display background makes the car seem much more real.
Except for the quarter in the foreground, the photo looks almost as if it could
have been taken at a race track in the 1930's.}

\bigskip\textit{Scaled-down machine tools and complete miniature shops are
always popular subjects for machinists. Notice the details like charts and
clocks on the wall.}

\bigskip\textit{These two excellent shops were seen at the North American Model
Engineering Society (N.A.M.E.S.) Exposition in Wyandotte, Michigan. Both date
from the days when machines were driven by a system of overhead pulleys and
belts.}

\bigskip\textit{Mike Foti  is  a young man in his 20's hailing from Hillsboro,
Oregon. He designed and built this American LaFrance hot rod fire truck, forming
the bodywork by hand from brass and soldering the pieces together.
All the small parts are machined. It was built to compete in the hot rod
division of the national modeling contest in Salt Lake City for 1999. Mike is
among the new generation inspired by the work of master modeler Augie Hiscano.
(See profile on page 180.) Although he didn't beat Augie, he got some good tips
from "the Master"\ and will be back with even hotter projects in the future.}

\bigskip\textit{Robert Shipley of Knoxville, Tennessee designed and built this
case to display an old clock from the dashboard of a 1920's automobile. The case
is detailed identically on the back where the works of the clock can be seen
behind the glass cover. In addition to providing protection, this case turns a
simple clock into an impressive display.}
