
\secdown

\secrel{\en{A special note to engineers reading this book\ldots}\\
\ru{Специальное примечание для инженеров, читающих эту книгу\ldots}}
\secdown

\secrel{\en{Machining for engineers and engineering for machinists}\\
\ru{Мехобработка для инженеров и инжиниринг для станочников}}

\begin{enen}
At first glance the subtitle on the cover of this book could be a bit deceiving.
What does tabletop machining have do with engineering you may ask? Compare it to
a book that has been written about the ocean. The seas could be described from
the perspective of a young man who has just sailed around the world in a
twenty-five foot sailboat or by a merchant seaman who has spent his career
aboard a giant ocean liner. Each would have an entirely different view of what
the ocean was all about. In a storm, the chap in the small boat would write
about surviving broken masts and mountainous seas while the merchant seaman
might write about seasick passengers. I believe you would learn more about the
ocean from the young man in the small boat, because in a sense he was more
involved in his subject. He was not just on it, he was \textit{in} it.
\end{enen}

\begin{ruru}
На первый взгляд подзаголовок на обложке этой книги может быть немного обманчив.
Вы можете спросить, как настольная обработка связана с инжинирингом? Сравним с
книгой, написанной об океане. Моря могли бы быть описаны с точки зрения молодого
человека, который только что ходил по всему миру на парусной лодке длиной
двадцать пять футов, или моряком торгового флота, который провёл свою карьеру на
гигантском океанском лайнере. Каждый из них будет иметь совершенно разную точку
зрения об океане. В шторм парень на небольшой лодке написал бы о выживании со
сломанными мачтами и волнах размером в гору, в то время как моряк торгового
флота мог бы написать о морской болезни пассажиров. Я думаю, что вы бы узнали об
океане больше от молодого человека на небольшой лодке, потому что некоторым
образом ему было сложнее в своём путешествии. Он был не просто в путешествии, он
был \textit{внутри} него.
\end{ruru}

\secrel{\en{Navigating the seas of machining}\\
\ru{Навигация по морям механической обработки}}

\begin{enen}
The ocean in this case is the world of machining. The craftsman using tabletop
machine tools is like the sailor in a small boat, while the professional
machinist with his big CNC shop tools is like the world-traveling seaman. The
process of producing complex, accurate parts cannot be described by looking in
the window of a quarter million dollar CNC machine. It would be like a merchant
seaman working in the engine room trying to describe a storm in the Atlantic
Ocean by telling you how much extra fuel the ship used. The professional's view
of the subject may be so cluttered with details that it is difficult to sort the
things you really need to know to sail in rough seas or make good parts. It is
the craftsman working with small tools, turning the cranks by hand, who will
have the most to tell you about the real world of working with metal.
\end{enen}

\begin{ruru}
Океан в нашем случае\ --- мир механической обработки. Мастер использующий
настольные станки, как матрос на маленькой лодке, в то время как
профессиональный станочник с его большим магазином инструментом на станке с ЧПУ,
как моряк несколько раз совершивший кругосветку. Процесс производства сложных,
точных деталей не может быть описан описан, глядя в окно ЧПУ станка за четверть
миллиона долларов. Это было бы похоже на моряка торгового судна, работающего в
машинном отделении, который пытается описать бурю в Атлантическом океане,
рассказывая вам, сколько дополнительного топлива использовал корабль.
Представления специалиста о предмете разговора может быть так наполнено
тонкостями, что трудно отсортировать вещи, которые вам действительно нужно
знать, чтобы плавать в бурных морях, или делать хорошие детали. Если этот мастер
работает с мелкими инструментами, вытачивает шатуны вручную, у него больше
возможностей рассказать вам о реальном мире работы с металлом.
\end{ruru}

\secrel{\en{Looking ot engineering from the craftsman's perspective}\\
\ru{Инжиниринг с точки зрения мастера}}

\begin{enen}
With the aid of computers, parts can easily be drawn that can't be built. CAD
programs allow a designer to put a perfect .0001"\ radius on the inside comer of
a pocket cut in tool steel. Hopefully after reading this book you will not ask a
toolmaker to do it, but if you do, you'll at least know it is going to cost a
great deal of money to try. Working with metal is far more difficult than one
would imagine. A false impression is gained by looking at the beautiful yet
inexpensive machined parts that we deal with daily. They have been produced in
very large quantities, and that five-dollar part you may consider a "rip-off
could easily cost five hundred dollars if you had to manufacture just one. New
engineers will often think a toolmaker is a failure when the seemingly simple
part they design ends up costing a thousand dollars to make. Most engineers will
eventually have to deal with the craftsman who turn their ideas into reality,
and in reading this book I would hope you come away with a new perspective of
what is really involved in producing a machined part or a product. An alternate
subtitle for the book might have been "Things they should have taught you in
engineering school but didn't". This book might be considered your textbook for
a course called "Reality 101".
\end{enen}

\begin{ruru}
Используя компьютер, очень легко смоделировать детали, которые не могут быть
изготовлены. САПР программы позволяют проектировщику поставить идеальный радиус
с точностью 0,0001\,мм на внутреннем ребре кармана, отфрезерованного в
инструментальной стали. Надеюсь после прочтения этой книги вы не будете просить
инструментальщика сделать такой элемент, но если вы это сделаете, вы по крайней
мере будете знать, что это будет стоить много много денег, даже только чтобы
попробовать это сделать. Работа с металлом является гораздо более трудной, чем
можно себе представить. Это ложное впечатление сложилось, глядя на красивые, но
все же недорогие детали, сделанные на станках, с которыми мы имеем дело
ежедневно. Они были сделаны серийно в очень больших количествах, и цена в 5
долларов за деталь, которую вы можете почитать надувательством, легко может
взлететь до 500 долларов, если вам нужно сделать одну такую деталь. Начинающие
инженеры часто думают что для инструментальщик плох, когда казалось бы простая
деталь в итоге оказывается с ценой в тысячи долларов. Большинству инженеров в
конечном итоге придется иметь дело с мастером, который может превратить их идеи
в реальность, и вы читаете эту книгу, я надеюсь, чтобы сформировать новую точку
зрения на то, что на самом деле происходит при обработки детали или продукта.
Альтернативным названием для книги могло бы быть "Вещи, которым вас должны были
научить в ВУЗе". Эта книга может считаться вашим учебником по курсу с названием
"Реальность 101".
\end{ruru}

\secrel{\en{Seeing production from the point of view of both the engineer and
machinist}\\
\ru{Производство с точки зрения инженера и станочника}}

\begin{enen}
My perspective on machining could be considered unique because, in order to
survive, I have had to deal with every aspect of product design from engineering
to prototyping to tooling to manufacturing to sales. In this book I have tried
to pass along the logic I used to solve the associated problems. Understanding
how a craftsman thinks and works is an essential part of getting projects done.
Unless you are willing to build your designs yourself, you are going to have to
learn how to deal with the craftsman who will actually build them. The more you
know about their methods, personalities and unique problems, the better your
chances are for success. Smooth sailing.
\end{enen}

\begin{ruru}
Мой взгляд на мехобработку можно считать уникальным потому что, для того чтобы
выжить на рынке, я имел дело с каждым аспектом дизайна продукта от
проектирования и прототипирования до производства и продаж. В этой книге я
попытался объяснить логику моих решений, связанных с этим проблемами. Понимание
того, как мастер думает и работает, является неотъемлемой частью получения
готовых проектов. Если вы не готовы изготавливать свои проекты самостоятельно,
вы все равно сможете узнать, как общаться с мастером, который будет на самом
деле их делать. Чем больше вы знаете о его методах, личных и уникальных
особенностях, тем выше ваши шансы на успех. Счастливого плавания.
\end{ruru}

\bigskip
\begin{enen}
---\ Joe Martin
\end{enen}

\begin{ruru}
---\ Джо Мартин
\end{ruru}

\secup
\secup
