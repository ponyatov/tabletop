
\textit{Under the glass watch crystal is a 1/30 scale, 1/4-20 threaded hex bolt
and nut along with a machined wrench to fit it. Next to it is a full size 1/4-20
bolt and nut for comparison. To learn more about a man who takes miniature
craftsmanship to extremely small scale read the following profile on Jerry
Kieffer.}

\secdown

\secrel{JERRY KIEFFER\ldots Craftsmanship to the smallest detail}

\textit{(Left) Jerry's Corliss steam engine is done in 1/30th scale\ldots
down to the smallest 1/4-20 bolt! It runs as well as it looks.}
\bigskip

\textit{(Above) This "banjo oiler"\ is typical of the model's detail.}
\bigskip

Jerry Kieffer is not a professional machinist. He is a marketing representative
for a utility company in Wisconsin. In fact, less than 10 years ago he had just
about no experience with miniature machine tools at all. But he is an excellent
craftsman who is good with any kind of tool and loves to make very small
projects, so getting his hands on precision miniature machine tools opened up a
whole new world of enjoyment for him.

His interests are wide ranging\ldots from repairing watches and guns to making
miniature engines and tools. Jerry delights in making incredibly small parts.
Some of them can only be appreciated through a magnifying glass. If a friend
makes a steam engine in the smallest scale possible, Jerry will make a half size
model of his friend's model! He spends several hours each evening in his shop
and the amount and quality of the work he has produced is truly amazing,
especially when you consider the level of perfection to which it is done.

When Joe began making Sherline tools, Jerry is exactly the customer he had in
mind. He uses inexpensive but accurate tools to make projects of great detail
and beauty. For anyone who asks, "are tabletop machine tools accurate enough for
what I want to do?", we simply pull out pictures of Jerry's work. Nothing more
needs to be said.

\bigskip
\textit{Jerry's shop is neat and simple. The magnifying loop on his glasses is
necessary for working on and fitting the minuscule parts he enjoys making.}
\bigskip

\textit{A "monkey-wrench"\ in 1/7 scale.}
\bigskip

Project: Stover CT 2 Hit 'n Miss Engine
\bigskip

\textit{This running Stover "hit 'n miss"\ engine is built in 1/6 scale and has
a 1/2"\ cylinder bore. Though it may not look it, Jerry says this is one of the
most difficult projects he has completed to date.}
\bigskip

Project: Harley Davidson Cylinders
\bigskip

\textit{Anyone who knows Harleys will recognize these right away. These
cylinders for a 1/6 scale 1947 Harley Davidson "knucklehead"\ motorcycle were
built to test the feasibility of a project to build a complete scale motorcycle
that will not only run, but will "sound like a Harley"\ when if does. Jerry also
plans to build a "repeater watch"\ from scratch. His goals are as big as his
parts are small, although 1 don V have the slightest doubt that if he decides to
take on a challenge, he will find a way to do it.}
\bigskip

Project: Miniature Spark Plugs
\bigskip

\textit{These working spark plugs are from top to bottom: a 1/4 scale Renz 775
spark intensifier plug, a 1/6 scale Champion "33"\ plug and a 1/4 scale Champion
primer plug.}
\bigskip

Project: Miniature nut and bolt
\bigskip

\textit{This is the one that really blows people away! The hex bolt has a .010"
shaft. The threads are 354 T.P.I. and .0005"\ deep. A hex nut is threaded onto
it. Next to the bolt is a wrench with milled contours. Compare the bolt to the
screw heads in the lady's wristwatch to the right which are the smallest
fasteners commercially available. Jerry says making the nuts and bolts is easy
once you have the taps and dies\ldots it was figuring out how to make the taps
and dies that was the hard part.}
\bigskip

\secrel{A Flying Pendulum Clock by Jerry Kieffer}

This is a variation of the "Ignatz"\ clock. It is called a "flying pendulum"
clock because the ball on the string wraps around first one post and then swings
a pivot 180$^o$ to wrap the string around the other post, then back again. It is
powered by a key-wound main spring. Although this type of clock doesn't keep
very good time (accurate to within about 10 minutes a day), it is fun to display
because people are always fascinated by its unique movement.

As a future project I would like to produce a kit with plans and materials so
that others can make this clock. It will include a video showing Jerry Kieffer
building it so you can see not only the steps involved, but also how a superb
craftsman plans and executes a job.

\bigskip
\textit{Mainspring, barrel, key and components of the winding mechanism.}
\bigskip

\textit{Glass or plastic domes are a good way to protect your project. This is a
traditional way to display clocks where the works are to be exposed.}
\bigskip

\textit{Jerry made this nicely displayed set of staking tools to make it easier
to press shafts and pins into gears.}
\bigskip

\textit{This is one of Jerry's setups using the mill and one of his special
staking tools to press-fit the pins which act as gear teeth on the third pinion
gear.}
\bigskip

\secrel{Chapter 1— Lathe workholding}\secdown

\secrel{CHUCKS}\secdown

For holding work on a lathe you have the obvious choice of using a 3-jaw or
4-jaw chuck. Sherline also offers a self-centering 4-jaw chuck and an assortment
of collets. There is a safe limit to how much these chucks can be tightened
without damaging them. If the "tommy bars"\ are being bent while tightening a
chuck and the work is still coming loose you have a problem that can't be solved
by tightening the chuck to its breaking point. A 3-jaw chuck is really designed
to hold only round work. It has the advantage of being able to clamp material
that is not perfectly round with the principle being the same as a three legged
stool. Remember, if the stock isn't perfectly round it would be impossible to
have it turn perfectly true. The method you use to clamp a part in a chuck can
have a profound effect on how true it runs.

\secrel{Clamping up work in the chuck}

With any clamping device, you should spin the work with your fingers as it is
tightened. The part should feel as though it is being held securely before the
final tightening is done. Don't insert the work into the chuck until the jaws
have been adjusted to their approximate position. This keeps the work, or tool
in the case of a drill bit, from getting caught between two of the three jaws.
This is especially true for the expensive chucks that have the jaws rotate as
they are tightened. I don't like them because of this fact, and the extra cost
to have a chuck run .001"\ or .002"\ truer just doesn't seem like a good
investment for the home machinist.

\bigskip
\textit{3-jaw, 4-jaw and tailstock chucks of various sizes for use on Sherline
tools or any tool with a 3/4-16 spindle nose thread. Other threads are available
to allow the chucks to be used on some Unimat, Sears and Cowells lathes. Arbors
are also available to adapt them for use on watchmakers lathes. The 2.5"\ 3-jaw
chuck in the middle of the bottom row is shown with its jaws reversed for
holding larger work.}
\bigskip

\secrel{Factors affecting chuck "runout"}

The chuck should run within .003 T.I.R. (Total Indicated Runout). This is the
amount of variation an indicator would read if the work were rotated with the
spindle. Getting a chuck to run perfectly true is difficult to accomplish
because so many things come into play. The threads have to be true and clean
along with the shoulder against which the chuck seats. If a part is accidentally
ripped out of the chuck during machining both the spindle and the chuck can be
damaged.

\secrel{The design of Sherline 3- and 4-jaw chucks}

The 3-jaw self centering chuck is the most popular of all the accessories
available for the Sherline lathe. It is available in both 2.5"\ diameter and 3.1"
diameter. These chucks will grip round or hexagonal work quickly since the jaws
move simultaneously to automatically center the work being held. The jaws on the
chuck are designed so that the same chuck can be used for both internal and
external gripping. Jaws are reversible for holding larger diameter work, Due to
the nature of the design of a 3-jaw chuck, it cannot be expected to run
perfectly true. Even 3-jaw chucks costing five times more than the one made for
this lathe will have .002"\ to .003"\ runout. If perfect accuracy is desired in a
particular operation, the use of a 4-jaw chuck is recommended. Each jaw is
adjusted independently so parts can be centered with total precision. Both a
2.5"\ and 3.1"\ 4-jaw chuck are available for the Sherline lathe. A self-centering
4-jaw chuck is also available.

\secrel{Capacities of the 3-Jaw chuck}

3-jaw chucks provide the quickest and easiest way of holding work in the lathe.
The Sherline 3-jaw chuck is designed so that it can be used to clamp externally
on bar stock or internally on tube stock. The 2.5"\ chuck is designed to grip
from 3/32"\ (2mm) to 1-3/16"\ (30mm) diameter stock with the jaws in the normal
position. The 3.1"\ chuck handles stock up to 1-1/2"\ (38mm) in diameter. For
larger diameter work, the jaws must be reversed. (See photo on previous page.)
The reversible jaws can grip to 2-1/4"\ (56.0 mm) for the 2.5"\ chuck and up to
2.75"\ (70 mm) for the 3.1"\ chuck. The chucks have a .687"\ (17mm) diameter
through hole with a 3/4-16 thread.

\bigskip
\textit{The Sherline 3-jaw chuck with 2 tommy bars for tightening the jaws to
hold a workpiece.}
\bigskip

Jaw Location

2st

2nd

3rd

Punch Mark

Standard Jaw Identification
\bigskip

\textit{Three-Jaw Chuck, standard jaw locations.}
\bigskip

\secrel{Precautions when using a chuck}

NOTE: DO NOT TURN THE LATHE SPINDLE ON UNTIL THE CHUCK IS TIGHTENED. The
acceleration of the spindle can cause the scroll to open the chuck jaws if not
tightened!

To prevent permanent damage, a 3-jaw chuck should only be used to hold finished,
turned or drawn stock. For rough castings, etc, use the 4-jaw chuck.

DO NOT OVERTIGHTEN THE CHUCK. Use only moderate pressure with the Tommy Bars
supplied.

\secrel{Reversing the chuck jaws on a 3-faw chuck}

Always start with position "A". (See illustrations above and on next page.) To
reverse the chuck jaws, rotate the knurled scroll until the jaws can be removed.
They can be easily identified by the location of the teeth to the end of the
jaw. To maintain chuck accuracy, the 2nd jaw must always be inserted in the same
slot even when the jaws are reversed. This slot was identified by a punch mark
next to the slot. Newer chucks are engraved with the letters A, B and C as
shown. Always insert the jaws in the order and at the location shown on the
drawings. Turn the scroll counterclockwise when viewed from the face of the
chuck until the outside start of the scroll thread is just ready to pass the
slot for the 1st jaw. Slide the 1st jaw as far as possible into the slot. Turn
the scroll until the 1st jaw is engaged.

\secup

\secup

\secup
