\secrel{Chapter 4\ --- Processes for metalworking}\secdown

This chapter contains information on the processes used to harden, plate,
finish, cast and join metals. While some of these processes can be done in the
home shop, a number of them are best sent out to a vendor as they require
equipment and processes that are not practical to set up in a home shop.
Understanding these processes, whether you attempt them yourself or not, is an
interesting part of working with metal.

\secrel{4.1\ --- HEAT TREATING}\secdown

\secrel{Some "do-it-yourselfers" have more success than others}

Tve got to be honest with you, heat treating is something I know less about than
I'd like to. After investing three days labor in a part only to destroy it
trying to heat treat it myself, I decided to have contract heat treating
companies take care of the small amount of heat treat work I generate. If you
have a small heat treating furnace available capable of heating up to
$2000^{o}F$ you can do simple heal treating jobs. If you plan to make parts that
require heat treating I would use "air hardening" steels such as A-2 for
standard use or S-7 for parts subject to shock loads. Don't consider using a
torch unless you're treating a very small part made of drill rod. It is very
important that the type of steel you use is documented so that proper heal treat
information is used.

\bigskip
\textit{A heat treating oven heats items to he hardened almost to their melting 
points making temperature control critical.}
\bigskip

\secrel{Case hardening vs. tool steels}

Tool steels harden all the way through and case hardened steels are only
hardened on the outer surface of the part. This gives a very hard surface that
can be controlled by the time it is heated to harden to a specific depth. The
part is brought to temperature in a cyanide bath where carbon atoms change the
makeup of the surface to allow it to be hardened. There is a limit to the depth
of this process (up to .010" or .25mm). but case hardening is inexpensive in
production quantities. You can case harden low carbon steels in a home workshop
with a marvelous material called "Kasenite". After the part has been brought to
about $1650^{o}F$; it is dipped into the powdered Kasenite. The powder adds
carbon atoms to the surface, allowing the surface to be hardened. You can get
more depth with this process by repeating it several times.

\secrel{Hardening your own parts}

The part should be placed in a stainless steel "envelope" to protect it from
excess oxidation. The stainless is 316 shim stock which is .002" thick and goes
by the trade name of "Tool wrap". It is very thin and can be easily folded. The
envelope should be as small as possible and sealed with a sharp fold. Stainless
steel wire can also be used to hold the envelope in place. A heat treat furnace
is a must and the temperature must be accurately controlled. Be sure the work is
allowed to "soak" at the recommended setting so that the entire part has time to
be brought to the proper temperature. A general guideline is one hour for each
inch of thickness. The makeup of the material you arc dealing with in heal
treating must be known to properly heat treat metals, and these rules must be
followed exactly. The temperatures used in heat treating arc very close to the
melting point of the material. The parts being heat treated must be heated
evenly. An acetylene torch used carelessly may "burn" the corners off while
never getting the center of the part hot enough to be hardened.

\secrel{Rockwell testing for hardness}

After the part has been hardened it is reheated to "draw" the part to a specific
hardness. The lower the drawing temperature the harder the part. As steel gets
harder it also becomes more brittle. The numbers arrived at will usually be a
compromise. There are several ways to determine the hardness, the most popular
being "Rockwell" testing. A diamond point is pressed into the surface of the
hardened part and the amount the diamond penetrates the surface at a given
pressure can determine the hardness.

\bigskip
\textit{A Rockwell tester is used for determining the hardness of a material by
measuring the penetration of a diamond point into the material's surface.}
\bigskip

If you ever have the time, look up the ingredients in a good steel and compare
these ingredients to stainless steel. The ingredients may be close, but the
final product is entirely different. In most cases, stainless can't rust and is
non-magnetic. I would recommend working with popular, common materials\ldots
nothing exotic. Talk to your local heat treat company for suggestions.

\secrel{Annealing is the opposite of hardening}

The term "annealing" is used frequently and is basically the opposite of
hardening ferrous steels. As an example, materials that have been work hardened
by machining may have to be annealed after being "roughed out" to size.
Annealing is usually accomplished by heating the part to a temperature
determined by the composition of the material and allowing it to cool slowly.
Nonferrous metals can be annealed by heating them to the desired temperature and
quenching them in water.

\secrel{Hardening and aging aluminum}

Aluminum can be hardened at much lower temperatures than ferrous materials. A
period of several days after heat treating may be required to get to the
required hardness. This is called "aging".

\secup

\secrel{4.2\ --- METAL FINISHES}\secdown

\secrel{Black oxide}

There are a variety of finishes that can be put on metal. A popular finish for
steel parts is "black oxide". This is basically "gun bluing" and a gun shop
would be a good place to purchase small amounts of the material needed to do
this at home. Fingerprints that haven't been removed can leave blemishes on the
final product in all plating processes. You have to work in a very clean way in
a very messy process to do plating. If you have a plating company do a small job
for you, it may save money if you don't demand fast service so they can run the
part with a larger order.

\bigskip
\textit{These parts have a tough black oxide finish. This finish is often used
on guns as it offers good protection as well as a nice looking blue-black
color.}
\bigskip

\secrel{Attempting plated finishes at home}

There aren't many other plating methods that can be done without building
special equipment. The process of plating is more electrical than chemical. I've
seen some interesting articles over the years on how to build plating tanks and
rectifiers to accomplished plating at home. Some craftsmen prefer to control the
quality of every step of their project. I believe the average machinist would be
heading in a direction with little rewards if they try plating at home. The
chemicals are difficult to buy in small quantities. It would be like developing
your own film if you only used one roll of film a year.

\bigskip
\textit{Anodized finishes like the black of the table and the red of the
handwheels provide a tough, easy-to-care-for finish on aluminum parts. In this
case, they also provide the additional benefit of helping the laser engraved
markings show up in high contrast. The laser cuts through the finish into the
raw aluminum underneath making what appears to be white lettering. Several other
colors of anodized finish are also available. The process gives a good looking
finish and is not particularly expensive.}
\bigskip

\secrel{Chromium plating is a multistep process}

Decorative chrome plating like you see on automobile bumpers doesn't work well
for small machined parts. The process has several steps and the first step is to
plate copper on to the steel surface. The copper is polished and a plating of
nickel is then added to the surface. Nickel has a "smoothing" effect to the
surface but also has a slight yellow color. Chrome is then plated on the nickel
in very small amounts which gives you a beautiful finish for bumpers, but it is
not too good for precise mechanical parts. This is because the chroming process
is done in layers with lots of polishing, and detail can be lost in small parts.
Plating also builds up more on corners than flat surfaces, and chrome is so hard
that it can't be machined off. A tapped hole on a part that has been plated may
cause problems even if you use an oversized tap. The first thread may build up
with plating material and the remaining threads will not be plated. Plating
doesn't get into holes unless special electrodes are used. A tapped hole that
will not accept a standard machine screw may be the result in this type of
plating process. Remember, a polished steel surface can have the appearance of
chrome if it is done properly.

\bigskip
\textit{This 1/12 scale model of a blacksmith's trip hammer by Jerry Kieffer
remains unpainted to show that no fillers were used or mistakes covered up.
Sometimes the ultimate finish is no finish at all.}
\bigskip

\secrel{Hard chroming has specific uses}

There is another chrome plating process called "hard chrome". This is expensive
and doesn't give the shiny surface usually associated with chrome. The process
is controlled by carefully positioning the part and electrode to control where
the chrome is applied. This method is used to create a good wear surface. Molds
that are subject to abrasive materials have their cavities hard chromed. Hard
chrome has to be polished or precision ground. It is also used as a "putting on
tool" to save a worn or undersize shaft, but the home hobbyist may find the cost
of doing this too expensive. When one-off parts are hard chromed you can expect
the cost to be over one hundred dollars.

\secrel{Barrel plating large quantities of small parts}

Barrel plating is a production method used on small parts. Screws would be a
good example of this type of plating. The parts are not "racked" one at a time
but are plated by the barrel as the name implies. Nickel and cadmium finishes
are commonly used with this process.

\secrel{Colored anodized finishes on aluminum}

Anodizing is the preferred method of finishing aluminum. This is a method that
forms aluminum oxide on the surface. Parts will have their surfaces grow
slightly with this process. The oxide finish has a "porous surface". This allows
it to accept colored dyes with excellent results. Black and red are two colors
that anodize well. The surface is then "sealed" with hot water, but please don't
ask me how that works. You should realize that colors will vary because of the
many variables that take place in the process. It is not an exact science unless
you are willing to pay for new chemicals and test runs. I would not bid a
contract part if I had to be responsible for the plating color. Many perfectly
good parts have found their way to the scrap heap because a buyer didn't like
the color. This is one of the reasons I gave up contract machining. There are a
hundred ways to lose, and you only win if everything goes exactly as planned.

\secrel{Hard anodizing}

Just like hard chroming, the anodizing process has hard anodizing. This allows
the surface to be built up with one of the hardest materials on earth, aluminum
oxide. It can be used to protect the surface against wear; however it isn't very
strong and can only be applied a few thousandths deep. One half the thickness of
the plating will be impregnated into the metal and the other half will be added
to the surface. As an example, if a 1" diameter bar were hard anodized .002",
the bar's diameter would increase by .002". This is somewhere around the maximum
amount you can hard anodize. If machining is attempted on a part that has been
anodized, enough material should be cut on the first pass to keep the cutter
from rubbing on the anodized surface. Carbide tools will help but the oxide
surface is still harder than carbide.

\bigskip
\textit{Though not particularly exotic and certainly not expensive, a nicely
done painted finish does add realism and appeal to a model. An airbrush will
give the most control, but good technique with a spray can yields good results
too.}
\bigskip

\secrel{There's always good old paint}

Painted finishes need special attention when working on miniature replicas of
machines. You just can't slap on a couple of coats of paint and be satisfied
with your work. You should carefully mask the surface to keep paint off the
areas where it shouldn't be. A primer should be used to assure the proper
adhesion of the paint. This should be applied in very light coats. I have always
preferred lacquer based paints because a surface can be built up with many coats
to get the desired results. The inside corners must be painted using very light,
dry coats or the lacquer may pull away from these areas because paint shrinks as
it dries. Enamels may wrinkle if a second coat is added without allowing enough
drying time between coats. The thinners in the paint must be allowed time to
evaporate before the next coat is applied.

\secup

\secrel{4.3\ --- CASTINGS}\secdown

There are many ways to cast metal parts, but this is not a process 1 would
recommend trying at home. This section is provided as general information so you
have some idea how metal casting is done. If you are one of those builders who
likes to do every operation yourself, read up on casting techniques before
buying or building any equipment.

\secrel{Casting techniques from the jewelry trade work well for small parts}

The type of castings that could be used for miniature projects at home actually
come from the jewelry trade. They have developed a small centrifugal casting
machine that slings the molten metal into a ceramic mold that has been made by
the "lost wax" method. Commercial casting companies use a sprue system. This is
a method of getting molten metal to the mold and casting more than one part at a
time. It resembles a funnel and many parts may be cast at one time by pouring
molten metal into the sprue system which feeds each part. The top of this
assembly has provision for a small pool of molten metal to continue to feed the
parts as they cool and shrink. They call this assembly a "tree". The sprue can
be attached to the part in a manner to allow variations in length which is
normal to the casting process. The part is then machined to its final length as
the sprue is cut from the casting.

\bigskip
\textit{Die cast lathe bases as they come from the foundry and before they are
machined. Proper draft angles and shrink rates must be considered in the design
so that the parts will release from the mold and end up the right size.}
\bigskip

\secrel{Making molds for investment casting}

The ceramic mold is made by coating the wax pattern with a slurry mix that air
dries. It may take several layers to get the desired thickness. The wax is then
removed from the inside of the mold with steam or heat. This leaves a mold with
no parting line. The ceramic mold is then healed to over $1000^{o}F$ to harden
it and burn away any remnants of the wax. After the mold has been filled with
molten metal and cooled, the mold is destroyed as it is broken away from the
casting. This is called "precision investment casting". The part is produced by
casting or machining a wax model of the final product. If many parts must be
made, a mold can be built to cast the wax pattern. The molds can be simple if
you are willing to accept some second operations machining the wax pattern.

\secrel{Shrink rates of various cast materials must be considered}

The model must have the appropriate amount of shrink rate figured in so the
final casting is to size. If the wax model was machined, the shrink rate of the
wax doesn't have to be figured in, but if it is cast it has to be calculated
from information that is available from the companies that manufacture wax. If
you want a close tolerance cast part it may be worth the effort to run a test
part to determine the shrink rate for a particular shape. The test part does not
need the detail the final product has, just the general size and shape. From
this information a wax pattern can be built that is surprisingly accurate. This
is a very accurate process and parts like turbine blades for jet engines are
made this way. The part finish of small model parts will usually duplicate that
of its full size counterpart. It is possible to cast many types of metals with
this method. Making the wax pattern and having a professional casting company do
the final casting may be the way to go. By taking off on tangents like this you
may end up never completing what you started and end up making jewelry, but
projects done in home workshops are for fun so do whatever you enjoy. As you can
see, it takes a lot of steps to make a good casting and if you're only making
one part, it could be easier to machine it from bar stock and sandblast it to
get the look of a cast piece.

\secrel{Die costing for commercial parts}

Most complex metal shapes found in commercial products are "die cast". The molds
are similar to injection molds for plastic. These molds are built to withstand
internal pressures of 10,000 P.S.I. (1800 $kg/cm^{2}$). These types of molds
are very expensive and should only be considered for production parts made in high
quantities. As an example, the Sherline steady rest mold would cost over
\$10,000 to replace. Molds of this type are attached to the platens of die casting
machines. The hot side is fixed and the ejection side can be moved. Leader pins
align the two halves with the utmost precision as the mold is closed. These
machines are rated by how many tons of force it would take to spread the two
halves apart once they have clamped. A 100-ton machine would be considered
small. They may use a past-center clamping arrangement or a hydraulic cylinder
to keep it closed. After injecting molten material into the mold, the casting is
allowed to cool for several seconds. Cooling time is dependent on the thickest
cross section of the part and the cooling method. When the ejector side of the
mold is pulled away from the hot side, a plate on the ejector side is pushed
forward. Attached to this plate are ejector pins. Die casting molds may have
over fifty pins. These pins push the part from the mold and let it drop out. The
mold then closes to start the sequence over again. I believe mold making is the
best and most interesting job in the machine trades, but it is very demanding.
The parts you work with daily are too valuable to get careless and screw one up.

\secrel{Sand casting\ldots an old method that it still in use today}

The oldest method of casting metals which is still used today is sand casting.
The tooling to produce cast parts can be a wooden (usually Honduras Mahogany)
model of the finished product you want to cast. The expert craftsmen who make
these wooden molds are called patternmakers. The trade requires not only good
woodworking skills, but a vast knowledge of how metal will react when molded.
Shrink rate has to be accounted for and draft angles are needed on sides that
would otherwise drag when the wooden model is lifted from the sand. The model is
set in a "box" and sand is packed around one half of the model. The opposite
side is done the same way and the model usually has some method to increase the
accuracy of this process. At me same time a sprue system is added. Sand used in
sand casting has been treated to stick together. A hollow section in the casting
can be obtained by having a "core". Cores are made of a processed sand that will

\bigskip
\textit{A mahogany pattern is the first step in the sand casting process. A
skilled patternmaker requires not only a knowledge of the properties of metal
and how it will react as it cools, but also requires masterful woodworking
skills. Here Howard Parry applies a wax bead to an inside joint. A heated metal
ball tool is used to shape the wax into the desired radius fillet.}
\bigskip

harden with temperature. The cores are very fragile and are designed to be this
way so they can be broken into small pieces to remove them from the finished
casting. Cores are held in place by the parting line of the mold. The two halves
are put together and molten metal is poured into the mold through the sprue
system. The sprue system will normally have an extra pool of melted material
that helps keep the mold filled while the metal cools. After the metal is cooled
the sand is knocked away from the part and the sprues are removed. It is
difficult to get the degree of accuracy needed in miniature models with sand
castings, but they are an inexpensive method to produce large parts in low
quantities. The base for our 24" (609mm) lathe is produced by the Edelbrock
Company. They have a sand casting facility that is second to none. They use
automated equipment that can fill a mold every 45 seconds. The sand molds are
made in two halves by automatic machines. The molds are also a cast product and
are usually aluminum. I made a model of what I wanted and they used this model
to build a three-cavity mold. The tooling cost was less than \$5000.00 and the
part price was reasonable by doing three at a time. You may be familiar with the
Edelbrock name because they manufacture many fine products like manifolds and
other performance parts for hot rods and race cars.

\secrel{Casting using matched plates and gravity}

There is also a method of casting that uses matched plates. This is simply a
metal mold made in two halves. The major difference between these molds and die
cast molds is the cavity is filled by gravity. not the very high pressures used
in die cast molding.

\secrel{Dealing with foundries to get your parts made at a price you can afford}

Most foundries specialize in only one material. It could take several different
foundries to get the parts you need and each one may have a minimum order. Be
sure to remind the foundry that this is a model project where time isn't as
important as money, and see if it would be possible to run your job with other
commercial parts to keep the price low. You have to be nice to large casting
companies who are willing to take an order to run a few parts for you even
though they probably still wouldn't make a profit on them at twice the price.

\bigskip
\textit{The quality of castings that come in kits varies from good like the ones
shown above to downright unusable.}
\bigskip

\secrel{Beware of poor castings in some kits}

Many steam engine kits contain castings that belong in the junkyard. The
tolerance of the part can be lost in a deformed casting. The manufacturers of
these castings do the industry a disservice by selling castings of poor quality.
A novice may believe they are at fault when a kit can't be finished, but the
truth could be the castings are in error. The sorry part is that sometimes these
novices give up and go on to another less demanding hobby because of it. Kits
should always be designed to make the project easier and more fun to build. If
the model engine kit manufacturers would switch to precision investment castings
for their kits, both manufacturer and customers would benefit.

\secup

\secrel{4.4\ --- OTHER WAYS TO FORM METAL}\secdown

\secrel{Extrusions}

Extrusions are something we are very familiar with at Sherline. Most of the
cross sections for the lathe and mill started out as an extrusion. The basic
method of producing a extruded shape is similar to decorating a cake. The
frosting is forced through a hole which gives the material coming out the shape
of the hole. A cross section with a hole can he extruded by forcing the material
around a core piece that is shaped like a torpedo.

It takes a lot of power to push a 500-pound billet of aluminum through a small
hole, so the first thing you need to extrude metal is a gigantic hydraulic ram.
The aluminum billet is heated but not melted as it is loaded into the cylinder.
The method of loading is similar to loading a cartridge into the barrel of a
rifle. The aluminum shape will usually be twisted and crooked as it leaves the
extrusion die. This shape is then pulled from both ends to straighten it. It is
the same method used to straighten a piece of wire or copper tubing only the
forces used to straighten a shape with a large cross section are tremendous.
This method doesn't allow tight tolerances. At Sherline we have to machine all
mating parts on the extrusions we buy.

Extrusion dies are relatively inexpensive, around \$1,000, and the cost to
produce these dies has come down because of a new machine tool called a "wire
EDM". A wire about .010 inches in diameter cuts the shape as it is fed through
the part by using the "spark erosion" method. The wire never touches the part it
is cutting. The sparks that jump the small gap between the wire and the part
erode the material away. This isn't a fast process, but it does the impossible.
The tool steel die can be heat treated before the shaped hole is cut. The table
is computer controlled and very accurate. A process called "cold drawn" is also
used to more accurately shape extruded parts in metals. This is similar to
extruding but it is accomplished in more than one pass.

\secrel{Forging}

Forging is taking a slug of metal and reshaping it in a die mounted in a press
that acts like two giant hammers hitting the part at the same time. One of the
main advantages of this process is the strength of the part produced. The
''grain" of the metal can be controlled by forging making the part much stronger
and less likely to fail. It isn't something a home hobbyist could do but worth
mentioning.

Quality wrenches are forged. This method can be more exact than you could
imagine. On TV I saw a complete frame for a handgun made in a few seconds by
forging, and it was a very complicated part. Forging tooling can be quite
expensive.

\secrel{Powdered or sintered metal parts}

Powdered metal parts are made in a specialized press that compresses "powdered
metal" from both the top and the bottom at the same time. This produces a part
that has a more equal density than one hit from only one side. The cast part
then goes through a "sintering oven" where the powdered metal is fused together.
Additional strength can be added by filling in the microscopic voids with a
material such as copper. After sintering a piece of copper is placed on the part
and it is run through the furnace again. If done correctly the copper will
disappear into the part making it stronger. Door locks are usually manufactured
in this manner.

\secup

\secrel{4.5\ --- JOINING METAL}\secdown

\secrel{Soldering irons and flux core solder}

The method of putting two pieces of metal together that most hobbyists are
familiar with is soldering. This type of soldering is called "soft soldering".
Soft solder melts at temperatures below $800^{O}F$ and is a combination of lead
and tin. Hobbyist have built many ingenious devices by soldering together piano
wire and brass tubes. These standard items are sold in hobby shops and are used
to build gadgets that are held together with solder. The first thing you find
out about soldering is the surfaces to be joined have to be clean and free from
oxidation. A flux is used to clean and protect the surface from oxidation as the
parts are heated. Many solders have a flux core. These are fine for work that
will be heated with a soldering iron. If you are heating the part with a
soldering iron, the iron should be cleaned by wiping. If a new iron is used or a
thin coat of solder doesn't cover the tip of the iron, the tip of the iron has
to be "tinned". This is accomplished by a dipping the tip of the hot iron
directly into the flux and melting solder on the tip and then wiping the tip
with a rag. Repeat until you get the desired results\ldots a shiny silver tip on
the iron. It is then ready to use.

\secrel{Soldering torches and solid tore solder}

For larger parts that have to be heated with a torch a paste flux must be used.
This keeps the area around the joint clean as the part is being heated. Propane
torches are somewhat dirty and flux is a must. Acetylene torches work better but
they are too costly for the average hobbyist. In general, solder will flow
towards the heat source (even if it is uphill). It is usually best to heat the
side with the most mass and apply the solder to the opposite side. Always use a
little flux and solder to transfer heat from the iron to the part. When soft
soldering a part that is dependent on the strength of the joint use a solder
that contains about 65\% or more lead. A 60\% tin content is used for soldering
electronic circuit boards and cabling, Flux should always be removed from the
joint after the solder has hardened. It will cause a corrosion problem because
it contains acid. You can also wipe the excess solder and flux away with a rag
before the joint has hardened. Plumbers use this trick to get an excellent
appearance when they solder copper tubing. In general, soldering is a skill that
has to be learned, and the best way to learn is by doing. Try soldering on a
similar piece of scrap material before trying to solder a part that may contain
a lot of work.

\bigskip
\textit{Here's a close-up detail of part of the blacksmith's triphammer shown on
page 47. The silver soldered parts are perfectly joined, filleted and finished
to look like castings. Jerry Kieffer's skills as a modelmaker are highlighted by
the number of different metalworking techniques he has mastered.}
\bigskip

\secrel{Brazing and silver soldering provides more strength}

When strength is critical brazing should be considered. Brazing materials melt
above $800^{o}F$ and below the melting temperature of the materials being
brazed. Some brazing materials may penetrate the surface of the materials being
joined giving very strong joints. They will always "wet" the surface for a
successful braze. Silver solder is probably the best known brazing material. It
is expensive but it is also used in such small amounts that cost shouldn't be an
issue. Flux is a must and surprisingly it is usually water based. Liberally
apply the flux to the area around the joint and heat the work in such a way that
you don't boil off the flux. I usually use a soft flame to start and get the
water out of the flux. This will leave a white coating. The part is then brought
up to silver soldering temperature. More flux may have to be added to insure
clean soldering surfaces, bleat the silver solder rod and dip it into the flux.
This coating will protect it from oxidizing as it is applied. Just like soft
soldering, the melted brazing material will go towards the heat. Try to bring
the work up to temperature evenly and apply the brazing material to the opposite
side of the heat source. I have found that when the flux melts and wets the
surface of the joint the temperature is ready for the silver solder to be added.
If it doesn't flow the work temperature must be raised. Keep adding flux if
necessary. Holding any materials at high temperatures for a long period of time
with a torch can cause problems. Propane torches are usually not hot enough to
silver solder large parts successfully. They also have a dirty flame that
creates problems at the higher temperatures used in silver soldering. If you
attempt to silver solder with propane you must use an excess of flux.

\bigskip
\textit{Some sample aluminum welds. Controlling the puddle of molten metal is
the key to a good weld.}
\bigskip
\textit{A close-up of an expert weld in aluminum shows a good, even head with
proper penetration. Like all skills, welding takes a lot of practice and a good
knowledge of your tools and materials to get good results. Being able to do some
basic welding is probably one of the more useful skills you can obtain. Though
not often used in miniature machining, it can come in handy for many other
construction and repair projects you will run across.}
\bigskip

\secrel{Arc welding}

The small parts normally made with Sherline tools are too small for arc welding,
which is the most common way to weld. Unlike brazing, welding is a process which
melts the material being joined and the joint can be as strong as the original
material.

\bigskip
\textit{Some welding shops specialize in exotic metals. Welding titanium such as
this Mako mountain bike frame requires special tools and a lot of skill. The
advantage gained here is a frame that weighs 2.5 pounds instead of the 5 pounds
it would weigh if made from steel.}
\bigskip

There are several ways to weld, and the difference is mainly where the heat
comes from. Low voltage and high amperage is the usual source. Arc welding
starts by dragging a flux coated rod across the weld joint until an arc is
established. A very steady hand is required to control the end of the rod that
may be 10" away from handle and clamp that is holding the rod. The puddle of
molten metal is controlled by minute movements. Good welders can arc weld a
perfect joint standing on their heads if need be. A poor quality weld in a hard
to get place can fail as easily as a weld on the front side. The weld
temperature is controlled by the machine settings and rod diameter. The "ground"
has to be perfect to weld properly. Welds must be perfect for they must hold
buildings, bridges and a large part of the industrial world together. Our lives
are much more dependent on the skill of this group of tradesmen than one would
think.

\secrel{Spot welding}

Resistance or spot welding is very common in our lives. It holds together most
of the automobiles that we drive daily. It is a quick process but only certain
materials can be spot weld successfully. The metal is clamped together by two
copper tips. A short blast of electrical energy is fed across the joint and the
metal fuses together. The welds done at automobile factories are now done by CNC
robots and have thankfully eliminated this job from assembly lines. I have used
spot welders that are very small and called "tweezers welders. They can be used
to attach thin shim stock to thicker parts.

\secrel{GMAW welding (or MIG welding)}

Another type of welding that is becoming very popular is technically known as
GMAW welding which stands for "Gas Metal Arc Welding", but it is more popularly
called MIG welding. An inert gas such as argon floods the area to keep oxygen
away from the melted surface. The welding material is a spool of wire that is
mechanically fed down the center of the welding cable. It comes out through the
torch through the center of a ceramic cup where the inert gas is also streaming
out. The rate and amperage has to be adjusted so the material will melt at the
same rate it is fed. The first time I tried it I moved the torch away from the
work and in less than a second I had a 6" long wire that was incredibly hot. I
realized this was going to take some getting used to. MIG welders have come down
in price over the years and can be useful to build stands and trailers, but it
would be hard to find a use for them on parts normally associated with miniature
machining.

\bigskip
\textit{Though an oxy/acetylene torch may be the first thing many of us think of
when we hear the word "welding", most industrial welding is done using
electrical current as a heat source. Shown here is a TIG (Tungsten Inert Gas)
welding setup that brings the price down in the "home shop" range. This foot
pedal controlled unit sells for about \$1300 (1998) with the wheeled cart adding
another \$170.}
\bigskip

\secrel{TIG welding}

TIG (Tungsten Inert Gas) welding is a method that may be of some use when
working with small parts, The tungsten rod is located in the center of a ceramic
cup that directs an inert gas such as argon over a clean surface that is going
to be welded. A high frequency contact is made between the work and the tungsten
tip. An arc is established and the high frequency is turned off automatically.
The size of this arc can be very accurately controlled by machine settings and a
foot pedal similar to an accelerator. You have to hold the tungsten very close
to the work to TIG weld. The diameter of the tungsten has to be proper for the
welding machine settings to keep it from melting. If a weld is critical, any
tungsten that falls or is broken off in the weld must be ground out. If the
tungsten is dipped into the melted puddle, the melted material will attach
itself to the tungsten and form a small glob. When this happens the torch is
hard to control and an arc may suddenly shoot out on an angle and destroy a
delicate part. The point of the tungsten must be shaped to a point to accurately
aim the "flame". I prefer a fine tip with an elliptical shaped taper going to
the outside diameter of the tungsten. One thing that you can do with TIG welding
is fuse two pieces of metal together. Stainless steel can easily be fused
together and very delicate welds can be done by fusing. When welding very small
parts a "chill block" may have to be used to protect the main body of the part
from excessive heat. These are usually made of copper and clamped just below the
weld.

\secrel{TIG welding aluminum}

When aluminum is TIG welded it must be very clean. The entire part to be welded
has to be cleaned with an acid bath before welding to insure a good weld.
Because aluminum can dissipate heat so quickly, large parts have to be
preheated. I use an oxygen/acetylene torch. By turning down the oxygen the flame
will create "soot". This soot will stick to the part and will be released from
the aluminum surface when the aluminum is the right temperature to weld.
Aluminum never looks hot, and a hot piece that has just been welded should never
be left alone to cool where someone may inadvertently touch it. At least put a
sign on it. Don't cool it with water as you may anneal or soften the material.

\secrel{Star Wars welding with lasers and electronic beams}

Exotic fusion welds are now being done with lasers and electronic beam welding.
These can be microscopic welds that are controlled with the utmost precision.
This type of welding is used by laboratories and production equipment, and
because of the high cost it normally wouldn't be considered for the home
hobbyist.

\secrel{Taking your skills in these areas to the next level}

In closing this section I want to remind you that I'm not an expert in these
fields. I have dabbled with each process enough to know that I would need a
lot more practice to be really good at any of them. On the other hand, trying
your hand at something really helps you appreciate a good job when you see it
done by an expert in the field, and there is a certain satisfaction in simply
knowing a good piece of work when you see it. Books have been written about each
subject mentioned and you really need to understand the process more than I
described it before attempting welding or brazing. If you have a friend who
understands these processes, buy them dinner and pick their brain, but remember
there aren't many welders who work with parts as small as we work with daily. It
will very seldom be useful to fix a mistake by welding a small part and, like it
or not, the best way will usually be starting over.

\bigskip
\textit{How would you remove this nut which appears to be thoroughly welded to
the bolt? This fun puzzle by Larry Lamp was an entry in the 1995 Machinist's
Challenge contest in Michigan.}
\bigskip
\textit{The solution is obvious once you know how it works, but many spectators
took one look at the massive weld and said it was impossible.}
\bigskip

\secup

\secup

\secrel{Modeling tricks of the trade from Phil Mattson}

\secdown

\secrel{Making a brass ship's propeller}

Most kits come with a cast propeller which you file to final finish. If you are
building a model from scratch, making a nice looking scale propeller can be a
tough project unless you have a plan. Phil Mattson has made many of them, and
here's how he does it.

\begin{enumerate}
  \item 
A hub is turned on the lathe and left on the end of its piece of stock. The
chuck is put onto the indexing attachment (or a rotary table) and held on the
mill. The headstock is offset to the angle of the desired pitch for your blades
and a slot is cut for each blade.
  \item 
Four blade profiles are cut from sheet brass and filed to shape. The photo shows
the hub and the four flat blades along with a finished prop.
  \item 
Each blade is placed on a block of hard rubber and a brass billet of the proper
diameter is laid on top of it. The billet is hit with a hammer (protect it with
a piece of wood) and the proper curve is bent into the blade. The photo shows a
fiat blade on the rubber sheet and a finished blade on top of the billet.
  \item 
The hub is placed on the shaft and blades are soldered into each of the slots.
As a final touch, the prop is polished to a mirror finish.
\end{enumerate}

\secup
