\secrel{Chapter 10\ --- Lubrication and maintenance}\secdown

\secrel{Cleanliness is one of the keys to accuracy}

In order to perform with consistent accuracy, all machine tools, regardless of
size, must be kept clean, lubricated and adjusted. Complaints about the lack of
accuracy of a machine can often be traced to a stray metal chip under a part, a
dinged up thread or dent in a mating surface. Before threading a chuck or end
mill holder onto a spindle, both the male and female threads as well as the
seating surfaces must be free from dings and chips. Accuracy of collets or the
seating of a tapered center or arbor can also be affected by chips. Keep your
tools and work surface clean by brushing off chips often and sweeping or
vacuuming them up.

\secrel{Lubrication}

Keeping all the moving parts of your machines and accessories lubricated will
not only make them easier to operate, it will also extend their life. Here are
some of the main surfaces that need lubricating and some suggested lubricants to
keep them in top condition:

\paragraph{MACHINE SLIDES} Use a light oil such as sewing machine oil on all
points where there is sliding contact. This should be done immediately after
each cleanup. At the factory, the slides are greased to ensure the lubrication
stays in place during shipping, but light oil will work fine once you begin
using the machine.

\paragraph{LEADSCREWS} Sewing machine oil should be placed along all the threads
regularly. At the same time, cheek that the threads are free from any metal
chips. Use a brush to keep them clean.

\paragraph{TAILSTOCK SPINDLE} Wind out the spindle as far as it will go, wipe it
off and lightly oil it with sewing machine oil.

\paragraph{HANDWHEELS} A few drops of light oil behind the handwheel will reduce
friction between surfaces and make operation easier and smoother.

\paragraph{HEADSTOCK BEARINGS} The bearings on Sherline machines are lubricated
at the factory for the lifetime of the machine and should not need further
lubrication. DO NOT break the seals.

\paragraph{MOTOR} Sherline machines use sealed ball bearings which require no
maintenance,

\paragraph{CHUCKS} On self centering chucks you should clean and lubricate the
jaws each time they are removed and reversed. Make sure they are clean and free
of chips before they are reinserted back into the scroll. Independent jaw chucks
like the 4-jaw should also be checked often to make sure there are no chips on
the threads or working surfaces and that the threads are lubricated with light
oil. Check also that the spindle threads and seating surface on the back of the
chuck is clean before installing it on the spindle. Chucks may be wiped down
with light oil or sprayed with a spray lubricant/preservative when put away to
keep them from rusting.

\bigskip
MOTOR

HEADSTOCK BEARINGS

CHUCK

LEADSCREWS

MACHINE SLIDES

TAILSTOCK SPINDLE

HANDWHEELS

LUBRICATION POINTS

NO LUBRICATION NECESSARY
\bigskip

\secrel{General precautions when using miniature machine tools}

\begin{itemize}
  \item WEAR EYE PROTECTION AT ALL TIMES!
  \item If you haven't already done so, read the safety rules for power tools at
  the beginning of this book.
  \item \textbf{Do not attempt to use the lathe or mill without first securing
  them to a mounting board} or directly to your bench. The mounting board keeps
  them from tipping and allows the machines to be easily moved and stored.
  Rubber feet on the bottom of the board reduce noise and vibration that occur
  when the machine is mounted directly to a bench.
  \item \textbf{DON'T OVERTIGHTEN!} One of the biggest problems in designing and
  manufacturing small metal cutting tools is the fact that the operator can
  physically be stronger than the machine. This is not normally the case with
  big machines. For example, a 10-pound force applied a couple of inches out on
  a hex key becomes a 650-pound force at the tip of the screw. If you tighten
  both screws on the tool post this tight, it becomes approximately 1300 pounds
  of force on relatively small parts. Tools or parts can become distorted and
  accuracy is lost. Overtightening hold-down screws and T-nuts in their slots
  can distort the crosslide or mill table. It is not necessary to overtighten
  parts and tools because loads are smaller on equipment of this size. Save your
  equipment and extend its life by not overtightening and by taking lighter
  cuts.
  \item \textbf{Don't overstress the motor}. It is important to remember that
  you can overload the motor on a small lathe or mill. The many variables
  involved in machining, such as hardness of material, size of cutter, shape of
  cutter and diameter of stock all affect the load on the motor. Follow the
  suggested speed charts, and if the motor sounds overloaded, take smaller cuts.
  Most importantly of all, just use COMMON SENSE!
  
  The motors on Sherline machines are thermally protected. That means they have
  a built-in circuit breaker that will shut down the motor if it gets too hot.
  This keeps the motor from burning out. The breaker will automatically reset
  as soon as the motor cools and you can go back to cutting. If the motor does
  shut down, immediately shut off the power switch and back your tool out of
  the work. The circuit breaker should reset in about 10 seconds.
  
  Then turn the machine back on and resume cutting using a lighter load. If your
  motor is overheating on a regular basis, it means you are taking too heavy a
  cut or operating at too high an RPM for long periods. Slow your speed down,
  reduce your cut or feed rate and you should have no further problems.
  
  Due to the nature of miniature machining, overloading the machine is a common
  problem. It is often tempting to try to speed up the process by working
  faster. Keep in mind this is a small machine and work with patience and
  precision. Don't be in a hurry. Your parts will come out better and your
  machine will last much longer.

  \item \textbf{Regularly Check the tightness of bolts and screws}. Vibration
  from operation can cause fasteners to loosen up. Also check all fittings on a
  new machine before using it to make sure nothing has loosened up during
  shipping.
\end{itemize}

\bigskip
\textit{Here is a beautiful oak and plexiglass storage unit and work surface
that was custom made many years ago. It includes drawers for storage of
accessories.}
\bigskip

\secrel{Storing your machines and accessories}

Your machine should be cleaned and lubricated before it is put away. It should
be kept covered to keep dust and grit off the working parts. Sherline sells
fitted vinyl covers for each machine which look and work great, but any type of
cover will extend the life of your machine. Accessories can be kept in their
original boxes or wrapped in cloth. They should not be allowed to bang around
against each other in a drawer. This can cause them to collect dirt and dents
which will adversely affect their accuracy.

You can bet Rembrandt took good care of his paint brushes. Miniature machine
tools are also capable of producing beautiful results when maintained properly,
If you expect to produce a masterpiece or just good, accurate parts, take care
of your tools and they won't let you down.

\secrel{Transporting and carrying your machines}

Never pick up a machine by its motor. The motor on the lathe seems a
particularly handy place to grab it to pick it up, but it is mounted to the
headstock with a diecast part that ends up supporting all the weight of the
machine...a task it was not designed for. To carry or move a machine, it is best
to have it mounted to a base. Pick it up by the base rather than by the machine
itself.

Several people have gone so far as to mount a brass carrying handle to each end
of the machine's base board. This is an excellent idea, particularly if you
cannot leave it permanently set up and must store it each time you are done with
it. The handles will give you a better grip and assure the machine doesn't get
dropped.

\bigskip
\textit{Parts are loaded on alt four sides of one pallet as a another batch of
parts is being cut inside this CNC machining center. Though the stakes are
higher and the sizes are bigger on a machine like this, good maintenance is
required for machines of any size to produce good parts.}
\bigskip

"When I am working on a problem, I never think about beauty. I think only bow to
solve the problem. But when I have finished, if the solution is not beautiful, I
know it is wrong."

--- Buckministcr Fuller

\bigskip
\textit{This "grasshopper"\ crosshead steam engine was built by Jerry Kieffer.
It is a 1/5 scale model of another model of the engine by Roy Ozouf a well known
modeler. One of Jerry's favorite challenges is taking something that is already
small and making it even smaller. A ladies wristwatch provides size scale.}
\bigskip

\textit{To carry the engine, Jerry made a tiny oak finger jointed box with
sliding lid. Jerry displays it at shows by setting it atop a pencil eraser to
provide scale.}
\bigskip

\textit{Even Jerry Kieffer admits that he is "pushing his limits"\ with this tiny
oscillating steam engine. It has two flywheels. 085 "\ in diameter. The bore is
.029"\ and the stroke is .032"\ The intake and exhaust ports are .008"\ and the
whole engine weighs 3.5 grains (1/131 oz.)}
\bigskip

\secup
