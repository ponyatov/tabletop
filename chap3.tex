\secrel{Chapter 3\ --- Materials for metalworking}\secdown

\secrel{Some good materials to work with}

As with many of the other sections of this book, I am purposely staying away
from engineering data and only giving the practical information you need to get
started. The book I would recommend for technical information would be
Machinery's Handbook. This book is the best book ever written on metals and the
machining of metal (See chapter 1 of this section.) As a beginner, I would also
avoid complex parts that require heat treating. If you try to accomplish this
with a torch, you will find it an excellent way to destroy perfectly good parts.
This can be heartbreaking to novices because they haven't "started over" enough
in their life. I recommend using "leadloy" 12L14 for soft steel parts. "Stress
proof steel is the next real improvement in steels. It is a better material that
is a little harder to machine. Use 4130 or 4140 steel for extra strong or tough
parts, 6061-T6 for aluminum parts, and 303 for round stainless steel parts.
These materials are readily available and work quite well. Ordering stock sizes
that aren't readily available can be a waste of time. If you only have one part
to make, it may take just a few minutes to cut a piece of stock to its proper
starting diameter, but it could take an hour on the phone to find the perfect
size.

\bigskip
\textit{A good cutoff saw is an important time-saver in any shop. Miniature
machine tools weren't meant to remove a lot of metal in a hurry, so getting a
rough part as close to size as possible with a cutoff saw saves a lot of time
and wear and tear on your machines.}

\bigskip

\textit{A treasure (rove of raw materials can be found at most scrap yards.
Knowing the properties of the material you are looking far is helpful, because
often it is not marked.}
\bigskip

\secrel{Eliminate machining time by cutting material close to size first}

A good investment for your miniature machine shop is a cutoff saw. There is a
saw available at many discount tool stores that is sold for less than \$200
(1998) that is just right for home shop use. It uses a band saw design and can
cut off diameters to four inches (100\,mm). It can also be used in a vertical
position making a standard band saw out of it. You need to have a way to cut
metal to length in a pleasant fashion, and, believe me, trying to saw off a
large piece of stock with a hacksaw isn't pleasant. It will drive you out of the
hobby faster than anything I know. When you are working with small machines it
is very important to eliminate as much machining as possible by cutting stock to
its proper length before starting. A little cutting oil now and then will help
the process and keep blades sharper longer. If you have a friend with a good saw
it is almost as good, but eventually you'll want your own.

\secrel{Cut off tool cautions}

A cutoff tool or "parting-off"\ tool shouldn't be used in place of a cutoff saw.
Cut your material to size with the cutoff saw, and only use the cutoff tool on
the lathe for separating the finished piece from the blank. By the way, don't
attempt to use a cutoff tool holder and tool on a lathe at any place other than
close to the three-jaw chuck. It will bind in the material and get ripped out of
its holding device and may end up damaging your machine.

\bigskip
\textit{Brass is easy to work with and polishes up nicely so it is a popular
material for model work. The chips are like little splinters, so work with it
carefully.}
\bigskip

\secrel{Dealing with minimum orders and small quantities The choices for metal}

seem endless, but there is a problem. It's called the "minimum order". Let's say
you want to start on a project that contains several types of steet, brass,
aluminum, and cast iron. Chances are each of these products will be sold by a
different supplier; each with their own rules when it comes to extra charges for
small orders. You should also understand that most bar material used in machine
shops comes in twelve-foot lengths and material used in construction and
fabrication shops comes in twenty-foot lengths. It may cost as much as \$25 to
have a single bar cut to length. If you call up one of these suppliers and try
to order a four-inch piece (100\,mm) of half-inch (12\,mm) aluminum they will
probably ask if you are joking. You will exist to service manufacturing
companies that buy in very large quantities, not for the home hobbyist.

\secrel{Sources other than the big industrial suppliers} 

You will have to order from a supplier that caters to the hobby market. We have
several listed on our web site and we wish them well because they provide a
service that is well worth the extra cost. You can order all your materials from
one source at one time. This allows you to spend your time building things, not
talking on the phone to somebody who considers your order not worth the effort.
If you are a novice, aways buy enough material to make three parts in case you
have to start over.

\secrel{Salvage yards}

If you are lucky, there could be a salvage yard in your area that sells bar
stock. Bring your own hacksaw because they will not cut it for you without price
is right. Surplus yards don't have a standard inventory, and just because they
may have a good assortment of material today, it doesn't mean it will be there
tomorrow. What is available quite often are bar ends. These are pieces left over
after a bar of stock has been cut up which are too short to use for their
particular part. The problem is knowing what the material is.

\bigskip
\textit{Wood can also be machined on miniature machine tools with the use of a
tool post to support (he hand held cutting tools. Small items like pens made
from exotic woods are popular items as gifts. Kits are available for the working
parts of the pen. Because these projects require such small pieces of material
you can afford to use beautiful and exotic woods. Low material cost is one of
the big advantages of working on small projects.}
\bigskip

\secrel{Identifying various aluminum materials in a scrap yard}

Material is usually color coded with paint on the end of the bar. I wish I could
put a chart of colors vs. material in this book, but for some idiotic reason,
each producer has its own colors. Aluminum will have the grades printed on the
entire length of the bar. The grade I recommend is 6061-T6. The "T" indicates
the hardness. Another grade you may find in a salvage yard is 2011-T3. This has
a texture similar to cast iron and it was developed for making round parts on
automatic machines. The chips are splinters and will not tangle a machine up
with long, stringy chips. Softer grades of aluminum don't machine very well and
lack the stiffness required in mechanical parts. The extruded aluminum shapes
sold in hardware stores are usually a soft, gummy aluminum such as 6063-T3.
Another type of aluminum available is the 2000 series which is usually found in
extruded shapes such as rectangles and squares. The last one worth mentioning is
the 7000 series. This grade is commonly used for aircraft parts and is available
in a hardness to T8. This grade can have a surface hardness equal to some steels
so it can be very useful for tooling.

\secrel{Buying scrap steel is a little tougher}

A whole new set of problems arises when you try to buy steel at surplus yards,
for it isn't labeled. If it is rusty, it probably is "hot rolled", which is a
terrible material to work with for small parts. These are fabrication materials
that, as an example, are good for making a wrought iron gate. The best steel
material to machine is called "leadloy" or 12L14, It is available in round and
square cross sections and can be case hardened. Using 12L14 will ease much of
the pain of machining steel as it is a pleasure to cut. Many of the parts of
Sherline tools are made of this material. Excellent finishes are easy to attain.
The chips break as they are machined off, eliminating the danger of long, sharp
chips.

\bigskip
\textit{Aluminum grades are normally printed on the bar stock. Color coded ends
also identify each type of aluminum, but color coding varies depending on who
produced it.}
\bigskip

A material called "stress proof would be the next real improvement over leadloy.
It isn't too expensive and machines better than cold rolled. Cold rolled steel
is miserable stuff to machine by comparison. It is tough and gummy and has a low
cutting speed, but it is slightly better for case hardening than leadloy.
Normally you would have to grind it to get a good finish. For some reason that I
don't understand, you can cut cold rolled and similar hard-to-machine steels at
cutting speeds way above the recommended speeds (as much as four times) and get
a mirror finish. The catch is you have to use carbide insert tools. These tools
can cut machine times in half and are a basic cutting tool in a modern machine
shop. (See Chapter 6 on Cutting Tools in this section.) The chips are very hot
and care should be taken to protect yourself.

Tools steels can be hardened completely, not just the outside as in case
hardening, and can be very expensive and hard to machine. They are used when
high strength or holding a sharp edge is important. A stamping die would be a
good example. I would recommend that you purchase these materials only when the
material is clearly identified. It takes too long to make parts out of this
material to have them ruined in heat treating. Because of the many uses of
steels and the many kinds available, I'm limiting the information I provide on
these materials. I would end up with a book filled with charts and tables that
would be just like the books that line the shelves of your library's engineering
section. Before starting on a project that requires heat treating, get the
advice of a local heat treat shop, The positives and negatives of brass, copper
and bronze Brass is usually sold in a half-hard condition and is very easy to
machine. I don't like to machine it because the chips are like small splinters
that stick in your hands and break off. However, the parts always look nice and
can be easily polished and plated. Copper can be machined but the surface has a
tendency to "tear" as it is being cut making for poor finishes. Diamond tools
are used to cut copper in a production environment if good surface finishes are
a must. Some grades of bronze can also be difficult to machine because they will
wear tools, even carbide, at an alarming rate.

\secrel{Cast iron}

Cast iron can be purchased as bar stock or cut out of an old piece of junk that
contains cast iron. It is easy to machine, but also dirty. The chips look like
powdered coal, and you should clean up your equipment after cutting it. The good
part about cast iron is it is very stable and will not warp as it is machined.
It is surprising how much some materials can warp when cut. Manufacturers of
complex, close tolerance parts will have materials "normalized"

\bigskip
\textit{With the addition of the wood tool rest, small wooden parts are easy to
turn on a Sherline lathe. Here, exotic woods are used to make a handsome key fob
and a small flute. Model ships and dollhouse miniatures also require small
turned wooden parts.}
\bigskip

several times between machining operations to counteract these forces. The
bigger the part the bigger the problem unless you are using cast iron.

\secrel{Woods and plastics}

Wood and plastic can be easily machined with tools designed to cut metals. For
best results when machining wood, use a very hard, fine grained wood such as
maple. Soft woods will crush rather than cut. This causes poor finishes and
splintering. Use two-fluted end mills when machining these materials.

The problem with plastic is "melting". You can't allow the chip to clog up the
cutting action when machining any type of plastic. For example, if a drill feed
is low and the RPM is high, the chips will be numerous and thin. They pile up in
the drill's flutes and not only create heat from friction but also work as an
insulator. The plastic melts and you have ruined your part. Plastic has a very
high temperature expansion rate when compared with other materials and it might
be wise to take the close tolerance cuts when the part has cooled down to room
temperature. Use high feed rates and sharp tools to eliminate or minimize these
problems.

\secrel{Don't knock plastic\ldots sometimes it's the best material for the job}

I was at a trade show once when a youngster in his teens stopped by our booth.
He brought with him a complex part used in the front end of a radio controlled
model car. It was made out of plastic and

\bigskip
\textit{A plastic mold used to make a housing for the new digital readout for a
Sherline mill. It can take a very expensive mold to produce what looks like a
simple part. The longer the machine runs, however, the cheaper the parts
become.}

\bigskip

\textit{A selection of plastic raw materials. The dark brown block in the upper
left is phenolic. Next to it is black Delrin. The red block is fiberglass. At
the far left is a piece of white Delrin and next to it is a block of white
Nylon. The two piles are new clear beads and recycled black chips used for
injection molding.}
\bigskip

the part was constantly failing in crashes. He wanted to buy a machine to build
this particular part out of metal and save money. He thought he was getting
ripped off because the plastic parts cost \$3.00 each. I told him I could not
build a part to replace the one he showed me in less than six hours and a metal
part probably would not last any longer, because metal will not spring back to
its original shape like the plastic part. I don't think he ever understood why
il would take me six hours to make a three -dollar part.

Plastics have changed the way we think about value. To injection mold a complex
part, it takes a complex mold, yet the machine cycle time and the skill to run
this plastic injection machine remain constant. The machine time is based on the
thickest cross-section, not the complexity of the part. In order for the mold to
work it must be made with very tight tolerances, even if the part itself has
liberal tolerances. With a good mold, plastic parts can be manufactured so
inexpensively we consider them disposable, but don't think for a moment that all
plastic parts are junk.

Take for example your new cellular phone. The molded forms of the body would be
very difficult to produce in any material other than plastic. They arc sturdy,
light in weight and have the desired color molded in so they don't have to be
painted. If they were made out of metal they would be heavier, more expensive
and no more attractive. Plastics can be formulated to achieve just about any
desired characteristic from flexibility to heat resistance to color or clarity.
The development of plastics in our lifetime is probably the most important
advance in materials since the discovery of how to work with metals many
centuries ago.

\secrel{Buying plastic for molding and machining}

Plastic can be purchased in granular form for injection molding or in bar stock
for machining. For machining I prefer Nylon where strength is important and
Delrin for general work. In bar form, plastic can be very expensive. A two-inch
diameter Delrin rod may cost over one dollar per inch compared with two dollars
a pound in granules for injection molding (1998). Injection molded products are
the preferred choice for use in manufacturing. As always, the tooling costs are
high in order to produce a low cost part. If you are thinking about a new
product to be produced in high volume you have to consider using plastic.

\secrel{Machining plastics}

Machining plastic can be fun. You don't need coolant and the chips are easy to
control and clean up. Use an RPM that allows heavy feed rates and use sharp
tools. This will keep the plastic part from melting as it is being machined.
When drilling holes use a very fast feed rate to keep the drill flutes from
clogging. We ran a very profitable screw machine part by drilling a hole in
Delrin at 10 times the suggested rate. The plastic would extrude out the flutes
without generating any heat yet the drilled hole had a better finish and
tolerance than our previous slower method. Remember that the surface of plastic
is slippery and softer than metal. You have to consider this when work is held
in a vise or chuck. Thin sections can be easily deformed when clamped.
Temperature is also a factor. Plastic has a high coefficient of expansion, and a
two-inch part can vary several thousandths of an inch from a hot day to a cool
day.

\secrel{Galling of materials in close fits}

In summary, I find aluminum the nicest material to work with. It is clean,
strong and rust free. You should also be aware that aluminum can "gall", which
is the surface of one part sticking to another of the same material. This
usually happens when you check a fit by putting a shaft in a hole made of the
same material. Stainless steel can be just as bad as aluminum when it comes to
galling. If you have to do a lot of work in these two materials plan to use an
anti-galling agent on close fits. It is available in automotive shops.

\bigskip
\textit{This 7"\ high display model of a 1911 "Baby"\  model airplane engine by
Edwin Teachworth was built entirely of styrene and painted to look like metal.
It is on display in the San Diego Aerospace Museum.}
\bigskip

\secrel{Hard-to-find fittings}

Novices believe that some catalog somewhere will always have what they need, and
all they have to do is order it. This is not always the case. Many catalogs are
filled with sizes that won't actually be produced unless they get a substantial
order. When you try to use "off-the-shelf"\ items, you often must compromise
your design. Of course, I'm not suggesting that you make things like nuts or
bolts, but you might have to make that special washer, There are plans around
for interesting projects that were drawn with the materials that were available
then. (Try and buy a BSA screw at your local hardware store and you will
understand the problem.) Unless you are building "super scale", you can save
yourself a lot of grief and use the materials and fasteners available today.
Fortunately, you have chosen a hobby that gives you an alternative if the exact
part you are looking for can't be found. You can always make it yourself.

\secup
