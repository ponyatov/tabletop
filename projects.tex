\secrel{PROJECTS\ldots A gallery of miniature craftsmanship}

This section is devoted to showing you some of the great projects made on
tabletop machine tools tike those discussed in this book. After all, it isn't
really the tools you are interested in so much as what can be made with them. A
column of figures about the size and accuracy of a machine will tell you how big
it is and how well it is built, but it still won't tell you what can be built
with it. These photos are some of the most important in the book because they
show what these tools in the hands of craftsmen have actually done. And yet, as
impressive as some of these projects are, they still only represent the best of
what has been done to date, not the best that will ever be done. That is up to
you.

Hundreds of years ago, craftsmen made timepieces and mechanical calendars that
required tremendous precision. Modelmakers made tiny ships in bottles and
detailed display models of ships. In fact, before naval architects began drawing
plans of ships and shipwrights knew how to read them, ship designers built
models and the builders used that as a guide. Despite the quality, accuracy and
detail of these old projects, the tools they had to work with were crude by
today's standards. As tools and materials have improved, it has become easier
and more fun to make very precise parts. Almost all of the projects shown here
were made by hobbyists, not professional machinists. If you have patience, some
skill with your hands and a desire to make projects like these, today's tools
will bring you a lot of satisfaction and enjoyment. There is not a project here
that couldn't have been built on a tabletop in your own kitchen, den or home
shop.

\bigskip
\textit{An American quarter and dime are used for size reference in many photos
in this book. For those outside the United States who might not be familiar with
these coins, they are shown here at actual size. A quarter (\$.25) is .950"\ or
24.1\,mm in diameter, while a dime (\$.10) is .705"\ or 17.9\,mm in diameter.}
\bigskip

\bigskip
\textbf{Steam tractor, Dennis Franz, Newton, Kansas}

A lot of detail is packed into a very tiny package. This model won 2nd prize at
the 1995 Sherline Machinist's Challenge contest in Michigan.

\bigskip
\textbf{Stover "hit'n miss"\ gas engine George Luhrs, Shoreham, New York}

Paint and pinstripes add a nice finish to this model which finished 4th in 1995.
It has a 7/16"\ bore and 5/8"\ stroke. The speed control is quite detailed and
complicated.

\bigskip\textbf{1/12 Ferrari V-12 F1 Engine
Bob Breslauer
Ft. Lauderdale, Florida}

Approximately 1500 hand made pieces went into this display model engine and
transmission. More photos of it can be seen in the profile on Bob on page 311 at
the beginning of Section 5.

\bigskip
\textbf{Single action steam engine
Chris Thompson
Colorado Springs, Colorado}

At the extreme small end of the size scale is this tiny steam engine with a
1/8"\ (3.2\,mm) bore and stroke.

\bigskip
\textbf{Gattling gun, George Britnell, Strongsville, Ohio}

This walnut and brass gun took 3rd place in the 1995 Sherline Machinist's
Challenge contest. The barrels rotate and the elevation mechanism also works.
The quality of finish on every part is superb.

\bigskip
\textbf{Miniature micrometer, Dennis Scherf, Cedarburg, Wisconsin}

Miniature tools are a popular subject for modelers. This tool and felt-lined box
can easily be carried in a pocket and is a great "conversation starter."

\bigskip
\textbf{Air Compressor, steam engine and miniature tools Kurt Schulz, Harper}

Woods, Michigan Not just a steam engine, but the air compressor to drive it too,
this handsome model is an interesting combination of round and hard edged parts,
satin and shiny surfaces. At the bottom are some other of Kurt's projects: a
miniature height gage and two small mill vises sitting on a ground surface
plate. The small vise would make an interesting tie tack!

\bigskip
\textbf{Hot air engine, Scotty Hewitt, Van Nuys, California}

This delicate engine is powered by the difference in temperature above and below
it. Set it on a hot cup of coffee, give it a turn and it will spin like crazy
for over 15 minutes. Scotty produced a short run of these to sell in toy stores.

\bigskip
\textbf{Lunkenheimer oiler, Jerry Kieffer, Deforest, Wisconsin}

Just like the full size prototype, this tiny oiler delivers measured amounts of
oil to a bearing or cylinder. The "sight hole"\ through the base allows the
engineer to check the drip rate visually.

\bigskip
\textbf{.010 Diesel model aircraft engine}

A simple design and nicely made aluminum parts make for an interesting little
engine. Not much in material cost here!

\bigskip
\textbf{1/30 Corliss steam engine Jerry Kieffer, DeForest, Wisconsin}

(Below) This model represents Jerry Kieffer's determination to build to scale
down to the smallest detail. Even 1/4-20 bolts are scaled to 1/30 size. Though
modelers will often use hidden springs to return the valve gear, the "pots" at
the bottom actually pull a vacuum just like the real ones. A portion of a
quarter can be seen at the bottom for scale. (More on Jerry and this engine can
be found on page 112.)

\bigskip
Above is a photo of the real 1909 Viltcr Corliss engine Jerry used as a
prototype for his model. It can be seen in a steam engine display in Sussex,
Wisconsin. It is said to produce about 200 horsepower at 90 RPM, had a 15"\ bore
and 36"\ stroke and a 10-foot diameter flywheel. The Vilter company still exists
in Milwaukee and now makes refrigeration equipment.

\bigskip
This masterfully built model runs flawlessly on air supplied from a tiny
aquarium air pump. Though others told Jerry he would not be able to achieve good
performance in a model this small if he insisted on scaling every part, he
proved them wrong.

\bigskip
The photoengraved name plate is typical of Jerry's devotion to detail. Notice
the hollow air line going into the large brass elbow. It has a functional
compression fitting and is made from a hypodermic syringe needle.

\bigskip
\textbf{U.S.S. Roosevelt Richard DeVynck U.S. Virgin Islands}

This model is now on display at the Bowdoin College Museum in Maine. To the left
is a detail of the ship's boiler. Below can be seen the stack and some of me
deck details. The model is left unplanked so that all the interior details can
be seen.

\bigskip
\textbf{1 -Cylinder 4-cycle overhead valve model airplane engine Ron 
Colonna, McKeesport, Pennsylvania}

(Above) Ron built this engine from a design by Eric Whittle of England. The
highly polished pieces and wood base make it a good display as well as a nice
piece of engineering.

\bigskip
\textbf{3-Cylinder engine, Jesse Brumberger, Macedon, NY}

(Above left) This radial model airplane engine was an entry in the 1996 Sherline
Machinist's Challenge.

\bigskip
\textbf{Assorted small projects, Robert Culpepper}

(Left) A small shop can tum out plenty of nice work.

\bigskip
\textbf{Robot Hand, Carl Hammons, Escondido, California}

Joe's partner Carl was interested in robotics and motion control. He built this
4"\ hand to test a concept he had in mind for gripping.

\bigskip
\textbf{Fantasy Gun, John Winters, Seattle, Washington}

Lost wax castings and machined parts are combined in this air powered, B-B
firing gun that looks as if it came straight from a Buck Rogers episode.

\bigskip
\textbf{Custom silver key ring, Jim Grabner, Leucadia, California}

The spiral and radial geometric pattern on this silver key ring helped it win a
blue ribbon at the San Diego County Fair in Del Mar. Projects like these are not
what comes to mind when most people think about "machine tools", but in the
hands of a creative person, a good tool makes many things possible. Jim used a
rotary table on the mill to create the patterns.

\bigskip
\textbf{Hula-hula radial engine, Russell Kutz, Clinton, Wisconsin}

(Left) This engine gets its name from the interesting action of the six
oscillating radial cylinders.

\bigskip
\textbf{1/6 Porsche piston and cylinder Pete Weiss, Escondido, California}

(Above) As part of a project to build a running 1/6 scale Porsche flat
6-cylinder engine, Pete has so far built a number of the components. See page
120 for more photos.

\bigskip
\textbf{Gap frame stamping press
Glenn Busch, St. Clair Shores, Michigan}

Here are two views of a solidly built and nicely finished model. The contrast of
brass and aluminum parts give it a very rich look.

\bigskip
\textbf{Gyroscope
Tim Schroeder, St. Joseph, Michigan}

(Right) This nicely finished gyroscope includes details like lightening holes in
the support arms and chamfered holes and edges on the wheel. Tim is a
professional photographer, so even the photos of his work are done with great
attention to detail.

\bigskip
\textbf{Marine engine and drill press/Water pump Scotty Hewitt, Van Nuys,
California}

Scotty's main project won 1st place in the 1995 Sherline Machinist's Challenge,
but he also took 5th place with this one. To the right is another of Scotty's
projects; an air powered water pump. Scotty's work always combines many
materials, skills and a lot of imagination. Notice how the wood bases add a
finishing touch like a good frame on a nice painting. For more photos and a
profile on Scotty see page 24.

\bigskip
\textbf{Radar study model
Frank Libuse, Carlsbad, California}

This waterline model was used to test radar targeting systems for antiship
missiles. A number of small deck fittings had to be fabricated from metal.
Simpler models were also made to see how much detail was needed for a missile to
be able to recognize and target a particular ship.

Frank is a pilot and industrial designer who started his own design firm and
industrial model shop after retiring from the Air Force. He is also Craig's
father. and Craig worked with Frank for several years in the design and model
business before starling his own design firm.

\bigskip
\textbf{Miniature Stuart 10V steam engine Chris Dinardo Springfield, Illinois}

The large hex bolt used as a display base really points out how small this
engine is. Despite its small size, all the working details are still there,
modeled in bronze, brass and steel.

\bigskip
\textbf{2-Cylinder marine engine
Raymond Hasbrouck, New Platz, New York}

(Below) This model exhibits a nice combination of materials and finishes. Notice
the engine-turned pattern on the base. The custom propeller is an interesting
project all its own. (To learn how to make one, see page 56.)

\bigskip
\textbf{Thimble steam engine Richard Long Wichita, Kansas}

(Above) This tiny butane powered engine drives a stamping mill. Using the
thimble as part of the design is a clever way to emphasize small size.

\bigskip
\textbf{Model airplane display engines
Edwin Teachworth, San Diego, California}

This display model of а 1911 compressed air engine was built for an exhibition
on the history of model aviation at the San Diego Aerospace Museum. Though it
looks like metal, it is made from cut and machined styrene components and
painted to look like metal. Styrene is easy to work with, glues together quickly
and is popular for modeling.

This is another display model from the same exhibit and is a model of a
Stringfellow steam model airplane engine. It is made from a combination of
materials including styrene, wood and brass. The original engine won a prize for
engine design in 1868 and developed about 1 horsepower. Display models need only
look like the prototype, while function is less important than looks, cost and
ease of building.

\bigskip
\textbf{Stop motion animation dog, Tom Brierton, Illinois}

With all the joint movements of a real dog, this framework is covered in clay
and then photographed one frame at a time as it is moved in progressive steps.

\bigskip
\textbf{Pre-lubricator for steam or air engine Salvatore Rubino, Naperville,
Illinois}

This device provides oil under pressure to lubricate bearings before an engine
is started. This extends engine life substantially since most wear occurs when
bearings are dry.

\bigskip
\textbf{Scroll saw and die filer conversion Milo Bresley, Bloomington,
Minnesota}

Mr. Bresley designed and built a die filer and scroll saw powered by his
Sherline lathe. The die filer is driven by the "Scotch yoke"\ principle. Many
people find the chief source of enjoyment in their hobby is designing and making
new accessories for their machines. It not only provides a fun and challenging
project, but your machine shop is that much more complete when you are done.

\bigskip
\textbf{Quick-change tool holders Roland F. Gaucher, Spencer Massachusetts}

This working model of an Aloris toolholder is built 3/10 the size of the \#1
size holder used for full size machines. These holders follow big machine
practice, allowing tools to be quickly locked onto a special dovetailed holder.
This is another good example of using the tools in your shop to build
accessories to make your shop that much better equipped.

\bigskip
\textbf{Model hot rod, Augie Histano, Miami, Florida}

This 1/25 scale hot rod won the top national award for model cars. Shown here is
just the engine and frame. Notice the scale Jaguar independent rear suspension
made up of almost 100 separate parts.

\bigskip
\textbf{Custom model boats and engines Don Martin, Sacramento, California}

Don's small shop turns out some excellent R/C drag boats. ЛИ his tools are
within easy reach and a vacuum cleaner rests under the 

(Above left) High performance machined out-drive components and exhaust tips
sparkle on Plum Nasty's transom.

(Left) A Connolley V-8 with supercharger sits on its test stand. The rig
provides readings on temperature and RPM.

Shown above is the front part of the hot rod engine being machined from an
aluminum block. Behind it is the photo of the actual Ford 427 engine Augie used
for reference in detailing his model. More on Augie and his award winning models
can be found in a profile on page 180.

\bigskip
\textbf{Double Corliss steam engine
Wilhelm Huxhold
West Hill, Ontario, Canada}

This beautiful and ambitious project demonstrates why retired machinist Wilhelm
Huxhold's work is considered among the best being produced. The closer you look
at every part, the better they look. Although Mr. Huxhold's shop is equipped
with many full-size machines, his favorite projects are very small in size and
are well within the capabilities of the tabletop machine tools discussed in this
book. Now that he is retired, he still puts in a full day's work, but he gets to
choose the projects.

\bigskip
\textbf{Triple expansion steam engine (left) and machinist's vise (above)}

This highly detailed steam engine won the 1997 Sherline Machinist's Challenge
contest. The vise is only a few inches long and duplicates every detail of the
original right down to the engraved angle scale. The handle is removable.

\bigskip
\textbf{Naval cannon display (left) and steam engine (below) Timmy Perreira,
Haiku, Maui, Hawaii}

This 17th Century 24-pound naval cannon is set in its own diorama. The ship's
deck setting adds a sense of purpose to the brass and oak cannon. Below is a
Rudy Kouhoupt-designed steam engine Mr. Perreira built from brass, aluminum and
cold rolled steel.
