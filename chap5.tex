\secrel{Chapter 5\ --- Using hand tools and abrasives}\secdown

The Nicholson® file company started manufacturing files in 1864 and produces
files that have become the standard for the world. Today (1998) you can still
purchase an 8" mill file, smooth, for \$6.12. There are cheaper files, but I
believe a good file is worth the extra money and is really a bargain when you
consider what can be accomplished with one. Files give a machined part its final
shape and finish. Too often I see parts that have been accurately made to size,
but lack that special appearance that a good craftsman can add. The "final
touch" is usually accomplished with a simple file, a good set of hands and a
good mind controlling the process. When machining parts for pleasure, you are in
a sense a sculptor, and your medium is metal. Parts should do more than just
work. They should have the detail to make them look perfect even if they are
scrutinized under magnification. Try as we may to eliminate the craftsman with
fancy CNC machines, it is still a craftsman's touch that produces "beautiful"
parts. You can't expect the average person to appreciate this extra effort, but
your peers will.

I am going to stay within the bounds of using files on small metal parts. The
material being filed must be hard enough to be successfully filed. Soft
materials like wood should be sanded.

\bigskip

\textsc{SINGLE-CUT FILE}

\textsc{DOUBLE CUT FILE}

\textsc{CURVED-CUT FILE}

\textit{The grooves in files can be cut at a number of different angles. For the
purposes of miniature machining, a single-cut mill file will be the most useful
tool.}
\bigskip

\textsc{MACHINIST'S FLAT BASTARD FILE}

\textsc{HAND FILE}

\textsc{PILLAR FILE}

\textsc{SQUARE FILE}

\textsc{ROUND FILE}

\textsc{THREE SQUARE FILE}

\textsc{HALF-ROUND FILE}

\textsc{KNIFE FILE}

\textit{Files come in all sorts of shapes and sizes. Each has a particular
purpose and having a good assortment on hand can make your life a lot easier.}
\bigskip

\secrel{File shapes}

Files come in all sizes and shapes. Round, fiat. triangular, square and half
round are just a few of the standard shapes. These shapes and sizes can be
ordered through mail order industrial suppliers. We have several listed on our
web site that have hundreds of styles listed, and it is up to the craftsman to
find the right tool to do the job. One point worth noting is that files can be
ordered with blank or "safe" edges. This is very helpful when filing up to a
shoulder. Swiss pattern needle files are sold in sets that have been
manufactured throughout the world and have become one of the real bargains for
the home machinist. A good needle file set is a "must".

\secrel{Riffles}

There is another type of file that not too many people know about. They are
called "rifles" or "rifflers" They are great to add detail to parts that have
nonstandard shapes. Riffles were developed for tool makers to put the final
touches on complex shapes.

\bigskip
\textit{Rifflers are used for filing the finishing touches onto complex shapes.
There are many choices of shape to choose from.}
\bigskip

The files have many cross sections and have a file on each end of the handle
that have been "bent" to get into tight places. When you get working at this
level of detail, engraving would better describe the work you are performing.

\bigskip
\textit{Needle files are also good for working in small areas and come in a
large selection of cross sections and shapes. Files are amazingly inexpensive
considering the work involved in making them.}
\bigskip

\secrel{Powered rotary files}

I'm sure everyone has seen kits that have a high speed motor to drive a wide
assortment rotary tools. Dremel is by far the largest manufacturer of these neat
little sets. They can be real handy but don't use them in a small cavity. If
they dig in, they bounce to the opposite side and you can destroy a part in a
split second. It is really worth the extra effort to machine good finishes in
cavities. They are very difficult to clean up later.

\secrel{Working with files}

A single-cut, flat mill file is my personal favorite. These files have ridges in
only one direction rather than two directions. This doesn't allow them to remove
material quite as fast, but when you are working with small parts, time isn't as
important as quality. Mill files have straight teeth angled to the body. This
creates a self-cleaning action. For general work such as deburring, I use a 6"
to 8" long file with a smooth cut.

When a file first comes in contact with a part, it should be allowed to find its
resting point before any pressure is applied. This keeps the edges from digging
in. The direction the file is driven should be at an angle that allows the teeth
to cut the part at an angle. If the teeth hit an edge squarely they have a
tendency to chatter. Remember, a file has a built in angled cutting edge.

\secrel{Keep your files clean and sharp}

Files should be replaced when they get dull. The sharpness of files should be
judged by their performance. When files start cutting on their ends better than
in the middle they are worn and it is time to replace them. A File should cut
with little "downforce". To cut properly they need to be clean. The grooves on
the backside of the cutting edges are the same as the flutes in a drill or mill
cutter. A drill wouldn't be expected to work with its flutes clogged and a file
will not work properly unless it is frequently cleaned. I use a small wire brush
that has the shape of a toothbrush.

\secrel{Precautions when working on spinning parts}

A file used on a lathe can be very dangerous. If the point of the file comes in
contact with the three-jaw chuck, it will be driven backwards into your hand.
These can be nasty injuries and can only be avoided by concentrating on the job
at hand. Machining is a process that requires hands to be in close proximity of
moving cutters and spindles. When operating manual machines, the only safety
device that really works is the brain keeping the hands out of trouble.

\secrel{Putting handles on files}

For safety, all files should have handles on them when they are used, but it is
especially important when lathe-filing. By the way, the proper way to seat a
file in a handle is to bang the handle end down on the workbench, using the
file's momentum to seat it into the handle. Never hold the handle end up and hit
the metal file end on the workbench.

\secrel{Magnifying eyewear makes miniature work easier}

I strongly recommend a magnifying headset for doing work like this. It is wise
to use the minimum amount of magnification that can be used to get the desired
results. Powerful lenses distort your surroundings so much they hurt more than
they help. They must be comfortable to wear. Magnifying glasses are inexpensive
and should also be considered a necessary tool for fine finishing.

\secrel{Deburring the edges and corners of a block}

If the edges on a small rectangular block need deburring, the amount of corner
break should be equal on all edges. This is when your eyes become more accurate
than you thought they were. You can

\bigskip
\textit{The photo above shows blocks in various stages of finish. The unfinished
block on the left has square. machined corners and a drilled hole. Its fly-cut
surface shows machining marks. The block in the middle has been lightly sanded
in two directions with 320 grit sandpaper on aflat surface and has had the
corners slightly softened with a flat mill file. The hole has been chamfered
with a chamfering tool. In addition to surface sanding, the right block has had
the edges beveled with a mill file and then sanded. The corners have been
slightly rounded off. The hole has been chamfered and lightly spot-faced with a
counterbore.}
\bigskip

easily distinguish a difference of only .005" (.127mm). If appearance is
important on this block I will go one more step to put the final touches on
exposed corners. The chamfered edges now have the appearance of three separate
edges joining. They would look better if these surfaces flowed together. After
the edges have been equally chamfered, use your file with light pressure and
file around the comer. This will round off the comer on one side. Sides two and
three are done the same way. The final comer will have a beautiful shape that
can be polished to perfection. This should only be done to comers that do not
mate with another part.

\secrel{Removing tool marks and final finishing}

The part is now complete, or is it? It has been milled to shape and carefully
deburred, but it doesn't have the look you want. You remember the parts you saw
which were built by one of the masters of our hobby. They looked more like
pieces of jewelry than mechanical parts. If this is what you want, the part
still needs a lot more work. The next step is removing tool marks. Nothing looks
worst to me than a surface that has been polished too soon. The shiny surface
will magnify the flaws.

\secrel{Polishing stones}

If you have a nice, flat, machined surface that shows a few tool marks, it will
not be improved by filing. I switch to abrasives. Inexpensive "polishing stones"
that come in a variety of shapes, sizes and grits are

\bigskip
\textit{A selection of small polishing stones. Keep them clean by frequently
dipping them in kerosene as you work. Work from rougher to finer grits to
achieve the finish you need.}
\bigskip

\textit{The photos in the background show the actual deck fittings from a
vintage Chris Craft boat. In the foreground is a model part that duplicates the
highly polished sheen of the chrome on the original bow light. The model is by
M. H. Ellett.}
\bigskip

available. These aren't things that are sold at your local hardware stores and a
mail order company would be your best source. I have found a 1/4" square shape
the most useful. The end can be shaped on a bench grinder to get into tight
spaces. The stone is dipped into kerosene frequently to keep the surface being
cut awash with fluid. I've found that by changing the direction of cut as much
as possible on each pass across the surface, the process will be speeded up. You
make as many passes as it takes to get rid of all tool marks. Do not try to do
large surfaces at one time. This is a very slow process and the average person
will get very sloppy if they try to take on too big an area. The next step is to
use a finer grit stone and remove the scratches left by the previous stone. This
process is repeated until you have a finish that is flat and free from all but
microscopic scratches. The next process is polishing the surface.

\secrel{Sandpapers}

1 have gotten excellent results with 320A wet and dry sandpaper glued to small
sticks, I buy these wooden sticks at hobby shops and prefer spruce. I glue the
sticks to the paper with instant "super glue" and make up a batch of them at one
time. The glue hardens immediately and the sandpaper can be trimmed to the stick
size. This glue works well because it doesn't seem to break down with the
kerosene that must be used to maintain a cutting action. Many times this is all
that is needed to finish a part to your satisfaction. Highly polished parts
don't always look good because the finish will exaggerate any flaws.

These "sticks" will work well for most metals, but may be to slow to remove deep
gouges in the harder materials. The 320A paper is the only type I would
recommend. Coarser papers don't cut as well. I believe a metal surface is just
too hard for them, and the maximum pressure that can be applied doesn't allow
them to work. On the other hand, finer grits like 400 or 600 cut so slowly they
can be a waste of time. If you want a better finish than you can get with 320A,
polish it.

320A wet and dry paper is also useful to finish flat surfaces. Larger surfaces
must be very fat or it will be easy for you to notice variations. The full size
sheet of paper is laid on a flat surface and soaked with kerosene. The part is
then dragged across the paper with little down pressure. This helps keep the
part square to the flat surface. I always use a surface plate. Again, I have
found I get better results by taking straight cuts from a variety of angles.

\secrel{Powdered and liquid abrasive polishes}

Most any abrasive powder can work, but diamond powder is by far the best and
also the most expensive. Most any abrasive type polish will work as well.
Automobile rubbing compounds will work and they should also be applied with a
stick. The diamonds come mixed with an oil. A small amount is put on a clean
stick and the stick is used like a polishing stone. The diamonds become imbedded
in the stick and will keep cutting as long as a very small amount of kerosene is
used. Of course, the diamonds will be eventually be worn and washed

\bigskip
\textit{A commercial "tumbler" uses abrasive particles to remove rough edges
from machined parts.}
\bigskip

away and more diamond compound will have to be added. You have to be careful you
do not contaminate the sticks with any grit and the part has to be meticulously
clean. You wouldn't polish a new Ferrari without washing it first. One little
piece of coarse grit can make a polished surface look as though it had been
attacked by a graffiti vandal.

\secrel{Tumbled finishes}

Production parts will be finished in a machine called a "tumbler". The parts are
put into the drum of the tumbler along with a quantity of an abrasive medium,
which comes in a variety of different shapes and sizes. When the machine is
turned on, the random abrasion between part and abrasive medium will rub the
tool marks off. For the home use, a small rotary tumbler used by the jewelry
trade may be helpful and costs less than industrial models. It isn't a fast
process, but the only labor is mixing the parts with the media and then adding
water with a very small amount of detergent. The media should outweigh the parts
by three times to keep the parts from damaging one another by contact. The
detergent keeps the media from getting slippery from oil which would stop the
cutting action. It will usually take several hours to get the desired results.

\secrel{Sandblasted and beadblasted finishes}

Sandblasting or beadblasling puts a soft, dull finish on metals. If you are
modeling a part where you need to use machining but you want the finished part
to look like a casting, sand- or beadblasting the part can give you the finish
you are looking for. It's a pretty "low tech" process. All you need is a
compressor, a sandblasting "gun", a hopper and a bag of silica sand, but unless
you plan to do a lot of it the mess it generates just isn't worth doing it
yourself. It's not very expensive to send your parts out for sandblasting and is
probably worth it not to have to deal with the very abrasive silica dust.
However, taking a few small parts to a professional sandblaster is usually not
practical. Their equipment is too large, and they are not really not set up to
handle small parts. An automotive sand blaster that has an enclosed box for
sandblasting small parts is probably your best bet.

If you do try and do it yourself, be sure to wear the proper eye protection and
the correct type of respirator mask. The fine particles of silica are the same
as what makes up glass, and getting them in your lungs can cause silicosis,
which is similar to asbestosis.

\bigskip
\textit{Here's what M.H. Ellett's Chris Craft runabout looks like finished and
in the water. The detailed and highly polished fittings make this model look
like the real thing no matter how close you get.}
\bigskip

\textit{Jerry Kieffer is seen here at a model exhibition with his 1/12 scale
Brown \& Sharpe Universal Milling Machine. He superdetailed his model based on
photos in a reproduction of a Charles A. Strelinger \& Co. machine-tool catalog
from 1895. Jerry is in the process of outfitting a complete 1/12 scale overhead
belt driven machine shop as it might have been found in the late 1800's. As is
typical of Jerry's models, every detail is represented to exact scale. Shown
below is the dividing head which sits on the mill's table.}
\bigskip

\textit{Jerry's fingers give a sense of scale to the small size of the dividing
head. The indexing plate has the correct pattern of 119 holes, each only .008"
in diameter.}

\secup
