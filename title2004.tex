\begin{titlepage}

\begin{centering}

.

{\Huge\bigskip\textsc{Tabletop Machining}\bigskip}

{\Large \ldots A basic approach to making small parts on miniature machine
tools}

{\Huge \bigskip \textbf{Joe Martin} \bigskip}

{\large \bigskip \textsc{Design, typesetting, illustration and photography by
Craig Libuse } \bigskip}

\bigskip

\copyright\ 2004 by Joe Martin

\bigskip
First printing\ -- 1998, second priming\ -- 1999. third printing\ -- 2001,
fourth printing\ -- 2004

\bigskip
All rights reserved. Printed in China by Winco (H.K.) Co. Ltd.

\bigskip
The author takes no responsibility for the use or application of any of the
materials or methods described in this book. All miniature projects shown were
either made or could he made using tabletop machine tools similar to or
identical to those described in this book.

\bigskip
To order additional copies of this book call:

Toll Free in the USA\ -- (800) 541-0735 $\diamond$ International\
-- 1-760-727-5857 

or write to: Joe Martin. 3235 Executive Ridge, Vista.
California 92081

\bigskip
{\Large \textbf{ISBN 0-9665433-0-0}}

\end{centering}

\end{titlepage}

\secdown\secdown

\secrel{Machining is not a "paint-by-numbers"\ process}

If you are looking for a book that will give you complete, step-by-step
instructions on how to build your particular machining project, this is not it.
In fact, that book probably does not exist. What this book will give you is all
the basic knowledge you need to start machining metal. Your imagination plus the
information in this book will allow you to make just about anything. The many
photos showing what others have done are here to spark your imagination. None of
the projects shown in the photos in this book came with detailed instructions.
Most came with none at all. They are, for the most part, not beginner projects.
I'd suggest you start with a relatively simple project and apply what you learn
from this book. As your skill and experience increase, you'll be ready to tackle
anything you see here. Read the parts about tools and materials. Read the parts
about speeds and feed rates. Study the photos of setups carefully. Everything
you need is right there, but you have to use some brainpower to apply it to your
projects. The level of satisfaction you achieve will be directly related to the
amount of effort you are willing to put forth.

The book is now in its fourth printing, and some have commented that it doesn't
contain enough project plans. I have avoided adding a lot of "how to" plans in
order to concentrate on the general skills, craftsmanship and techniques needed
to create a good part. These will never be found in a set of plans. For those
looking to take what they've learned here and apply it to a specific project,
there are many sources of kits and plans on Sherline's web site at
\url{www.sherline.com}. Several magazines like \textit{The Home Shop Machinist}
and \textit{Machinist's Workshop} offer new plans in every issue.

\secrel{Thanks to those who helped}

Joe Martin and Craig Libuse would like to thank all of those who took the time
to read this book word for word and sent in suggestions for corrections in the
previous printings. Our thanks go to Marc Cimolino, Jim Clark, Glenn Ferguson
Jr., Mort Goldberg, Alan Koski and especially Huntly Millar for their extremely
diligent, voluntary efforts. Among other things, this book addresses the issue
of quality and the quest for perfection, so we have made every attempt to
eliminate any typographical errors. We welcome your input in a continuing effort
to improve the quality of this book. Though rarely achieved, perfection is a
goal always worth pursuing.

\secup
\secup
