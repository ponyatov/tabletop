\secrel{Chapter 1\ --- Getting information on machining}\secdown

\secrel{The book every machinist needs}

People new to machining are bound to have many questions. How fast should I turn
my cutter for a particular material? How do I figure out the pitch of a gear?
What are the tolerances for a "sliding fit" or a "press fit?" The most
traditional sources for this type of information are books that can be found in
your local library or bookstore. On-line sources like \url{www.amazon.com} and
\url{www.barnesandnobel.com} make shopping for books even easier.

The single source I have always turned to first is \emph{Machinery's Handbook},
which is published by Industrial Press. This book has been published since 1914
and is updated every few years. The small but thick book has over 2500 pages of
information. A larger "easy-read" version is also available.

I used to recommend this book without reservation and always suggested that a
purchaser get the most recent version available. After picking up a copy of the
latest edition, however, I am inclined to change my opinion on the subject. 1
have noticed that the direction Industrial Press has chosen to go with the book
is to direct it more to the engineer than the machinist. Information about
machining has been deleted to make room for information about the strength of
gear teeth and other details that will never be of concern to the average
machinist. In light of this development, I now recommend that rather than
coughing up \$80.00 for the latest edition, you can get all the valuable
information you need about machining and save money too by buying a used earlier
edition from a used book store or over the Internet. Auction sites like eBay.com
often have issues for sale. The most valuable information on metals, formulas
and processes hasn't changed significantly for the past few years, and the 20th
will probably provide all you need for a lifetime of machining. Use the money
you save to buy an accessory or another book.

\bigskip\textit{Machinery's Handbook should be considered the basic reference
source for metalworking questions.}

\secrel{The Internet gives instant access to a world of information}

If you are one of those who is resisting getting hooked up to the Internet, you
are putting yourself at a great disadvantage in today's world. With computers as
cheap as they are now, there is no excuse for missing out on this great
resource. A connection to the Internet gives you access to search engines that
can find information about any subject almost instantly. Newsgroups can also put
you in touch with others who share your interests and allow you to ask and
answer questions about your hobbies. You also have instant access to product
information about tools, accessories, raw materials and other resources for your
hobby or business. More importantly, you will be missing out on a lot of fun,
as, in addition to pure information, the Internet also gives you access to
personal sites that feature projects made by many talented individuals. You may
even find that you want to have a web site of your own where you can display
your machining accomplishments to the world.

With a computer, a scanner or digital camera and an Internet connection, you are
plugged into a new world of information. Internet access can cost as little as
\$20 a month or less for a regular phone line connection to \$50 or more a month
for higher speed connections. It just depends on how fast you want the pictures
and information to load on your screen. They say "time is money." and in this
case, less time is more money.

\secrel{Search Engines do the work for you}

It amazes me that I can type in a few words on a search engine and have it
search the entire worldwide web for those words in a few seconds. For example. I
just did a search for the word "lathe"\ on my favorite search engine,
\url{www.google.com}. It took just .15 seconds to return a list of 183,000 sites
using the word lathe. Of course, if I were looking for more specific
information, I would have refined my search with other words like a brand name
or model number, but the fact that a free resource can search so much
information so quickly is simply incredible. Keep in mind that, unlike an
encyclopedia, anyone can post information to the Internet, so the responses
turned up by your search have to be judged by you as to their authenticity and
reliability.

\secrel{Newsgroups and chat groups}

There are special interest groups on virtually any subject you can think of
where people of like interest can ask and answer questions by e-mail. Machinists
have many groups they can join that can be about machining specific projects
like steam engines or about using specific machines like Sherline tools. For
example, the Sherline group can be found by going to
\url{http://www.yahoogroups.com} and doing a search for the word "Sherline". As
of June, 2001 the group had 579 members.

Newsgroups generally offer a list of "threads"\ of conversation that you can
read. If you feel like responding, you can send your message to the group via
e-mail. Your message will be posted and others can respond to it, Chat groups
are more personal and are more like talking to other people through your
keyboard. You type a response and the answers pop up on the screen as fast as
others who are on-line at the time can type them in.

\secrel{Model engineering societies and clubs}

There are local and national groups for modelers and machinists that have
meetings and shows. This can be a good source of meeting people in your area
that not only share your interests but who can also help you in learning a new
skill. They may have tools that you don't for special jobs or they may have
experience in an area like heat treating or casting that you need to learn
about. A couple of examples are the North American Model Engineering Society in
the Midwest (\url{www.modelengineeringsoc.com}) and the Pacific Rim Model
Engineering Society in the West (\url{www.evmes.org}). These organizations each
have an annual show that is open to the public. Many areas also have local
metalworking clubs that meet monthly. This can be a great source of information
for the new machinist, because it puts you in touch with people with years of
experience who are willing to share their expertise with you. A list of local
clubs that have web sites can be found at
\url{http://www.metalworking.com/clubs.html}.

\secrel{Robot battles on TV}

This is a little off the subject of learning about machining, but a new trend on
television has the potential to bring a new and younger audience into machining.
Several television shows now offer competition for home-built robots that
include speed and agility tests as well as the ultimate test of a person's
building ability\ --- competition. While some robot purists see this as a
negative portrayal of robotic abilities, I see it as an excellent introduction
to the need for precision metal parts for a whole new group of young thinkers
and builders. Shows like Comedy Central's BattleBots® and The Learning Channel's
Robotica® follow the lead started by a British show called "Robot Wars." Events
include racing around a figure "8" course or through a maze of obstacles. The
final winners are usually determined in bot-to-bot combat. Although these
"robots" are actually just armored radio-controlled vehicles, they do exhibit
many innovative features and it requires a great deal of skill to build a
winner. Despite the juvenile over-dramatization of some of the shows, the heart
of the matter is that they are getting kids thinking about actually building
things that will stand up to the riigors of competition. I feel this is a
healthier trend than encouraging them to simply lose themselves in video games,
and I hope that its success will lead to more bright kids finding satisfaction
in building things. This is the group that will become the future engineers and
designers who will shape the inventions of our future.

\begin{framed}

\secrel{So universal they mode toolbox drawers just to hold it}
 
Toolboxes made for machinists have a number of flat, felt lined, drawers to hold
cutting tools and measuring instruments. In addition, many have an oddly shaped
vertical drawer right in the center. It was designed to hold Machinery's
Handbook. It was and still is considered as important as any other tool in the
machinist's toolbox.

\bigskip\textit{This old machinist's toolbox was made by H. Gerstner \& Sons.
The company is still producing high quality wood toolboxes. The clock must have
been added by the machinist who wanted to know when it was getting to be
"quittin'time".}

\end{framed}

\secup
