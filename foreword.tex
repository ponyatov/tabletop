\secrel{FOREWORD}\secdown
\secdown

\secrel{What is "tabletop machining"?}

Tabletop machining is about operating miniature machine tools. These are
machines that can be picked up and set on a small bench or, if need be, a
kitchen table, and used to build precise metal parts. They are inexpensive
compared to their full- size shop equivalents, but are just as versatile and
accurate as long as the size of the part is appropriate for the machine. The
"Unimat" was the first miniature lathe mass produced and well known. Thousands
of Unimats were sold, and today many are still in usc . Il had a wide variety of
accessories manufactured for it and a price that was affordable. A number of
other miniature machine tools have been manufactured since the Unimat, and the
company I own, Sherline Products Inc .• has become lOday's leader for this class
of machine. I believe the fact I am both a hobbyist and toolmaker gave me more
insight into what our customers needed when it comes to both accessories and
instructions.

\bigskip\noindent
\textit{The original Unimat lathe was the first miniature machine tool to
achieve intenational ppopularity. It came in professional looking wood box and
offered a versatile design and many accessories at a reasonable price. It's
two-rail bed design made it too flexible for jobs requiring a high degree of
accuracy, but it introduced many people to the fun of machining in miniature.}
\bigskip

\secrel{Beating the system}

For me there has always been something special about projects that have been
built on these small machines. The machinist who works with miniature machine
tools will have beaten the system by not spending thou sands of dollars on tool
s . These craftsmen build beautiful projects for enjoyment, not wages. These are
special people who may suddenl y have an urge to accurately build that model
they have dreamed of for years. The machini sts who are succe ssful will realize
there is a learnin g c urve involved in accomplishing thi s. Thi s book is about
shortening that learning curve and giving you a new sense of what craftsmanship
is all about.

\secrel{Not just the ``how'' but also the ``why''}

The tables and charts can be found in Mochil/ery S Handbook , and I don ' t plan
to duplicate them in this book. Library shelves are full of books of thi s
nature. The infonnation in thi s book won't be found in charts and graphs. I' m
going to attempt to give you the infonnation to actually start making "parts".
Instructions that tell you " how" to do ajob too often sk ip the most basic
information, and that is "why" you would want to do a job thi s way or that way.
I believe the c ustomers who purchase miniature machines are intelligent enough
to find the spec ifi c infonnation they need al a library. These customers just
don't happen to know much about machining. However, I also believe this book
contains enough general rules lO get a job don e. Get started on a project as
soon as you have your tool s set up and working.
Read a littl e, machine a littl e. Never cut metal without a plan that includes
dimensions. "Making chips" without a plan can develop terrible work habits. Thi
s trade has few choices when it comes to parts fitting together. To work in uni
son they must be accurate, and your first task should be to make parts "to
size".

\secrel{How to read this book}

A book like this doesn't need to be read from front to back like a novel. You
will probably skip around reading first the sections that interest you the mosl.
Therefore, thi s book may seem at time s to be redundant. I have attempted to
make each chapter relatively complete in and of itself. and some rules apply to
more than one machining operation. Some of the more important ones may be
repeated wherever they apply. To keep you interested and make the book more fun,
we have included many pictures of actual projects and the people who made them.
The examples of what has actually been done using tabletop machine tools speak
more eloquently about their capabilities than anything I could say.

\secrel{Why Sherline lools are used in Ihe examples}

I must say up front that Sherline tools will be used in the examples throughout
this book. It is not my intention to use this book as a tool to sell Sherline
tools, but rather to use these tools to demonstrate the techniques I am
discussing. The reason should be obvious; that is, they are what I have
available and what I know the most about. The principles involved in using these
tools are pretty much typical of all machine tools, even larger full size shop
tools, so what you learn through these examples should be able to be applied to
whatever brand of tools you are using. Also, we have sold many thousands of
these tools over the past twenty-five years, so the knowledge specific to Sheri
inc tools wi ll be of additional benefit to those of you who are using them as
Yf)U work with this book. In addition, I hope the infonnation I've included
about how thi s tool line was developed and how our business is run might
inspire some of you to follow your dreams and start a business of your own,
whether it is in the area of machining or in any area that interests you.

\bigskip\textit{Craig Libuse is seen at the drawing board with
author Joe Martin. Craig has been doing all of
Sherline 's illustrations, instruction sheets, magazine
advertisements and catalogs since shortly after Joe
started the company in the mid-1970's. He ran his
own graphic design studio for 22 years doing
Sherline 's work on contract before coming on board
full time as Marketing Director in 1995.}\bigskip

\bigskip\textit{According to builder Edward J. Young of Mobile.
Alabama, this model Stuart 1OH steam engine runs
"smooth as silk " when powered by compressed air.
The inset photo shows the plexiglass cover he made
to replace the plate over the valve so its action can
be viewed as the engine runs.}\bigskip

\secup
\secup
