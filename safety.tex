\secrel{Safety rules for power tools}\secdown\secdown

\textbf{A patternmaker's interview for employment}

\bigskip
One of the best patternmakers I ever knew apprenticed in the trade for many
years with his father. When he went to work for U.S. Steel in their pattern
shop, the foreman who was interviewing him for the job asked him to hold out his
hands. When the foreman could see that the applicant still had all ten fingers,
he was hired. The foreman could see from his work that the patternmaker was a
good craftsman, but he figured that if he had been working in the trade that
long and still had all his fingers he must be a good, safe worker too, and that
was just as important.

Spinning tools that are powerful enough and sharp enough to remove metal can
also remove just about anything else that gets in their way. Though less
dangerous than their larger full size shop counterparts, small power tools can
still cause serious injury to those who don't show them the proper respect. Even
hand tools used improperly can cause injury. Talking about safety is not nearly
as fun as talking about the beautiful miniature machining projects in this book,
but working safely is part of the skill of a good craftsman.

Working safely is simply a series of habits that you develop. Once they become
habits, it takes no longer and is no less enjoyable to work that way than to
work with unsafe habits. Injuries definitely take the fun out of working with
tools, and fun is what miniature machining is all about. Please read these rules
and apply them until they become habits so that you can enjoy your hobby to the
fullest.

\begin{enumerate}
\item\textcolor{Red}{KNOW YOUR POWER TOOL}\ --- Read the owner's manual
carefully. Learn the tool's application and limitations as well as the specific
potential hazards peculiar to this tool.

\item\textcolor{Red}{GROUND ALL TOOLS}\ --- If a tool is equipped with a
three-prong plug, it should be plugged into a three-hole receptacle. If an
adapter is used to accommodate a two-prong receptacle, the adapter wire must be
attached to a KNOWN GROUND. Never remove the third prong. (See drawing on next
page.)

\item\textcolor{Red}{KEEP GUARDS IN PLACE}\ --- and in working order.

\item\textcolor{Red}{REMOVE ADJUSTING KEYS AND WRENCHES}\ --- Form a habit of
checking to see that keys and adjusting wrenches are removed from the tool
before turning on your machine.

\item\textcolor{Red}{KEEP WORK AREA CLEAN}\ --- Cluttered areas and benches
invite accidents.

\item\textcolor{Red}{AVOID DANGEROUS ENVIRONMENT}\ --- Do not use
power tools in damp or wet locations. Keep your work area well illuminated.

\item\textcolor{Red}{KEEP CHILDREN AWAY}\ --- All visitors should be kept a safe
distance from the work area.

\item\textcolor{Red}{MAKE WORKSHOP KID PROOF}\ --- with padlocks, master
switches or by removing starter keys.

\item\textcolor{Red}{DO NOT FORCE TOOL}\ --- Do not force a tool or attachment
to do a job for which it was not designed. Use the proper tool for the job.

\item\textcolor{Red}{WEAR PROPER APPAREL}\ --- Avoid loose clothing. neckties,
gloves or jewelry that could become caught in moving parts. Wear protective head
gear to keep long hair styles away from moving parts.

\item\textcolor{Red}{USE SAFETYGLASSES}\ --- Also use a face or dust mask if
cutting operation is dusty.

\item\textcolor{Red}{SECURE WORK}\ --- Use clamps or a vise to hold work when
practicable. It is safer than using your hand and frees both hands to operate
the tool.

\item\textcolor{Red}{DO NOT OVERREACH}\ --- Keep your proper footing and balance
at all times.

\item\textcolor{Red}{MAINTAIN TOOLS IN TOP CONDITION}\ --- Keep tools sharp and
clean for best and safest performance. Follow instructions for lubrication and
changing accessories.

\item\textcolor{Red}{DISCONNECT TOOLS}\ --- Unplug the tool before servicing and
when changing accessories such as blades, bits or cutters.

\item\textcolor{Red}{AVOID ACCIDENTAL STARTING}\ --- Make sure the switch is
"OFF" before plugging in power cord.

\item\textcolor{Red}{USE RECOMMENDED ACCESSORIES}\ -- Consult the owner's
manual. Use of improper accessories may be hazardous.

\item\textcolor{Red}{TURN SPINDLE BY HAND BEFORE SWITCHING ON MOTOR}\ --- This
ensures that the workpiece or chuck jaws will not hit the lathe bed, saddle or
crosslide. and also ensures that they clear the cutting tool.

\item\textcolor{Red}{CHECK THAT ALL HOLDING, LOCKING AND DRIVING DEVICES ARE
TIGHTENED}\ --- At the same time, be careful not to overtighten these
adjustments. They should be just tight enough to do the job. Overtightening may
damage threads or warp parts, thereby reducing accuracy and effectiveness.

\item\textcolor{Red}{WHEN WORKING THROUGH THE SPINDLE, DO NOT LET LONG, THIN
STOCK PROTRUDE FROM THE BACK END OF THE SPINDLE SHAFT}\ --- The end of
unsupported stock turned at high RPM can suddenly bend and whip around.

\item\textcolor{Red}{It is not recommended that the lathe be used for
grinding.} The fine dust that results from the grinding operation is extremely
hard on bearings and other moving parts of your tool. For the same reason, if
the lathe or any other precision tool is kept near an operating grinder, it
should be kept covered when not in use.

\item\textcolor{Red}{WEAR YOUR SAFETY GLASSES}\ --- Foresight is better than NO
SIGHT! The operation of any power tool can result in foreign objects being
thrown into the eyes, which can result in severe eye damage. Always wear safety
glasses or eye shields before commencing power tool operation. We recommend a
Wide Vision Safety Mask for use over spectacles or standard safety glasses.

\end{enumerate}

\secrel{ELECTRICAL CONNECTIONS}

The power cord used is equipped with a 3-prong grounding plug which should be
connected only to a properly grounded receptacle for your safety. Should an
electrical failure occur in the motor, the grounded plug and receptacle will
protect the user from electrical shock. If a properly grounded receptacle is not
available, use a grounding adapter to adapt the 3-prong plug to a properly
grounded receptacle by attaching the grounding lead from the adapter to the
receptacle cover screw.

\textcolor{Red}{NOTE}: Electrical circuits designed into the speed control of
the Sherline lathe or mill read incoming current and automatically adapt to
supply the correct 90 volts DC to the motor. As long as you have a properly
wired, grounded connector cord for your source, the machine will operate on any
current from 100 to 240 volts AC and 50 or 60 Hz. without a transformer\note{The
first DC units built in early 1994 did not include the circuits to adapt to
other currents. The capability to include that feature was not available to
Sherline at that time. As soon as it was, it was included. If you think you may
have an early DC model, remove the plastic speed control housing and look. for a
label on the aluminum speed control frame. If it has a small metallic label on
top of the frame that lists input voltage as 120 VAC, DO NOT ATTEMPT TO CONVERT
THIS UNIT TO OTHER CURRENTS. Models that can be used with any current have a
paper label on the end of the speed control frame which lists the model number
as KBLC-240DS.}. This should include just about any country in the world. Prior
to 1994, an AC/DC motor was used. Use the AC/DC motor ONLY with the power source
for which it was intended. It will not automatically adapt to any other current
and using it with an improper power source will bum out the motor or speed
control.

\bigskip

GROUNDING TYPE 3-PRONG PLUG

PROPERLY GROUNDED TYPE OUTLET

USE PROPERLY GROUNDED RECEPTACLE AS SHOWN

PLUG ADAPTER

GROUND WIRE

\bigskip
\textit{Proper grounding of electrical connections.}
\bigskip

\textcolor{Red}{Older AC/DC motors available from Grainger}\bigskip

Sherline's supply of older AC/DC motors is slowly being depleted. A very large
run must be custom ordered to get more, and this is not economically feasible.
However, the Grainger catalog stocks a 1/5 horsepower motor identical to the one
used on early Sherline tools. The catalog number is 2M139. They have locations
in every state and can be found in the Yellow Pages under "Electric Motors".
Their web address is \url{www.grainger.com}. Your other option would be to
upgrade your motor and speed control to the newer, more powerful DC version.

\bigskip
\begin{centering}

"Common sense is instinct, and enough of it is genius."

--- Josh Billings

\end{centering}
  
\secup\secup
