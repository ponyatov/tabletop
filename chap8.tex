\secrel{Chapter 8\ --- Coolants and cutting oils}\secdown

\secrel{Coolant in industrial applications}

When you are working with machines as small as Sherline tools, calling the
fluids "coolant"\ can be a bit misleading. In production machinery, these fluids
are used to lubricate and cool the cutting process. The entire part will be
flooded with coolant and it will be carrying heat and debris away. Notice that I
mentioned debris. Chips can be made at a very rapid rate, and it takes a large
volume of cutting fluid to control these chips. Most modern machines will use a
ninety-five percent water based mixture for cutting metal. Water is still one of
the best heat transfer materials on earth despite its ready availability and
almost negligible cost.

\secrel{Water soluble coolants vs. oil type coolants}

There are many types of water soluble coolants available for different
materials. Water soluble oil will mix with water and was very popular until the
new synthetic cutting oils came on the market. Soluble oil will go rancid if it
lies still, especially in warm weather. If pure oil were used on a modern
machine it would probably catch on fire. Oil type coolants are still used in
machines that use the oil to lubricate both the machine and the cutting process.
An "automatic screw machine"\ is a good example because of all the mechanisms
that are exposed to the machining process. Now let's take a look at the use of
coolant in a home shop environment.

\bigskip
\textit{In industrial applications, parts being cut are flooded with coolant to
carry away heat and debris. In the home shop, cutting loads are lower, less heat
is generated and fluids are used as a lubricant to keep material from sticking
to the cutting tool. Flooding a part with coolant is messy and not necessary.}
\bigskip

\secrel{Coolant in the home shop}

The purpose of "coolant"\ in the home shop is to keep material from sticking to
the cutting tool. When material sticks to the tool, the flutes may clog up,
generating heat from friction and stopping the cutting action. If you are
determined to machine without coolant, you'll be breaking cutters and ruining
work. You really must use it to be successful, and any type of coolant is better
than none. I usually use an oil type and apply it sparingly with a small acid
brush. You don't want to use an expensive brush for this operation because there
is a good chance the brush may get a haircut if it gets caught by a cutter.

\secrel{Purchasing coolant in small amounts}

Purchasing coolants in small amounts is a problem encountered by most home
machinists. If you went to a good hardware store and asked for cutting oil, they
would probably try to sell you thread-cutting oil. This type is available in
small amounts, but it has a very high sulfur content and it is black. It's dirty
and smelly, but it does help produce a good thread on difficult-to-machine
metals that "tear"\ in the cutting process. For machining, however, I would stay
away from it like the plague. The next place you might try to purchase cutting
oil would be an industrial supplier or a mail order company. They usually don't
sell small quantities\note{Some industrial suppliers now offer cutting fluids in
quantities as small as 1 gallon. Check with your favorite supplier. The higher
unit price is well worth the savings compared to buying far more than you
need.}, and they try to sell a mixture of ingredients that is normally used for
tapping holes. The problem is that when working with small machines you often
have your nose close to the machine, and the bad smell makes that a disagreeable
prospect. I would even consider it a health hazard if you constantly worked with
it. The easiest way out would be to buy five gallons of an oil type cutting
fluid, the smallest amount sold, from a local oil company and you'll have a
lifetime supply for working with miniature machine tools. You might get some
fellow machinists to split the purchase with you and you'll all have plenty.

\bigskip
\textit{For home use, a small amount of lubricant applied with an inexpensive
brush is sufficient. Avoid smelly, surfur laden fluids like oils used for
tapping threads.}
\bigskip

\secrel{Some precautions when using cutting oils}

Cutting oils have a higher flash point than regular oils which reduces the
chance of fire. Kerosene has been used with a mixture of oil and will give good
finishes on aluminum, but I wouldn't use it because of its smell and low flash
point. Automobile oils are too "slippery"\ and smell too bad from detergents to
work with. The surface finish will suffer from the use of these types of oil. To
summarize, don't flood the cutting surface on a miniature machine tool unless
you want to make a big mess. Apply a small amount of cutting oil with a small,
inexpensive brush and also brush the chips away with this same brush. Avoid
smelly fluids. Wear eye protection to keep not only debris out your eyes, but
also cutting fluids.

\secup
