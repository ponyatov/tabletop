\secdown\secrel{INTRODUCTION}\secdown

\secrel{The essence of ``craftsmanship''}

I wrote the introduction to this book last. That's because when I started
writing, I didn't quite know where I was headed. I knew that over the years I
had written many instructions for our products which contained enough knowledge
and advice to be valuable. I also figured I could start writing answers to
questions that had been asked of me over the years. I could fill the remainder
of the book with pictures and charts and end up with a book that wouldn't be any
different or better than what was already out there. For me, therefore, the most
important part was to try and instill in a potential machinist the value of good
craftsmanship. Great craftsmen not only get the job done, they add a certain
"look"\ to parts they build. It is almost a signature. I have seen the same part
made by two different craftsmen using the same drawing. They were both highly
skilled toolmakers. Both parts met the specifications perfectly, yet I could
easily tell who built each part. Machining should be considered a form of art.

\secrel{Some pretty good advice}

\bigskip\textit{Professional photographer, Tim Schroeder of Michigan built
these five identical Stirling hot air engines to polish his skill as a new machinist.
By making each part five times, he was able to get in more machining time with
each setup and learn more in a shorter period\ldots a pretty good way to
learn.}\bigskip

On the wall of my Uncle's shop when I was a boy was a sign which I still
remember. I'm not sure who said it. but I think it expresses what I'm trying to
say pretty well. It said:


\bigskip
\textcolor{Cyan}{``A man who works with his hands is a laborer.}

\bigskip
\textcolor{Cyan}{A man who works with his hands and his brain is a craftsman.}

\bigskip
\textcolor{Cyan}{A man who works with his hands, his brain and his heart is an
artist.''}
\bigskip

When I was building model aircraft, my friends and I had an interesting way of
judging the quality of a model. We would set the model aircraft on the ground
and start backing away from it until it looked good. A three-foot model would be
considered superb and a fifty-foot model was one that was pretty crude. There
were also models that wouldn't look good no matter what the distance was or the
viewing angle. In those cases, the failure was in the design, and the best
craftsman in the world can't make a bad design look good.

\secrel{The best design is usually not your first design}

The home machinist usually has more control over a design he is working with
than a professional does. Don't use the first idea that comes into your head
without proving to yourself that it's the \emph{best} way. When a product has
been designed properly, no one would even consider building it in a different
way.

It is the way it is supposed to look because it's obvious. Unfortunately, these
are the designs that are the hardest to come up with. They are also the designs
you will get the least credit for even though they are your best. The assumption
is that the obvious solution is also the easy solution, but this is usually not
the case. The home craftsman also doesn't have to work within the constraints of
commercial products where costs limit your choices. For us, time is not money,
it's fun!

This is what craftsmanship is all about. Too few citizens really appreciate what
good craftsmen do, Because their work doesn't fail it is taken for granted. A
good craftsman can tell at a glance when someone's work is better than his, and
he can start improving his work to be Number One, It is almost a form of
competition between craftsmen where time and quality are considered at the same
lime. Do you think Michaelangelo would be considered a great artist if he had
only carved one statue and painted one picture? He produced so much good work in
his lifetime that he set a standard that is still sought after today. One good
part doesn't make you a craftsman. You are judged on the body of your work.

\bigskip\textit{Author Joe Martin is shown with some of the miniature machine
tools produced by Sherline Products. The small size of miniature machine tools
makes them easy to use and not too intimidating for new machinists.}\bigskip

I not only wanted my writings to be useful to the hobbyist/machinist who builds
parts for pleasure. but also to those future craftsman who want to build parts
that have that "look". Please realize the parts being referred to in this book
are not production parts. Machinists who produce these kinds of parts have the
training and skill to make automatic machines build good parts. The only thing
an automatic machine will manufacture automatically is scrap. It stilt takes
that craftsman's touch to make machines run perfectly. The parts being discussed
in this book will be parts built one at a time... "one off. These parts are
usually part of another assembly that would be considered the final product.

\secrel{You don't become a machinist by buying a machine}

You should strive from the beginning to make better and more accurate parts than
you think you need. Work to closer tolerances than the job demands. Be on the
lookout for ways to make a job easier or better. I hope you will enjoy the
process of creating accurate parts from raw metal. Buying a machine won't make
you a machinist, but using it along with the skill and knowledge you acquire
along the way eventually will.

\secrel{What new machinists like most and least}

If you are new to machining, you may find it to be either one of the most
rewarding skills one can learn or the most frustrating thing you have ever
attempted. What makes machining fun for some is the complexity and challenge.
The same thing will drive others up the wall. One person may be overjoyed
because he can now make parts that were not available for purchase. Another may
wonder why he just spent all day making a part that is similar to one he could
have purchased for two dollars. (The difference, of course, is that it is not
the same as the two dollar part\ --- it is \emph{exactly} the part needed.)

\bigskip\textit{Jewel-like projects like this miniature marine winch are a
showcase for the kind of craftsmanship machinists strive for. Being able to
display your work on a desk or coffee table or even carry it with you in your
pocket is an advantage of working on small projects.}\bigskip

\secrel{There are no shortcuts}

Machining is a slow process because parts are made one at a time. The
interesting thing is, a skilled machinist may take almost as long to make the
same part as a novice. Shortcuts usually end in failure. Unlike some other
trades, mistakes cannot be covered up. There are no erasers, white-out or
"putting-on tools" for machinists. You simply start over. Do a lot of thinking
before you start cutting. To expand a little on an old rule: "\emph{Think three
times, measure twice and cut once!}"

\secrel{Anticipation of a tool's limitations is the crafts man's strength}

The skill in machining isn't just "moving the dials". It is a combination of
engineering and craftsmanship. A file is just as useful a tool to a machinist as
a multi-thousand dollar machine tool. Tools "deflect"\ or bend under load, and
anticipating this bend is what it is all about. Sharp tools deflect less than
dull tools, but with each pass the tool dulls a little and the deflection
becomes greater. If you try to machine a long shaft with a small diameter, the
center will always have a slightly larger diameter than the ends because the
part deflects away from the tool where it has less support. You can go crazy
trying to machine it straight, or you can simply pick up a good, flat mill file
and file it straight in a few moments. Machine tools will never replace the
"craftsman's touch", and machining is a combination of both good tools and good
technique.

\secrel{The great parts about running a business like this}

I'm a hobbyist who has been lucky enough to make a living at a hobby I enjoy. I
own and manage Sherline Products Inc. and enjoy coming up with new products.
After working at if for over twenty-five years, this has become more of a hobby
to me than a business. I still work the same number of hours, but it's more fun
now that I don't have to worry about making payroll. I have a good staff to take
care of the day-to-day business, and I get to spend most of the day thinking
about better ways of doing things and deciding which new products to make. I
appreciate it all the more because it wasn't always that way. At first I had to
do it all; buying and maintaining machines, making parts, assembling, packaging
and shipping them, doing the bookkeeping and paying the taxes. I realized I had
reached a real benchmark in business when I found that a product had gone from
raw material to delivery and 1 didn't know one thing about it.

\bigskip\textit{Here's a miniature machine tool you won't often see. The ManSon
lathe is a fully functional miniature machine tool made in the 1940's by a Los
Angeles company. It had a number of accessories available, but its extremely
small size limited the projects you could actually make on it. It is one of a
number of miniature machine tools collected for display by the author. (Sherline
chuck and toolpost are for size comparison.)}\bigskip

\secrel{The satisfaction of watching others progress}

Another thing I enjoy is determining how a particular part will be run through
the shop. Designing new products has become easier for me now because of the
wide assortment of tools we own\ --- about a million dollars worth. In 1985, I
could set up and operate every machine I owned, but that time has passed. I
don't operate my own machines now because they are too complex to casually start
pushing buttons. I have to rely on my employees, and I get a lot of enjoyment
out of watching employees progress as they become accomplished craftsmen in
their chosen trade. However, I still don't believe anyone in the shop knows more
about making good parts than I do. I may not know what button to push any more
but I'm still the best at solving problems in the shop. I've learned a lot about
machining over the last 30 years and I'm going to try to pass on some of that
knowledge. Because of my experience 1 can compare methods used by a hobbyist and
a professional machinist. I've have also added information that I hope you will
find interesting about machining. It will give me a lot of satisfaction if I
inspire readers to strike out on their own and start a new business with a
product that has been "prototyped" on Sherline machines.

\secrel{The Inspection Department only finds mistakes after it's too late}

Most of this knowledge I've gathered has been learned the hard way because money
was too tight to hire experts. At Sherline we make all of our own parts and only
contract out the plating, heat treating, and powder coating. In the past, we
have also done a lot of contract machining and I've learned the problems one can
get into by finding errors in the inspection department. It's just too late.
Parts must be inspected as they are built, not after. Errors found after the
parts are made mean you start over. Design errors found after the parts are made
will always result in scrap. The only difference is who pays for the scrap.

\secrel{Work extra hard to eliminate errors when ``the chips are down''}

I've never met a good craftsman who wants to do a job over, even when he is
getting paid for it. It goes against his nature. I have also never met a good
craftsman who has never had to do a job over because of his own mistakes. This
is a good time to stay away from him, because he is mad at himself. The fact is,
you can't work with this many types of tools, dimensions, and materials without
making an occasional error. The trick is not to make errors when it counts. Good
toolmakers will work with an entirely different attitude when they are making an
inexpensive fixture than they will when working on apart that has thousands of
dollars worth of material and labor in it.

\secrel{Inattention can lead to more than just scrapped parts}

You can't have a couple of beers and machine good parts. The job is too
demanding. Machining is a serious business. Inattention can result in scrap or,
worse yet, injury. You can always make another part but you can't grow a new
hand. Even a machine as small as a Sherline lathe or mill can give you a nasty
cut. Machinists may have to work for days at a time with their hands in close
proximity to moving cutters and parts, yet there are few injuries. They pay
attention to what they are doing.

\secrel{The credit for a good part goes to the craftsman}

Good craftsmen know when they have made an exceptional part and get much
satisfaction from it. They also have the ability to produce good work on
machines that should be in a junkyard. It just takes them longer. I have a great
respect for good craftsmen, because they have to work without excuses or
erasers. I try to keep reminding you of this fact in this book, because it is
the craftsman, not the machine, who builds the beautiful things we see daily in
this world. Modem machines have given this talented group of people a way to
produce more and better work, but it will always be their "touch"\ that makes
those parts beautiful. In my eyes they just don't seem to get enough respect.

\secrel{An open invitation}

If you ever travel to San Diego, California, the Sherline factory is less than
an hour away to the North. It's also about two hours South of Los Angeles. I
always offer an open invitation for anyone to stop by to see how modern
production machines produce parts used in Sherline tools.

\bigskip
\textcolor{Cyan}{``You've achieved success in your field when you don't know
whether what you're doing is work or play.''}

\bigskip
\textcolor{Cyan}{--- Warren Beatty}
\bigskip

\bigskip\textit{Sherline's facility has a showroom where you can see the entire
line of tools and accessories as well as some sample projects built on the
tools. Factory tours are available for anyone who would like to see how
miniature machine tools are manufactured.}\bigskip

\secup
\secup
