\secdown\secrel{Introduction}\secdown

\secrel{The essence of ``craftsmanship''}

I wrote the introduction to this book last. That's
because when I started writing, I didn ' t quite know
where I was headed. I knew that over the years I
had written many instruct ions for OUT products which
contained enough knowledge and advice to be
valuable. I also figured I cou ld start writing answers
to questions that had been asked of me over the
years. I cou ld fill the remainder of the book with
pictures and charts and end up with a book that
wouldn't be any diffe rent or better than what was
already out there. For me, therefore , the most
important part was to try and instill in a poten tial
machinist the va lue of good craftsmanship. Great
craftsmen not only get the job done, they add a
certain "look" to parts they build. It is alm ost a
signature. I have seen the same part made by two
different craftsmen using the same drawing. They
were both highly ski lled toolmakers. Both parts met
the specifications perfectly, yet I could easi ly tell
who built each part. Machining should be considered
a fonn of art.

\secrel{Some pretty good advice}

\bigskip\textit{Professional photographer, Tim Schroeder of Michigan built these
five identical Stirling hot air engines to polish his skill as a new machinist.
By making each pari five limes, he was able to get in more machining lime with
each setup and learn more in a shorter period. .. a preuy good way to learn.
}\bigskip

On the wall of my Uncle 's shop whcn I was a boy
was a sign whi ch I st ill remcmber. I' m not sure who
sai d it, but I think it expresses what I' m trying to
say pretty well. It said:

\bigskip
\textcolor{Cyan}{A man who works wilh his hands is a laborer.}

\bigskip
\textcolor{Cyan}{A man who works wilh his hands and his brain
is a craftsman.}

\bigskip
\textcolor{Cyan}{A man who works wilh hands, his brain and
his heart is an artist}

When I was building model aircraft, my friends and I had an interesting way of
judging the quality of a model. We would set the model aircraft on the ground
and start backing away from it until it looked good. A three· foot model would
be considered superb and a fifty·foot model was one that was pretty crude. There
were also models that wouldn't look good no matter what the distance was or the
viewing angle. In those cases, the failure was in the design, and the best
craftsman in the world can't make a bad design look good.

\secrel{The best design is usually not your first design}

The home machinist usually has more control over a design he is working with
than a professional does. Don't use the first idea that comes into your head
without proving to yourself that it's the best way. When a product has been
designed properly, no one would even consider building it in a different way.

It is the way it is supposed to look because it's obvious. Unfortunately, these
are the designs thai are the hardest to come up with. They are also the designs
you will get the least credit for even though they are your best. The assumption
is that the obvious solution is also the easy so lut ion , but this is usually
not the case. The home craftsman also doesn't have to work within the
constraints of commercial products where costs limit your choices.
For us, time is not money, it's fun!

This is what craftsmanship is all about. Too few
citizens really appreciate what good craftsmen do .
Because their work doesn't fail it is taken for
granted. A good craftsman can tell at a glance when
someone's work is better than his. and he can slart
improving his work to be Number One. It is almost
a form of competition between craftsmen where time
and quality are considered at the same time . Do YOll
think Michaelangelo would be considered a great
artist if he had ,only carved one statue and painted
one picture? He produced so much good work in
his lifetime that he set a standard that is slill sought
after today. One good part doesn't make you a
craftsman. You are judged on the body of your work.

\bigskip\textit{Author Joe Martin is shown with some of the miniature machine
tools produced by Sherline Products. The small size miniature machine tools
makes them easy to use and not too intimidating/or new machinists.
}\bigskip

I not only wanted my writings to be useful to the
hobbyist/machinist who builds parts for pleasure,
but also to those future craftsman who want to build
parts that have that " look". Please realize the parts
bcing referred to in this book are not production
parts. Machinists who produce these kinds of parts
have the training and skill to make automatic
machines build good parts. The only thing an
automatic machine will manufacture automatically
is scrap. It still takes that craftsman's touch to make
machines run perfectly. The parts being di scussed
in this book will be parts built one at a time ... "one
off'. These parts are usually part of another assembly
that would be considered the final product.

\secrel{You don't become a machinist by buying a machine}

You should strive from the beginning to make better and more accurate parts than
you think you need. Work to closer tolerances than the job demands. Be on the
lookout for ways to make ajob easier or better. I hope you will enjoy the
process of creating accurate parts from raw metal. Buying a machine won't make
you a machinist, but using it along with the skill and knowledge you acquire
along the way eventually will.

\secrel{What new machinists like most and least}

If you are new to machining, you may find it to be
either one of the most rewarding skills one can learn
or the mo st frustrating thing you have ever
attempted. What makes machining fun for some is
the complexity and challenge. The same thing will
drive others up the wall. Onc pcrson may be
overjoyed because he can now make parts that were
not available for purchase. Another may wonder why
he just spent all day making a part Ihat is similar to
onc he could have purchased for two dollars. (The
difference, of course. is that it is not the same as the
two dollar part- it is exactly the part needed.)

\secrel{There are no shortcuts}

Machining is a slow process because parts are made one at a time. The
interesting thing is. a ski lled machini st may take almost as long to make the
same part as a novice. Shortcuts usually end in failure.
Unlike some olher trad es. mistakes cannot be covered up . There are no erasers,
white-out or "putting-on 100 15" for machinists. You simpl y start over. Do a
lot of thinking before you slart culling.
To expand a littlcon an old rule: "\emph{Think three limes, measure twice and
cut once!}"

\secrel{Anticipation of a tool's limitations
is the crahsman's strength}

The skill in machining isn't just " moving the dials". It is a combination of
enginee ring and craftsmanship. A file is just as useful a tool to a machinist
as a multi-thousand dollar machine tool. Tools "deflect" or bend under load, and
anticipating this bend is what it is all about. Sharp tool s deflect less than
dull tools, but with each pass the tool dulls a little and thc deflection
becomes greater. If you try to machine a long shaft with a small diameter, the
center will always have a slightly larger diameter than the ends because the
part deflects away from the tool where it has less support. You can go crazy
trying to machine it straight, or you can simply pick up a good, flat mill file
and file it straight in a few moments. Machine tools will never replacc the
"craftsman's touch", and machining is a combination of both good tools and good
technique.

\bigskip\textit{ Jewel-like projects like this miniature marine winch are a
showcase for the kind of craftsmanship machinists strive for. Being able to
display your work on a desk or coffee table or even carry it with you in your
pocket is an advantage of working on small projects.}\bigskip

\bigskip\textit{Here's a miniature machine fool you won 'f often see. The ManSon
lathe is a fully functional miniature machine tool made in the J940 's bya Los
Angeles company. It had a number ofaccessories available, but ils eXlremely
small size limited the projects you could actually make on it. It is one ofa
number ofminiature machine tools collected fhr display by the author. (Sh er/ine
chuck and too/post are for size comparison.) }\bigskip

\secrel{The great ports about running a business like this}

I'm a hobbyist who has been lucky enough to make a living at a hobby I enjoy.
I own and manage Sherline Products Inc. and enjoy coming up with new products.
After working at if for over twentyfive years, this has become more of a hobby
to me than a busi ness. I still work thc same number of hours, but it ' s morc
fun now that I don't have to worry about making payroll. I have a good staff to
take carc of thc day-to-day business, and I get to spend most oCthe day thinking
about bettcr ways of doing things and deciding which new products to make. I
appreciate it all the more because it wasn't products has become easier for me
now because of always that way. At first I had to do it all; buying the wide
assortment of tools we own- about a million dollars worth. In 1985, I could set
up and and maintaining machines, making parts, assembling, packaging and
shipping them, doing the operate every machine I owned, but that time has
bookkeeping and paying the taxes. I realized I had passed. I don't operate my
own machines now reached a real benchmark in business when I found because they
are too complex to casually start that a product had gone from raw material to
pushing buttons. I have to rely on my employees, and I get a lot of enjoyment
out of watching delivery and I didn't know one thing about it.

\secrel{The satisfaction of watching others progress}

Another thing I enjoy is detennining how a particular believe anyone in the shop
knows more about part will be run through the shop. Designing new products has
become easier for me now because of wide assortment of tool s we own-about a
million dollars worth. Ten yea rs ago, I could set up and operate every machine
I owned, but that time has pa ssed. I don 't operate my own machines now because
they are too complex to casually start pushing butlons. I have 10 rely on my
employees, and I get a lot of enjoyment out of watching them progress to become
accomplished craftsmen in their chosen trade. However, I still don ' t believe
anyone in the shop know s more about making good parts than I do. I may not know
what button to push any more, but I'm still the best at solving problems in the
shop. I've learned a lot about machining over the last 30 years and I' m going
to try to pass on some of that knowledge. Because of my experience I can compare
methods used by a hobby ist and a professional machini st. I've have also added
information that I hope yo u will find interesting about machining. It will give
me a lot of sati sfaction if I inspire readers to strike out on their own and
start a new business with a product that has been "prototyped" on Sherline
machines.

\secrel{The Inspection Department only finds mistakes
aher it's too late}

Most of thi s knowl edge I've gat hered has been learned the hard way because
money was too tight to hire experts. At Sherline we make all of ou r own parts
and onl y contract out the plating, heat treating, and powder coating. In the
past, we have also done a lo t of con tract machinin g and I've learned th e
problems one can get into by findin g errors in the inspection department. It 's
just too late. Parts must be inspected as they are built, not after. Errors
found after the parts are made mean yo u start over. Design errors found after
the parts are made wi ll always result in sc rap. The only difference is who
pays for the scrap.

\secrel{Work extra hard to eliminate errors when
``the chips are down''}

I' ve never met a good craftsman who wants to do a job over, even when he is
getting paid for it. It goes again st hi s nature . I have also never met a good
craftsman who has never had to do a job over because of his own mistakes. This
is a good time to stay away from him , because he is mad at himse lf.
The fact is, you can' t work with thi s many types of tool s, dimensions, and
materials without making an occasional error. The trick is not to make errors
when it count s. Good toolmakers will work wit h an entirely different atti tude
when they are making an inexpensive fixture than they will when working on a
part that has thousands of dollars worth of material and labor in it.

\secrel{Inattention can lead to more than just scropped parts}

You can't have a couple of beers and machine good
parts. The j ob is too de manding. Machining is a
seri ous business. Inattenti on can result in scrap or,
worse yet, injury. You can always make another part
but you can't grow a new hand. Even a machine as
small as a Sherline lathe or mill can give you a nasty
cut. Machinists may have to work for days at a time
with their hands in close proximity to moving cutters
and parts, yet there are few injuri es . They pay
attention to what they are doing.

\secrel{The credit for a good part goes to the craftsman}

Good c raftsmen know when th ey have made an exceptional part and get much sati
sfaction from it. They also have the ability to produce good work on machines
that should be in a junkyard. It just takes th e m longer. I ha ve a g reat re
spe c t for good c raftsmen , bec au se the y have to work w ith out exc uses o
r erasers. I try to keep reminding you of thi s fact in thi s book, because it
is the craft sman , not the machine, who builds the beaut iful things we see
daily in this world. Modern machines have given this talented group of people a
way to produce more and better work , but it wi ll always be the ir "touch" that
makes those parts beautiful. In my eyes they just don ' t seem to get enough
respect.

\secrel{An open invitotion}

If you ever travel to San Diego, California, the Sherline factory is less than
an hour away to the North. It's also about two hours South of Los Angeles. I
always offer an open invitation for anyone to stop by to see how modern
production machines produce parts used in Sherline tools.

\bigskip
"You've achieved success in yourfield when you don't
know whether what you're doing is work or play. "
\bigskip
--- Warren Beatty
\bigskip

\bigskip\textit{Sherlille s facility has a showroom where you can see the entire
line of tools and accessories as well as some sample projects built on the
tools. Factory fours are a vailable for anyone who would like to see how
miniature machine tools are manufactured.}\bigskip

\secup
\secup