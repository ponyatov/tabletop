\secrel{Chapter 6\ --- Cutting tools for metalworking}\secdown

\secrel{6.1— GENERAL NOTES ON CUTTING TOOLS}\secdown

\secrel{Tool Deflection}

In machining, the cutting tools are the interfaces between the machine and the
part. This is where the part is cut and shaped. Machines are built which are
capable of incredibly accurate movements, but these accurate moves can't help
the process if the cutter is deflecting by large amounts. A 1/2"\ (13\,mm) end
mill can deflect as much as .050"\ (1.25mm). The deflection is a function of the
diameter of the tool compared to its length, plus the sharpness of the tool, the
feed rate and the depth of cut all have to be considered. You must understand
this fact to successfully make machined parts. It is not a business of just
numbered movements. It is a process that requires a great deal of thought and
planning to produce good parts. The better you get at anticipating these basic
problems, the better you will become at machining metals.

\bigskip
\textit{A long, thin piece of stock will deflect away from the cutting tool.
Overcoming the shortfalls of a process or the quirks of a particular machine to
make consistently good parts is what separates a hacker from a craftsman.}
\bigskip

A machinist will use every available skill to overcome these obstacles by
"sneaking up"\ to final dimensions to make the first part good. If more identical
parts are to be made, ways can then be found to speed up the process. Common
sense should tell you that if a cutter deflects a large amount on heavy cuts, it
will still deflect a small amount on light cuts. Tool or part deflection has to
be considered in every case. Whether you are operating a machine the size of a
locomotive or as small as a Sherline, the problems remain the same, but
fortunately, the costs don't. A large part can have thousands of dollars worth
of material and labor invested before a machinist touches it. How would you like
to be the employee that has to tell his boss he had just bored a hole oversize
in a twenty-thousand dollar forging?

The home machinist will have to learn these facts by trial and error. They don't
usually have the advantage of having a skilled machinist nearby to rely on for
information. I'll never understand why someone relatively new to the trade will
resent a skilled craftsman suggesting a better method of doing the job.
Information is being passed on that could be of tremendous help to them. I
imagine they have never spent days of their own time trying to solve a problem
that could have been solved in minutes by a skilled craftsman. The best and
fastest way to teach yourself is by constantly checking the part as it is being
machined. Learn how much the machine and tools deflect as they are being used.

\secrel{Overcoming Problems on a Lathe}

Lathes are easier to deal with than mills because the operations are
straightforward. They are two dimensional machines, and diameters are easy to
measure. The problems encountered with lathes usually involve the part bending.
A perfectly aligned lathe will not cut absolutely straight unless the part is
rigid enough to support itself. The shape of the cutting tool will influence the
amount of deflection. A sharp pointed tool, with the cutting edge perpendicular
to the diameter being machined, will have less deflection force than a tool with
a large radius. Large radius lathe tools can be a loser if you are working on
parts that are small or on a worn out machine because the forces are directed
away from the headstock and towards the part. This can bend the part or load
worn out slides. On a Sherline lathe a maximum tool radius shouldn't exceed
.015". On high speed steel lathe tools, I will grind a small flat on the point
rather than a radius to improve the finish. You can't grind the small radii
required free hand, but you can grind a flat on a bench grinder which will work
almost as well. Hand ground lathe tools are described completely in the section
on "grinding your own tools"\ later in this chapter. They can be not only the
fastest method of removing metal, but also the cheapest.

\secrel{Electrical discharge machining or ``EDM''}

There is another way of cutting metal with which many home machinists are not
familiar. I want you to know about this process, because I'm sure you may be
curious about how a modem machine shop deals with hardened materials. It is
called "Electrical Discharge Machining"\ or EDM. This is a slow process that does
the seemingly impossible by using an electrode to remove metal. The electrode is
made to the opposite shape you want to produce. (For example, a square electrode
will produce a square hole.) When the electrode gets within .001"\ of the work,
small sparks jump across the gap. The temperature of these sparks is hotter than
the surface of the sun, and the metal actually disintegrates. In a sense, this
happens one molecule at a time, which is why it is a slow process. It also takes
place submerged in a special oil. The electrode, which is usually a pure form of
graphite, has been shaped to burn a pocket into the hardened steel. The
electrode also disintegrates during the process. EDM machines have controls that
can be set to cut faster at the expense of the electrode; therefore, several
identical electrodes are usually used to produce each shape. Each succeeding
electrode makes the cut more and more accurate.

\bigskip
\textit{An EDM removes metal by actually vaporizing it with a high powered
electrical discharge or spark. It is capable of great accuracy and of making
shapes that would be impossible with other methods.}
\bigskip

Debris is created as the EDM operates. This will "short out"\ the cutting action
so the part must be flushed with oil to wash away the debris even though it is
submerged in oil. Much of the skill in operating these machines is how you
accomplish this. It is much more difficult than it sounds. I use an EDM machine
to build plastic molds because the electrode is many times easier to shape than
the cavity. Molds are built with materials which can be hard to machine.

\secrel{Other special uses for EDM}

A simple form of EDM machine has been developed just to remove broken taps from
parts. The electrode is simply a tube that has a diameter which is less than the
inside diameter of the hole in which the tap has broken off. Electrolyte is
pumped down the tube and the simple EDM burns the center of the tap away. The
broken tap can then be "picked out"\ and, if done correctly, the hole will not be
damaged by the sparks. There is also a highly sophisticated EDM that cuts with a
small diameter wire which is fed through the part like a bandsaw. They are used
to make cutting tools like punches and dies and are amazingly accurate but very
expensive\ --- around \$100,000. EDM machines have changed the way parts are
designed and built. Many businesses have been started that take advantage of the
capabilities of these machines which are expensive and have a limited use. I've
even seen several designs for "homemade"\ EDM machines that will work as long as
great deal of detail isn't required.

\secrel{EDM, CNC, CAD/CAM and other acronyms of the computer age}

Today, EDM electrodes are made on CNC mills which are controlled by computer
programs that take information directly from the drawing that has also been
produced on a computer. This type of work is where the term CAD/CAM (computer
aided design/computer aided manufacturing) came from. These can be exciting
times for the machinist who leams the new rules that govern this new way of
doing things.

\secup

\secrel{6.2\ --- CUTTING TOOLS FOR BOTH THE LATHE AND MILL}\secdown

\secrel{Using Center Drills}

Center drills have been designed to drill a hole that a tailstock center will
use to support the part. They have a short straight section at the end of a
sixty-degree cutting edge. As the chart shows, numerous sizes are available to
take care of any job. However, they do have a few problems you should be aware
of. The straight section doesn't have any clearance on the sides like a drill.
Without cutting oil the chips clog up the flutes and they can twist off. The
reason they don't have clearance is to make them "find"\ the center of bar stock
in a lathe. Drills that are flexible can start drilling off center without the
help of a center drill to get them started. Even with cutting oil you can't
drill to the required depth without backing out the center drill completely and
adding cutting oil to it's tip. The straight part of the hole was to make a
pocket to keep lubricants for lubricating the supporting sixty-degree tailstock
dead center. A great deal of heat from friction can be generated at the center
without proper lubrication. White lead was the preferred lubricant of its day
because it didn't gall, but for obvious health reasons it isn't available any
more. On lathes we now use live centers which rotate with the part to eliminate
friction.

\bigskip
\textsc{COMMONLY AVAILABLE CENTER DRILL SIZES}
\bigskip

\secrel{The difference between a Center Drill and a Spotting Drill}

Center drills are an important tool in machining and are used as much on a mill
as a lathe. It is a simple fact of life that drills will "walk"\ when starting a
hole that hasn't been started with a short, rigid drill. Center drills were the
only tools available that could start holes without wandering. The only problem
was, it took too long to drill a pocket for the lubricant that wasn't needed.
Years ago, I saw this problem

\bigskip
\textit{A center drill (top) and a spotting drill (bottom).}
\bigskip

and started making "spotting drills"\ for use in our own shop. We could save
minutes of machine time on one part by not using center drills. They have
finally brought to the market drills of this type called "spotting drills". When
compared to the complexity of a standard drill, spotting drills cost more than
they should, for they could be manufactured cheaper than a center drill.

\bigskip
\textit{A center drill being used to find the center of a piece of stock. The
$60^{o}$ angle will locate the point of the tailstock center for turning between
centers. The smaller hole will serve as a reservoir for lubricant if a live
center is not to be used.}
\bigskip

A drill press can be the most dangerous machine in a machine shop. They are so
simple to use that operators don't give them the respect they deserve. New
employees are often given drill press jobs to start their machining career.
Never hold a part by hand when drilling into metal or similar types of
materials. The drill can grab the part and start it spinning at the RPM the
machine was set at. The greatest chance of this happening is when a hole is
enlarged with a larger drill. The softer the material, the better the chance of
this happening. The first part of this scenario is the where the drill screws
into the part, lifting it off the table. When the drill jams in the part, the
part will spin. If this happens. step away and try to safely shut down the
spindle. Whatever you do, don"t be a hero and try to grab the part. You could
end up looking like you have just been through a war in the 16th century. Also,
when drills are broken off they may leave a very sharp point in either the
spindle or the part that could cut to a depth that could result in permanent
injury. Clamp your parts down firmly and cut down on injuries.

\secrel{Who came up with these stupid drill sizes anyway?}

Manufacturers of cutting tools sometimes have a strange way of applying logic.
Take the standard drill indexes that are available today. They are still the
same as sets from fifty years ago. If anyone could explain the logic of the
sizes they manufacture I would love to know, They simply keep making the same
old sizes they have always made without questioning the logic. Overseas
manufactures aren't any better for they just copy what is available. (Obviously
this is one of my pet peeves.) The quality of the better drills manufactured
today is superb, but it is too bad they don't make them in the proper sizes. The
better industrial suppliers may offer high and slow helix (flute twist), but I
would recommend that you stick with the standard "jobber"\ drills. They are
cheaper and work quite well. The hobbyist/machinist will want to buy a drill set
for working with the miniature machine tools. My first choice would be a number
1 to 60 set. (Sizes .040"\ to .228".) This set will have the drills you need for
drilling and tapping small holes. A miniature tool doesn't have the power or the
size for large drills. On smaller machines, larger holes are bored rather than
drilled.

\bigskip
\textit{The metal box holds a set of drill bits sized from number 1 through 60.
The red plastic box holds a selection of very small bits from number 61 through
80. In the foreground are \#1, \#60 and \#80 sizes.}
\bigskip

There is a series of drills called "screw machine drills"\ that are shorter and
better suited for small machines. They make them in both left and right hand.
Only right hand drills will work on a Sherline because of the direction of
rotation. Use high speed steel drills because of their low cost and long life
when properly used. Stay away from carbide drills unless you have a particular
need for them. They are very brittle and most small tools can't produce the
spindle speeds needed for carbide to have an advantage. Printed circuit drilling
machines, for example, may run carbide drills at over 60,000 RPM.

\secrel{Using drill bits properly}

Drills get more abuse than any other metal cutting tool made. Watching a home
repair expert stick a drill back in a hole that they just drilled with a slight
error and try to make the hole oblong by bending the drill should give a
machinist the horrors. To begin with, the flutes on a drill were never designed
to cut, but somehow they do when enough pressure is applied. They are designed
to pull the chips out of the hole and a small amount of coolant is a must,
especially for holes going more than two drill diameters deep. If you want
accurately drilled holes, you can't abuse drills. Small diameter drills are
usually discarded when dull to maintain accuracy. Hand-sharpened drills will
normally drill oversize because the cutting lips are not exactly equal. To keep
drills sharp, they have to be turned at a speed that is forty percent or less of
the material cutting speed they are drilling. Exceeding this speed can instantly
destroy the cutting edge. Dulling tools in this manner is foolish because speed
is one of the easiest variables to control. Think about the way American Indians
would start the campfire. They would turn a straight, dry stick back and forth
with the palms of their hands until friction generated enough heat to start a
fire on the wood where the stick point was rubbing. Now consider how fast they
could have started that fire if they powered that stick with a five horsepower
motor! Hopefully I have made my point.

With the centering hole drilled and the RPM set correctly it is time to drill
the hole to size. The Sherline motor doesn't have the horsepower for large
diameters and high feed rates, but that isn't a problem as long as you work
within the capabilities of your tools. It just takes a little longer. The real
problem is keeping a constant feed rate and knowing how deep the drill can go
without "pulling"\ the drill.

The drill point must be lubricated and the chips must be removed from the drill
flutes. With automatic production machinery, the drill can go as much as three
and a half times the drill diameter into the hole before it is backed out to
clear the chips. From that point on, I usually program depth of cut in one drill
diameter increments. Using manual machines I would recommend two times the drill
diameter for the first cut and one diameter beyond that point. When using small
drills, the amount that can be drilled without clearing the drill is as small as
the drill diameter. This is very important and the reason I am emphasizing it. A
drill that has been broken off in a part will usually result in "scrap"". To
make matters worse, when you start over again, unless you have a spare bit, the
one you need is now broken. Good drilling technique makes this one problem that
is avoidable.

\secrel{Should a hole be drilled in stepped sires or directly to size?}

I have found that once the part is center drilled, drilling a hole to size with
a single drill will give better results than enlarging a smaller hole with a
series of larger drills. Every time a different tool enters a part it can pick
up off center, because the drill before it may have left a sharp burred edge in
the hole. If this method must be used, make sure the spotting or center drill
will cut to a diameter that includes all the sizes you will be using and has a
ninety or sixty degree tip. Brass is a difficult material in which to enlarge a
drilled hole. In fact, it can be downright dangerous, because a standard drill
can be pulled into a pre-drilled hole and lock up. Machinists often put a slight
flat on the cutting lip of drills used to enlarge a hole in brass to avoid this
problem.

\secrel{Drilling stainless steel}

First-time machinists will discover it's a whole new ball game the first time
they attempt to drill a hole in a piece of stainless steel. It is called "work
hardening"\ and it is just as the name implies. I'm sure everyone has taken a
wire and bent it back and forth until it broke. What was done was the metal kept
getting harder at the bend until it became so brittle it fractured. Stainless
steel as well as many other exotic metals can surface harden while being
machined. Small drills will make this fact even more apparent. If a drill is
allowed to rub the surface without cutting, it will work harden the surface it
is trying to drill. This happens at the bottom of a hole that is being drilled.
The cause can be a dull drill or not using a sufficient feed rate to keep the
drill cutting. This will work harden the surface, which will dull the drill,
which will work harden the surface even more until the drill will not cut at
all. To work around this problem, the spindle has to be turning slow enough to
have a feed rate that allows the drill tip to maintain a constant, uninterrupted
cut. The calculated spindle speed may be much faster, but the machinist can't
keep up with the higher RPM. Also, don't leave the drill at the bottom of a hole
when it isn't cutting. Raise it immediately. If a hole gets work hardened, check
the drill for sharpness and start drilling again with a slight "tap". This will
sometimes allow the drill point to break through the hardened surface and start
drilling again.

\secrel{A Note on very large drill bits}

Drill bits at the large end of the size scale usually have a tapered shank and
could never be used on a Sherline, for they are as big as the machine. However,
there is still something to be learned from large diameter drills. In many
cases, larger drills have a tapered shank with a "tang". A tang is a rectangular
shape on the small end of the taper that engages with a holder designed to
accept it so the drill cannot spin. The added cost for manufacturing this type
of drill is worth every penny if it keeps the drill from spinning, as this will
ruin the shank. The way standard twist drills are kept from spinning is by
tightening the drill chuck with a key to create enough friction to lock them in
place. If they are not tightened enough, they will spin in the chuck, and you
simply can't tighten the chuck enough by hand without a key. Many machinists
never learn this fact and the results of their carelessness can be found in tool
cribs around the world.

\bigskip
\textit{A large drill bit with a tapered shank. On larger machine tools where
cutting forces are higher the flat tang keeps the drill bit from spinning in the
chuck. Miniature machine tools don 7 have enough power to require this kind of
holding force, but you still need to tighten your chuck with a key so the drill
can 7 be ruined by spinning in the chuck.}
\bigskip

Additional information on large drills is not included because it isn't very
useful for small machines like the Sherline. \emph{Machinery's Handbook} will be
the best source of information for large drills. I have also learned much
information from catalogs on cutting tools printed by mail order suppliers. I
would also suggest you inspect a small drill under a magnifying glass some time
to appreciate the level of perfection to which these tools have been made. Their
very reasonable cost will be appreciated even more once you get a really close
look.

\secrel{Countersinks}

The countersinks I prefer to use have only one cutting lip, and they will cause
you very few problems with chatter. Countersinking tools are used to chamfer the
sharp edge left after a hole has been drilled. A wide variety of angles are
available. They are also used to allow flat-headed screws to fit flush with the
part surface. They should be run at a very low rpm when used in manual machines.
You can instantly destroy these tools in stainless steel by using excessive
speed with a hand drill. If the hand drill you are working with doesn't have a
speed control, keep the RPM low by pulsing the trigger switch. The speed should
be such that you can see the tool cutting. If the tool is not perpendicular to
the hole, the chamfer will be uneven.

\secrel{Step drills}

Step drills have one or more diameters and are made to eliminate changing tools
to save time in machining operations. Only consider the step drill's smallest
diameter for feeds and its largest diameter for speeds. It is easy to twist the
tip off if not enough cutting oil is used. They are expensive, and because of
the limited sizes available will usually be custom-made by local toolmakers.
Only use them when the cost is justified by the time saved.

\secrel{Counterbores}

Counterbores are used to enlarge existing holes. They have a "pilot"\ shaft which
can be changed to different sizes on larger diameters counterbores. One of the
most common uses for these tools today would be counterboring a clearance screw
hole for a socket head cap screw. (Many times these screws are referred to as
"Allen cap screws".) Spot facing rough or curved castings so bolts can be pulled
down on a

\bigskip
\textit{Top: a 1/4"\ short shank counterbore of high speed steel. Above: a longer
style counterbore is well suited for screw heads and spring pockets. There are
many more styles and shapes,}
\bigskip
\textit{A pilot sized to the drilled hole goes in the front of each counterbore.
They are normally sold separately and are held in place with a set screw.}
\bigskip

flat surface is another common use. If the pilot is the right size it keeps the
tool from wandering, Counterbores will usually have three flutes and a very slow
helix, making their diameters difficult to measure, They have to be turned at
low RPM with a moderate amount of pressure similar to using a countersink. They
are expensive.

\secrel{Reamers}

Reamers are available in more sizes than drills. A reamed hole will have more
accurate diameters and better finishes than a drilled hole. However, if they are
not used properly, a good drilled hole may be better than a reamed hole. They
come with straight and spiral flutes. In theory, straight flutes will work
better if the hole goes all the way through the part because a lot of chips get
in front of the cutting edge. Spiral flutes will help pull the chips from the
bottom of a "blind"\ hole. A right-hand cut, left-hand spiral will push the chips
forward and should never be used in a blind hole. I always order straight flutes
for general use. A feed rate of .001"\ per tooth is recommended, so the RPM must
be slow enough to maintain this feed. Ample cutting oil must be applied, and the
reamer must be pulled out of the hole often so chips can't build up in the
flutes. The reamer must be located directly over the center of the hole to work
properly.

The safest way is to ream the hole to size immediately after drilling. I usually
allow less than .008"\ (.2mm) on sizes less than 5/16"\ (8mm) and for over this
diameter I allow 1/64"\ (.4mm) for clean-up with a reamer. A reamer can start
cutting off-center if the drilled hole has chips lying on the hole edge; in fact
on CNC machines, we occasionally put a program stop to allow the operator time
to blow the chips out with an air hose. What happens is the chip will start
rotating with the reamer, shutting down the cutting action on one side. The
reamer will cut oversize until the chip is forced from its location. It will
then start cutting on-size leaving the hole with a "bell-mouthed"\ entrance.
Fractional inch sizes are quite a bit cheaper than decimal size reamers. Try to
design your parts to have finish diameters that take advantage of this fact.

\secrel{Taps}

Taps are cutting tools to cut screw threads in holes (see thread cutting
instructions for thread definitions) and are made in many sizes and shapes. Taps
cut several thread shapes and the sixty-degree thread is standard. The "acme"
and "square"\ threads are not available in diameters under one half inch and will
very seldom be used by the hobbyist. For the type of work done in miniature
machining, I would not consider large taps. If needed, you can usually get the
information from most industrial supply catalogs.

The same rules should be applied to large and small taps. Taps are defined by
their diameter first, which may be a wire gauge size, and followed by the number
of threads per inch. Metric taps will be defined by diameter and actual pitch.
Drill and tap charts will give the information to cut 75% of a full thread.
Engineers have discovered long ago that a full thread isn't necessary. A 75%
thread is almost as strong and the difficulty in cutting a full thread just
isn't worth the extra effort and cost.

To tap a high quality thread, you need to start with a proper hole size and use
a lubricating cutting fluid. A chamfered hole will start the tap better and will
not raise the surface where the hole is being tapped; however, thin parts may
require no chamfer in order to have the maximum number of threads for a given
thickness.

\secrel{Why cheap taps are a poor investment}

If you have a choice as to the size you can use, look at a tool catalog and pick
out a size that is readily available. The savings in cost will be worth the
effort. Remember also that just because the size may be listed doesn't mean you
can buy it at every industrial supplier, never mind a local hardware store. Buy
taps before you start the job, and it would be wise to buy an extra one. They
are very easy to break. If possible, I get my tapping done early on complex
parts. If a tap does break and ruins the part, you are not scrapping a lot of
labor. If you are having a problems tapping holes, the first question to be
asked is about the quality of the taps you are using. Inexpensive tap sets can
be so bad they wouldn't cut butter. They usually are found in department stores.
Stay away from them no matter how tempting they may look. Tap sets sold by
industrial suppliers are better, but I still advise buying only what you need,
when you need it. Buy quality taps and save hours of grief.

Taps are made in a variety of styles, and my favorite are the "spiral point gun
taps". In smaller sizes they have two flutes and are the strongest series of
taps available. "Spiral pointed"\ means that the chip is pushed ahead of the tap.
This eliminates the need to back up a tap to break the chip as four-fluted hand
taps require. These taps are made with two choices for the cutting tip: "plug"
and "bottoming". Use the plug style for general use. Spiral point taps have been
designed to push the chips out the bottom of the hole that is being tapped. In
blind holes, the chips will pile up at the bottom and could cause a problem.

\bigskip
\textsc{SPIRAL POINT TAP}

\textsc{SPIRAL FLUTE TAP}
\bigskip

\secrel{Spiral pointed taps vs. spiral fluted taps}

Don't confuse spiral pointed taps with spiral fluted taps. Spiral fluted taps
pull the chips out of the hole, which may seem like an excellent idea until the
strength of the tap is considered. They break easily, because under load they
"unwind"\ and jam in the hole. The problem eliminated by these taps can be
canceled out by their inherent weakness. If possible, drill the hole deeper than
it will be threaded to make room for these chips. After tapping, the chips can
be removed from the hole with tweezers.

A problem can arise when a tap is reversed. The chips can jam the tip in a blind
hole, breaking the tap, but tap breakage is less with two-fluted than with
four-fluted "hand taps"\ because of the extra strength of these fine cutting
tools. In small sizes, the four-fluted hand taps should be avoided but some
sizes are only available in that form. "Starting"\ taps have a shallow cutting
angle mat will cover many threads. Each cutting tooth will form a small amount
of the thread. While this may seem like a good idea, it does not always work
well. With many threads cutting at the same time, the torque required to drive
them may be higher than a plug style spiral pointed gun tap, but they do start
the tap straight down the hole with less effort.

\secrel{Tapered pipe thread taps and dies}

Pipes used in plumbing use a National Pipe Thread (NPT) standard which is a
tapered thread. Tightening the parts together seals them as the tapers meet,
making a tight joint. When tapping a hole for a tapered thread, measure the part
you will be threading into it to make sure you don't run the tap too deep. NPT
threads start at 1/16-27 and go up to over 3 inches. You might use some of the
smallest sizes on miniature plumbing fittings for steam engine models or pipe
fittings in small engine blocks.

\secrel{Hand tapping a hole}

When tapping holes by hand, the tap must be perpendicular to the hole. I
recommend having the part clamped down so the tap will be pointed towards the
center of the earth. It is easier to line up when the tap and handle are square
to its surroundings. The quickest way to break a tap is casually holding the
part in your hand and a tap in the other. If the tap is not "square"\ it will
not follow the hole and will progressively start to cut one side more than the
other. This will quickly reach the point of no return when the cut exceeds die
strength of the tap. It is very difficult to straighten a tapped hole. Again,
you must use a cutting fluid. It not only keeps chips from sticking to the
cutting edge, it also keeps the tap from sticking to the cut thread. Remember
that "like"\ materials will gall and stick together, and this is one reason taps
break at a alarming rate.

\emph{NOTE}: Broken taps can be dangerous! Pieces of the tap can be thrown at a
very high speed and easily do eye damage, Always wear eye protection. The broken
tap can leave a sharp cutting edge where it has broken and give a nasty, deep
cut that could require stitches.

\secrel{Tapping stainless sleel}

Hand tapping a small diameter screw thread of 75\% may seem impossible if the
material is stainless steel. The tip of the tool will twist' to the point of
almost breaking before the tap will even start to cut. Dull or poor quality taps
will not stand a chance in stainless. A 65\% thread may help, but a change such
as this must be authorized if the part isn't for your own use, and considered if
it is for your own use. Don't use a tap until it breaks. They are disposable
tools and have a life-span that can be short with exotic materials.

\secrel{Tap wrenches}

The tap wrenches for holding taps used by amateurs will usually be the standard
hand tap holder. Buy the size that just goes to one-quarter inch and, if you can
locate a smaller size, buy it too. It will be useful to have a very small tap
wrench for the small sizes that you will encounter while working on very small
parts. When you work with equipment that is too big for the job, you lose that
"feel" that keeps taps from breaking. A tap wrench has a chuck similar to a
drill chuck but it has only two jaws. The jaws allow the tap to be held by the
square, machined end on the tap. Good quality tap wrenches will clamp the tap in
line with the tap wrench body, making it easier to get the tap square with the
part. When working with full size mills, I will use the spindle to line the tap
over the hole. Some tap wrenches have a sliding shaft protruding from the
handle. Using this shaft, the holder can then be held in a collet or drill chuck
and be supported so that the holder is always square and directly over the hole.
To cut successfully, taps must start square while cutting the first threads.

\bigskip
\textit{A standard "T"\ handle tap wrench and a tap wrench with a pilot shaft
for holding in a collet or drill chuck.}
\bigskip

\secrel{Tapping threads on big CNC machines}

Automatic tapping on a modern CNC milling machine is accomplished by reversing
the spindle. The tap is held in rigid holders and the spindle is electronically
synchronized with the feed to turn in relation to pitch of the tap. It does this
so accurately, a "floating"\ holder is not necessary. It is amazing to see a
part being tapped at 2000 rpm! Metal will cut better at a higher rpm and the
quality of the tapped holes will be better. Of course, you can't do this on
inexpensive equipment, but I thought you might find it interesting.

\secrel{Tapping heads for manual mills and drill presses}

What you may be able to afford some day is a small tapping head. They can be
used with manual drill presses and mills. A good quality tapping head costs
about \$400 (1998). They are designed to turn in one direction when the pressure
is applied to drive the tap into the part and in the opposite direction to
unscrew the tap when the tapping head is raised. The tapping head is usually
geared to turn the tap at twice the input rpm when backing the tap out of the
work. This makes the process more efficient. They make several models, and for
small taps I prefer a cone clutch design over the dog clutch. Dog clutches
engage with a "snap"\ which has a tendency to break small taps. Also, they
cannot tap left-handed threads. Cone clutches operate more smoothly.

When a tapping head is used, the operator must put enough pressure on it to
start the threading process. The tap will then be screwed into the hole at the
rate of its pitch. If the feed rate, controlled by the operator, does not equal
the pitch of the tap, the clutch will disengage. On a drill press, the depth of
the tapped hole will be controlled by the "depth control stop". When you are
tapping blind holes, you set the stop to make sure the tap will not go all the
way to the bottom. Measure the distance to go and reset the stop to eliminate
any error. A tap will go deeper than the stop is set because it has to disengage
the clutch. When the tap is unscrewed, a slight withdrawal pressure is needed to
operate the opposite clutch and any excess pressure will "tear" the thread as
the tap leaves the now-threaded hole. A definite feel for the process is needed
to operate these labor saving devices effectively.

\secrel{Tricks of the trade\ldots A tapping tip from Bob Shores}

I have read many tips on tapping holes\ --- some good, some not. Five years ago
I dreamed up a tapping method for small holes. I tap a lot of holes with 0-80
and 2-56 threads, and since I have been using this method, I have not broken a
tap in five years.

After drilling the hole in your part to the proper size, the drill bit is
removed from the chuck without disturbing the work, A 2"\ aluminum disk, knurled
on the outside and drilled and tapped for a 4-40 hex bolt grips the tap just
above the flutes. The end of the tap is gripped in the drill chuck and lowered
until it just touches the work. The chuck is then loosened to allow the tap to
turn freely. The disk holding the tap is turned with your thumb and forefinger.
The drill chuck acts as a guide to keep the tap running true, and your fingers
are very sensitive to the amount of torque being applied. To break a tap you
would have to apply a lot of force.

\bigskip
--- Bob Shores
\bigskip

HOLE SLIGHTLY LARGER THAN TAP

2"\ DIA $\times$ 1/4"\ KNURLED ALUMINUM DISK

4-40 $\times$ 1"\ HEX BOLT

DRILL CHUCK

7"\ KNURLED ALUMINUM DISK

TAP
\bigskip

\secup

\secrel{6.3 LATHE CUTTING TOOLS}\secdown

\secrel{Right- and left-hand tool shapes}

Again, the difference between a lathe and a mill is that the work turns on a
lathe and the tool turns on a mill. Before we get into grinding lathe tools
let's define this type of cutting tool as a single point tool that is fixed
(doesn't cut by rotating) and held in a tool block on the crosslide of a lathe.
By the way, lathe tools can be cut as "right-hand" or "left-hand" shapes. The
reason we call a tool a "right-handed" tool when the cutting edge is on the left
is because it is designated by which way the chip leaves the cutting tool. A
left-handed tool is designated as such because the chip will go to the left as
it cuts. The cutting edge will be on the right. The standard tool is a
right-hand tool, and a right-hand ground tool bit is included with each Sherline
lathe to help get a novice started. Scissors or sheet metal shears are also
defined this way, because the cutoff falls to the left or right. Right-handed
people will usually prefer "left-hand" shears.

\bigskip
CHUCK

TAILSTOCK

LEFT-HAND TOOL

RIGHT-HAND TOOL

\bigskip

\textit{A tool is called a "left-hand tool" because the chip comes off to the
left even though the cutting face is on the right. On a right-hand tool the chip
comes off to the right.}

\secrel{Why you should ovoid cheap carbon steel tools}

The Sherline lathe has been designed to use 1/4" square tool bits. You don't
have many choices when it comes to grinding your own lathe tools. The shapes you
will need to produce special parts can't be made with standard cutting tools. If
you read an old book on machining, it may have mentioned carbon steel tools, and
these are case hardened tools. The only way you might find these inferior tools
would be in cheap imported sets. Today, the labor and machine cost to produce a
good cutting tool exceeds the material cost by so much that it just doesn't make
sense to use cheap steels.

\bigskip
\textit{A selection of 1/4" high speed steel and carbide tipped cutting tools
for miniature machining. Shown are left to right: A high speed steel boring
tool, right-and left-hand cutting tools, right- and left-hand carbide tools and
a carbide $60^o$ threading tool.}

\secrel{High speed steel tools}

For the average machinist, "high speed steel" will always have a use in a
machine shop. It is inexpensive and easy to grind and shape. High speed steels
comes in a variety of grades. For the average work done in a home shop, M-2 tool
steel is more than adequate. M-5 would be considered top-of-the-line. Some tool
steels contain several other metals to add to their life. Cobalt is a common
additive that makes the cutting edge less prone to "chipping" and is effective
in adding life to a tool. Usually, the more high speed tool steel costs, the
harder it will be to grind and the longer it will hold an edge. You will find
the answers to all engineering problems are a trade-off, just as this one is.

\secrel{Brazed carbide and diamond tipped tools}

Sherline offers brazed carbide tools in left, right, and $60^o$ threading.
Brazed carbide tools have a shorter life than inserted tip carbide tools because
the carbide has been brazed to the holder and their different expansion rates
can causes problems. The left-hand brazed carbide tool works well with the
flycutter and is included with its purchase. They can be useful but you need a
diamond wheel to resharpen them. One thing you must realize is that some
materials can only be cut with carbide. You don't have a choice. These materials
usually have an abrasive nature and are not that "hard". In general, hard
materials (heat treated tool steels) can't be successfully machined. You can't
chuck up an end mill in a lathe and machine its shank down with carbide or
diamond cutting tools. Diamond cutting tools are used to get beautiful finishes
on nonferrous materials and shouldn't be used on steels. The carbon in the
diamond will weld to the carbon in the steel and destroy the diamond.

\secup

\secup
