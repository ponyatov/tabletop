\secrel{Chapter 6\ --- Cutting tools for metalworking}\secdown

\secrel{6.1— GENERAL NOTES ON CUTTING TOOLS}\secdown

\secrel{Tool Deflection}

In machining, the cutting tools are the interfaces between the machine and the
part. This is where the part is cut and shaped. Machines are built which are
capable of incredibly accurate movements, but these accurate moves can't help
the process if the cutter is deflecting by large amounts. A 1/2" (13\,mm) end
mill can deflect as much as .050" (1.25mm). The deflection is a function of the
diameter of the tool compared to its length, plus the sharpness of the tool, the
feed rate and the depth of cut all have to be considered. You must understand
this fact to successfully make machined parts. It is not a business of just
numbered movements. It is a process that requires a great deal of thought and
planning to produce good parts. The better you get at anticipating these basic
problems, the better you will become at machining metals.

\bigskip
\textit{A long, thin piece of stock will deflect away from the cutting tool.
Overcoming the shortfalls of a process or the quirks of a particular machine to
make consistently good parts is what separates a hacker from a craftsman.}
\bigskip

A machinist will use every available skill to overcome these obstacles by
"sneaking up" to final dimensions to make the first part good. If more identical
parts are to be made, ways can then be found to speed up the process. Common
sense should tell you that if a cutter deflects a large amount on heavy cuts, it
will still deflect a small amount on light cuts. Tool or part deflection has to
be considered in every case. Whether you are operating a machine the size of a
locomotive or as small as a Sherline, the problems remain the same, but
fortunately, the costs don't. A large part can have thousands of dollars worth
of material and labor invested before a machinist touches it. How would you like
to be the employee that has to tell his boss he had just bored a hole oversize
in a twenty-thousand dollar forging?

The home machinist will have to learn these facts by trial and error. They don't
usually have the advantage of having a skilled machinist nearby to rely on for
information. I'll never understand why someone relatively new to the trade will
resent a skilled craftsman suggesting a better method of doing the job.
Information is being passed on that could be of tremendous help to them. I
imagine they have never spent days of their own time trying to solve a problem
that could have been solved in minutes by a skilled craftsman. The best and
fastest way to teach yourself is by constantly checking the part as it is being
machined. Learn how much the machine and tools deflect as they are being used.

\secrel{Overcoming Problems on a Lathe}

Lathes are easier to deal with than mills because the operations are
straightforward. They are two dimensional machines, and diameters are easy to
measure. The problems encountered with lathes usually involve the part bending.
A perfectly aligned lathe will not cut absolutely straight unless the part is
rigid enough to support itself. The shape of the cutting tool will influence the
amount of deflection. A sharp pointed tool, with the cutting edge perpendicular
to the diameter being machined, will have less deflection force than a tool with
a large radius. Large radius lathe tools can be a loser if you are working on
parts that are small or on a worn out machine because the forces are directed
away from the headstock and towards the part. This can bend the part or load
worn out slides. On a Sherline lathe a maximum tool radius shouldn't exceed
.015". On high speed steel lathe tools, I will grind a small flat on the point
rather than a radius to improve the finish. You can't grind the small radii
required free hand, but you can grind a flat on a bench grinder which will work
almost as well. Hand ground lathe tools are described completely in the section
on "grinding your own tools" later in this chapter. They can be not only the
fastest method of removing metal, but also the cheapest.

\secrel{Electrical discharge machining or ``EDM''}

There is another way of cutting metal with which many home machinists are not
familiar. I want you to know about this process, because I'm sure you may be
curious about how a modem machine shop deals with hardened materials. It is
called "Electrical Discharge Machining" or EDM. This is a slow process that does
the seemingly impossible by using an electrode to remove metal. The electrode is
made to the opposite shape you want to produce. (For example, a square electrode
will produce a square hole.) When the electrode gets within .001" of the work,
small sparks jump across the gap. The temperature of these sparks is hotter than
the surface of the sun, and the metal actually disintegrates. In a sense, this
happens one molecule at a time, which is why it is a slow process. It also takes
place submerged in a special oil. The electrode, which is usually a pure form of
graphite, has been shaped to burn a pocket into the hardened steel. The
electrode also disintegrates during the process. EDM machines have controls that
can be set to cut faster at the expense of the electrode; therefore, several
identical electrodes are usually used to produce each shape. Each succeeding
electrode makes the cut more and more accurate.

\bigskip
\textit{An EDM removes metal by actually vaporizing it with a high powered
electrical discharge or spark. It is capable of great accuracy and of making
shapes that would be impossible with other methods.}
\bigskip

Debris is created as the EDM operates. This will "short out" the cutting action
so the part must be flushed with oil to wash away the debris even though it is
submerged in oil. Much of the skill in operating these machines is how you
accomplish this. It is much more difficult than it sounds. I use an EDM machine
to build plastic molds because the electrode is many times easier to shape than
the cavity. Molds are built with materials which can be hard to machine.

\secrel{Other special uses for EDM}

A simple form of EDM machine has been developed just to remove broken taps from
parts. The electrode is simply a tube that has a diameter which is less than the
inside diameter of the hole in which the tap has broken off. Electrolyte is
pumped down the tube and the simple EDM burns the center of the tap away. The
broken tap can then be "picked out" and, if done correctly, the hole will not be
damaged by the sparks. There is also a highly sophisticated EDM that cuts with a
small diameter wire which is fed through the part like a bandsaw. They are used
to make cutting tools like punches and dies and are amazingly accurate but very
expensive\ --- around \$100,000. EDM machines have changed the way parts are
designed and built. Many businesses have been started that take advantage of the
capabilities of these machines which are expensive and have a limited use. I've
even seen several designs for "homemade" EDM machines that will work as long as
great deal of detail isn't required.

\secrel{EDM, CNC, CAD/CAM and other acronyms of the computer age}

Today, EDM electrodes are made on CNC mills which are controlled by computer
programs that take information directly from the drawing that has also been
produced on a computer. This type of work is where the term CAD/CAM (computer
aided design/computer aided manufacturing) came from. These can be exciting
times for the machinist who leams the new rules that govern this new way of
doing things.

\secup

\secup
