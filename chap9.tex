\secrel{Chapter 9\ --- General machining terms}\secdown

\secrel{General rules for cutting speeds, feed rates and depth of cut}

Three terms frequently used in machining are "speed", "feed"\ and "depth of cut".
Reference to the diagrams below will show what is meant by these terms. Normal
turning on a lathe, when used to reduce the diameter of a work piece, involves
advancing the cutting tool perpendicularly to the lathe bed by an appropriate
amount (depth of cut) and feeding the tool along parallel to the lathe bed to
remove material over the desired length. The diameter will be reduced by twice
the depth of cut because the tool doesn't cut just one side.

\secrel{"Cut"\ and "feed"\ on a lathe}

In normal machining on a Sherline lathe, the depth of cut can be directly set by
the crosslide handwheel. Some lathes are manufactured with a dial on the
crosslide that is calibrated to the actual "change in diameter". This is common
and is a good question to ask before using a different lathe. Sherline's lathe
has a crosslide that reads directly. This allows the crosslide to be used as a
milling table and have the calibrated handwheels reading correctly. The feed is
provided by the handwheel on the end of the bed. When facing off the end of a
work piece held in a chuck or faceplate, the depth of cut is set by the
handwheel on the end of the bed, and the feed is provided by the crosslide
handwheel. If the work extends beyond the safe working limits, additional
support may be needed. A center or steady rest should be considered.

\bigskip
NORMAL TURNING

FACING

FEED

CUT
\bigskip

\textit{Lathe cutting terms}
\bigskip

\secrel{"Cut"\ and "feed"\ on a mill}

When using a mill, cut is determined by the amount of depth the cutter is set to
by the Z-axis handwheel. Feed is supplied by either or both the X- or Y-axis
handwheels depending on the desired direction of the cut.

\bigskip
CUT ("Z"\ AXIS)

FEED ("X"\ AXIS)

FEED ("Y"\ AXIS)
\bigskip

\textit{Mill cutting terms}
\bigskip

\secrel{Effects of cutting rates on tool life}

Speed, feed and depth of cut all have an effect on tool life. Speed is the rate
the cutting tool is moving in relation to the material being cut. Speed has the
greatest effect on tool life of any of the three variables mentioned, which is
why it is mentioned so many times in this book. I don't want you to ruin
perfectly good cutting tools by breaking these simple rules. For example, a 1/4"
diameter drill cutting stainless steel should have a correct cutting speed of
800 RPM. This is a speed that can easily be exceeded on all but the largest
lathes. The Sherline lathe and mill are capable of 2800 RPM and at half that
speed, the drill could be ruined. The largest diameter of the stock to be turned
should be considered when determining the cutting speed on a lathe. With
Sherline tools it is so easy to adjust the RPM that it would be beneficial to
adjust the speed to the diameter you are working on.

Turning CNC machines will automatically adjust their spindle speed as the
diameter is decreased, saving those valuable seconds. Remember also that the
main difference between a lathe and a mill is that on a lathe the work turns,
while on a mill the tool turns.

\secrel{Cutting speed vs. type of tool and material being cut}

When considering cutting speeds, only the difference in speed between the tool
and work should be considered. However, the type of tool that is being used may
dictate a different cutting speed. Different tool shapes will have cutting
speeds that not only consider the type of material being machined and the
material the tool is made out of, but also consider the chip "flow". Drills are
a good example of this, A drill, deep in a hole, can't get rid of chips the same
way a lathe tool can. Basic cutting speed charts for materials represent lathe
tools. Drills and similar tools have a much lower cutting speed. A home
machinist must work with speeds that are well below the calculated speeds
because it is quicker in the long run. It isn't worth the effort to calculate
the exact rpm of a machine that is going to make one or two cuts, but it is time
well spent if thousands of similar parts are to be made. A portion of a chart
showing these differences is provided below. Machinery's Handbook will provide
tables in greater detail should you need more information.

\clearpage
\bigskip
GUIDE TO APPROXIMATE LATHE TURNING SPEEDS

\bigskip

\begin{tabular}{|l|l|l l l|}
\hline
MATERIAL & Cut Speed & 1/4"(6mm) & 1/2'(13mm)& 1"\ (25mm) \\
& S.F.M. & Diameter & Diameter & Diameter \\
\hline
				
Stainless, 303& 67 & 1000 RPM & 500 RPM & 250 RPM \\
Stainless, 304 &&&&\\
Stainless, 316 &&&&\\
Steel, 12L14 &&&&\\
Steel, 1018 &&&&\\
Steel, 4130 &&&&\\
Cray Cast Iron &&&&\\
Aluminum, 7075 &&&&\\			
Aluminum, 6061 &&&&\\
Aluminum, 2024 &&&&\\
Brass &&&&\\
\hline
\end{tabular}
\bigskip

\secrel{Cutting speed}

Proper cutting speed is the rate a particular material can be machined without
damaging the cutting edge of the tool that is machining it. It is based on the
surface speed of the material in relationship to the cutter. This speed is a
function of both the RPM of the spindle as well as the diameter of the part or
size of the cutter, because, as the part diameter or cutter size increases, the
surface moves a greater distance in a single rotation. If you exceed this ideal
speed, you can damage the cutting tool. In the lathe and mill instructions we
give some examples of suggested cutting speeds, but what I wanted to get across
here is that the damage doesn't occur slowly, It isn't a case of getting only
one hour of use instead of two. A tool can be destroyed in just a few seconds.
The cutting edge actually melts. If you machine tough materials like stainless
steel, you will ruin more tools than you care to buy if you don't pay a lot of
attention to cutting speeds.

\bigskip
\textit{With big machines in a production environment, time is money, so feed
rates and cutting speeds are critical. Saving time and making parts fast is not
as big a factor for the home machinist.}
\bigskip

\secrel{Calculating spindle speed for cutting}

To calculate the RPM's of a cutter or work piece, you first must know the
cutting speed of the material that is being machined and the material from which
the cutter is made. High speed steel will be our standard tool material, and
cutters that use carbide tips will be considered separately. You could fill a
library with all the books written on speeds and feeds, and the reason is
simple. Time is money, and when parts are being manufactured in large
quantities, you have to use the most efficient method to machine a part. The few
seconds of time that is saved on each part will be multiplied by the total
number of parts manufactured, giving the total time saved. As an example, a part
might be produced for the automobile industry at a rate of three million parts
per year. A clever machinist comes up with a way to save three seconds per part.
This will save around 2500 hours, which would probably be worth over \$100,000
to the company. The corporation would probably award the machinist with a "thank
you"\ and a check for around \$300 and the managers would take all the credit for
the increased profits. (I guess I've been reading "Dilbert"\ comic strips for too
long.) With this kind of money at stake, you must get it right, which is why
books go into such detail on speeds and feeds. Home shop machinists, however,
need not clutter their minds with a bunch of complicated tables, Since
production speed is not usually a factor in home shop work, knowing a few simple
rules and speeds will keep you out of trouble.

If you work in the inch system, there is an easy way to calculate the spindle
speed. Multiply the material's recommended cutting speed (given in feet per
minute) times four. Then divide the answer by the diameter of the work (lathe)
or cutter (mill) given in inches. This will give you an approximate answer that
is more than accurate enough to use while working on Sherline or similar
miniature machine tools. Here is the formula:

\begin{equation}
RPM=\frac{CS \times 4}{D}
\end{equation}

\begin{description}
\item[CS] Cutting Speed of the material from your table in SFM (Surface Feet per
Minute)
\item[4] feet-to-inches conversion factor (12) divided by $\pi$ (3.14159) =
3.82 which is rounded off to 4
\item[D] Diameter of part (lathe) or cutter (mill) in inches
\item[RPM] Spindle speed
\end{description}

By rounding off the conversion factor, it makes this formula easy enough to do
in your head in most cases, yet it is still accurate enough.

Example: On a mill, what RPM should a 3/8"\ (.375") diameter high speed steel
cutter turn if it is to have a 100 fpm cutting speed?

$$\frac{100 \times 4}{.375} = \frac{100 \times 4}{3/8}=
\frac{100 \times 4 \times 8}{3}=1067\ RPM$$

If the metric system is used, the cutting speed will be given in meters per
minute and the diameter will be given in millimeters. The same problem would be
worked out as shown below using 3.14 to equal the value for $\pi$.

$$
\frac{30.39 \times 1000}{3.14 \times 9.5}=
\frac{30'390}{29.83}=
1019\ RPM
$$

(1000 = the conversion from meters to millimeters in which the diameter is
given.)

\secrel{A factor for the shape of the cutting tool}

There is one more factor to consider when estimating spindle speeds (RPM).
Cutting speeds are given for a lathe tool unless otherwise stated. This is the
most optimistic number given for cutting speeds and must be modified for other
types of cutters, as using them at the suggested speeds will shorten their
cutting life. Multiply the calculated spindle speed by the following factors for
the following types of tools:

\bigskip
\begin{tabular}{l l l l}
End mills = .9 &             
Reamers = .6 &
Drills = .7  &                
Taps = 25 \\
\end{tabular}
\bigskip

Example: Since the cutter in the example above is a mill, the speed would be
multiplied by .9 to give a final speed of 960 RPM (inch) or 917 RPM (metric). If
the speed in the example above were for a drill, you would multiply the answer
by the factor .7 which would lower the spindle RPM to 747 (inch) or 713
(metric). The factor listed for taps is, of course, only for automatic machines
or for use with a tapping head, as you could never hand tap that fast.

\bigskip

END MILLS (Slot and side milling)

\bigskip

\begin{tabular}{l l l l l}
\hline
MATERIAL & CUT SPEED (S.F.M.) & 1/8"\ DIA & 1/4"\ DIA. & 3/8"\ DIA. \\
\hline
Stainless Steel, 303\\				
Stainless Steei, 304\\				
Sloinless Stella\\				
Steel, 12L14\\				
Steel, 101B\\				
Steel, 4130\\				
Gray Cast Iran\\				
Aluminum, 7075\\				
Atuminum, 6061\\				
Aluminum, 2024\\				
Aluminum, Cast\\				
Brass\\				
\hline
\end{tabular}

\bigskip
DRILLS
\bigskip

\begin{tabular}{l l l l}
\hline
MATERIAL & CUT SPEED (S.F.M.) & 1/16"\ DIA. & 1/4"\ DIA. \\
\hline
Carbon Steel\\			
Cast Iron, Soft\\			
StuinlessSleel\\			
Copper\\			
Aluminum, Bat\\			
Aluminum, Cast\\			
\hline
\end{tabular}

\bigskip

\secrel{Feed Rates}

When operating manual tools, often the operator may not be able to move the
handwheels fast enough to accomplish ideal feed rates. It is a better choice to
slow down the spindle speed rather than taking feed cuts that are so fine they
will work harden the surface, dull the cutter or induce tool "chatter".

Other than speed, feed rate is the other great force that has an effect on tool
life. Although feed is described in many ways, it is always describing the
amount of material that is removed by each cutting edge per revolution of the
spindle. Many times you will find it described as inches per minute (or mm/
min.). With the information we now have, it is easy to calculate a single
tooth's feed rate. Feed rate controls deflection by either the tool or work and
turns into a load that has a very dramatic effect on accuracy. It is impossible
to consistently work accurately with tools that are deflecting, and the amount
they deflect is dependent on both the depth of cut and the feed rate. (See "End
Mills"\ on page 81 in Chapter 6 of this section.)

\secup
