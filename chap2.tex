\secrel{Chapter 2\ --- Do you need a lathe, a mill or both?}\secdown

\textit{The lathe and the vertical milling machine each have their place in the
machine shop. Though the lathe is a basic metalworking tool, the mill is the
workhorse in most machine shops. Unless your needs are very limited, you will
eventually need the capabilities of both to be able to accomplish all your
machining tasks.}
\bigskip

\secrel{Which tool is the most important when getting started?}

The lathe is the first complex piece of machinery an apprentice machinist will
use to cut metal. A mill is the machine an apprentice machinist will use to make
his or her first complex part. Lathes have always been a great way to leam about
cutting metal, but as soon as you have that urge to build a miniature version of
a steam engine, you'll find out that a mill is needed more than a lathe. The
truth is, to make complex parts you need both. I would estimate I have spent 90%
of my time working with a mill compared to a lathe; however, when a part has to
be turned and threaded you need a lathe.

A lathe is a good place for a novice to start. It will allow you to find out
about cutting metal and requires a smaller investment. I don't recommend buying
every metal cutting tool in sight until you know you like cutting metal. Metal
cutting is a complex, slow process that should be enjoyable before any large
investment is made. I would suggest that if your funds are limited and you still
want full machine shop capabilities, get started by buying the best mill you can
afford and the least expensive lathe you can get by with. The reason is that
most likely the majority of critical operations you will perform will require a
vertical milling machine.

\secrel{The difference between mill and lathe isn't just square or round parts}

The main difference between a lathe and mill is that the work turns on a lathe
while the tool turns on a mill. Most people believe the difference would be
rectangular vs. round material. This isn't true. A four-jaw independent chuck
can be used to hold a rectangular part on a lathe. Four-jaw independent chucks
allow the work to be mounted with the center being wherever you want it. The
only problem is the offset weight of the part can make for a very out-of-balance
setup. When tools are cutting, the material being cut doesn't care which is
turning, and the same cutting speed laws govern the process whether it is a
lathe or mill. Old manuals on machining will show many ingenious setups using
lathes. Mills were slow to catch on because the end mills we use today were not
available. The new tool steels that improve modern day cutting tools constantly
affect the design of the machine tools to which they are mounted as engineers
take advantage of these new materials and products.

\secrel{Milling with a lathe and vertical milling column attachment}

Sherline has an attachment to turn a lathe into a mill that works well as long
as the work is small enough. A lathe doesn't have to be as rigid as a mill
because the cutting loads are lower; therefore, in this configuration the XY
portion will not be as rigid as a mill. This accessory is called the vertical
milling column. It is basically the same as the vertical slide on the Sherline
mill; in fact, the vertical column attachment that is presently manufactured
could be bolted directly to the XY milling base. The headstock/motor/speed
control assembly from the lathe is exactly the same as the one used on the mill
and can be switched from one to the other in less than a minute. This allows a
novice to work or buy his way into miniature machining a little at a time.

To summarize, money saving alternatives that would allow both lathe and mill
operations to be performed would then consist of 1) a lathe with a vertical
milling column, or 2) a lathe and mill XYZ base. The headstock/motor/speed
control unit would be switched back and forth between the two machines saving
the cost of duplicating a second drive unit. The first alternative is the least
expensive but results in slightly reduced milling capabilities. If most of the
parts you make are turned on a lathe and you only occasionally need to do a
milling operation this would be a good choice. The second alternative, although
it takes a few seconds to change drives from one machine to the other, results
in no compromise in milling capabilities.

\bigskip
\textit{Maybe we'd better rethink the pricing on this new accessory just a
bit\ldots}
\bigskip

\secrel{Pricing based on the ``flinch factor''}

In the future we may manufacture a mill that could be turned into a lathe, but
that would result in a higher initial investment for customers. I try to come up
with prices that are set by what I call a "flinch factor". This is what
prospective customers automatically do when they find out what something costs
at a trade show. If they walk away without asking any questions, the price may
be too high. If they take out their hard-earned money on the spot and buy one,
it may have a price that could be raised. I price things accordingly, and I
don't necessarily make the same percent of profit on each item. I try to make a
profit on the overall product line rather than each item. I guess I will always
be more of a hobbyist than a businessman.

\secrel{Inch vs. metric machines}

Machine tools come calibrated in cither inch or metric increments. Choosing a
system of measurement will be one of the more basic choices you will have to
make. For a more thorough discussion of the advantages and disadvantages of each
system see Chapter 5 in this section on measuring. The simplest advice I can
give is to buy a machine in the system you are most familiar with and for which
you already have measuring tools. If you grew up with the inch system, buy an
inch machine. If you think in millimeters, buy a metric machine. All Sherline
tools and accessories are offered in cither system at the same price. There is
no significant advantage in accuracy to a machine calibrated in one system as
opposed to the other.

Converting dimensions from one system to the other is a pain and a possible
source for errors. If you buy a metric machine thinking that is the way the
world is going, but you are buying plans for projects that are dimensioned in
inches and own inch micrometers, you are going to be in for a lot of extra
work. Although it is possible later on to convert a Sherline machine from inch
to metric or vice versa, it involves more than just swapping handwheels as the
leadscrews and nuts must also be changed. In most cases, the choice of which
system to use will be an easy one for you, and that is to simply choose the
system with which you are most comfortable.

\secup

\secrel{Finding the right Sherline tool for your needs... a guide through the
product lineup}
\secdown

The motor and speed control are the same on all Sherline lathes and mills, so
the difference in tools is mainly in size and included features. Assuming you
have made a choice of your system of measurement, I will describe the
differences and advantages of the various machines in the Sherline tool line.

(NOTE: Where model or product numbers are listed, the inch version is listed
first followed by the metric version in parenthesis.)

\secrel{LATHES}\secdown

Model 4000\ --- The basic Sherline lathe is the Model 4000 (4100) which has a
15"\ bed with 8"\ of clearance between centers. This is the modern version of
the original Australian-designed Model 1000 lathe introduced in the early
1970's. All Sherline lathes come with a faceplate and drive dog, two dead
centers, hex adjustment keys, tool post, presharpened 1/4" cutting tool and
spindle bar.

\paragraph{Model 4400}

The longer lathe Model 4400 (4410) has a 24"\ bed with 17"\ of clearance
between centers. It also comes with adjustable "zero" handwheels on the
leadscrew, crosslide and tailstock spindle. It replaces the standard tool
post with a rocker tool post. This allows you to precisely control the height of
the cutting edge of the tool, which gives better control over the cut. This is
particularly useful when using older, resharpened cutting tools where the height
of the cutting tip may have changed when it was resharpened.

\bigskip
\textit{This photo shows the relative size of the two lathes. The main
difference is the length of the bed which provides 8" between centers on the
Model 4000 and 17" between centers on the Model 4400. (This photo shows the
older style tailstock without the cutout for table clearance which was added in
1996.)}
\bigskip

\secrel{Is Sherline's bigger lathe worth the extra money?}

Additional capacity may just be wasted if you don't need it. If you don't need
17" between centers you might as well save the workbench space and spend the
money you save on more accessories. If you do need it, however, the relatively
small extra cost is well worth it. The additional 9" of distance between centers
obviously allows larger parts to be worked on. It also allows for greater
versatility in setup and the use of a larger 3/8" tailstock chuck. Using larger
chucks and tools on the smaller machine is difficult as the length of the chuck
and tool eats up a good portion of your available 8" leaving little space left
for a part. The longer bed lathe was only introduced in 1993, but it now
accounts for just about half the Sherline lathe sales. Though more expensive,
because of the extra features it offers as well as the increased capacity, I
feel that dollar for dollar it is the better bargain of the two.

\secrel{What you need will be determined by the hardness of the parts you intend
to make}

That gives you the physical limitations of the machine, but what does the
hardness of the material you wish to turn do to those numbers in the real world?
A good rule to remember when it comes to purchasing any lathe is to take the
average diameter you plan to work with and multiply that times 3 for free
machining materials and times 4 for tough materials like stainless steel. If the
materials you plan to work with are free machining (aluminum, brass and free
machining steel), you will be pleased with a Sherline Lathe if the average part
you make is approximately 1" (25mm) in diameter. Wood and plastic are so easy to
machine that only size limitations need be considered. I don't mean to imply
that you can't machine a 3" flywheel, but if you are planning to consistently
make parts of that size, you will probably be happier with a larger machine and
more horsepower. Removing large amounts of metal on a small machine takes time.
If you have lots of time, the size of the part is less critical. Users of any
machine are happier with its performance when they are not consistently pushing
the limits of its capabilities. If you usually make small parts well within the
capabilities of the Sherline lathe and every once in a while need to turn a part
sized near the machine's limits, you will be very satisfied with the its
performance.

\secrel{Accessory Packages offer more ``bang for the buck''}

The accessory or "A" packages are a good investment for a new purchaser, as they
include the most popular accessories in a package that offers a price savings
over purchasing them separately. These are all accessories you would no doubt be
buying anyway, so you might as well save some money. In addition to the standard
equipment mentioned previously, the "A" packages include a 3-jaw chuck and
tailstock chuck, chuck key and arbor with drawbolt to use the tailstock chuck as
a drill chuck in the headstock. A4-jaw chuck may be substituted for the 3-jaw if
you already have one, but in most cases, the 3-jaw is what most people will want
to start out with. In the case of the Model 4000A (4100A) the 3-jaw chuck is
2.5" in diameter and the tailstock chuck is a 1/4" Jacobs chuck. The longer
Model 4400A (4410A) includes a larger 3.1" 3-jaw chuck and a 3/8" Jacobs
tailstock chuck and key.

\secup

\secrel{VERTICAL MILLING MACHINES}\secdown

\textit{Sherline's basic Model 5000 mill.}
\bigskip

\secrel{The physical limits of a mill}

A vertical milling machine is capable of holding larger parts than a lathe of
similar size because the part is held and only the tool turns. A mill also has a
much longer table throw (X-axis). A deluxe version of the Sherline mill also is
available which offers an additional 2" of Y-axis travel compared to the
standard Mill. It also includes a mill headstock spacer block which adds 1-1/4"
to the throat distance (clearance between the tool and the vertical column).
With the addition of a horizontal milling conversion, surfaces up to 6" x 9" can
be machined without moving the part. This is a very large machinable area for a
tool of this compact size.

Because of its importance in the machine shop, adding a vertical milling machine
was one of my first priorities when 1 took over production of the Sherline tool
line in the early 1970's. The standard Sherline mill is the Model 5000 (5100)
which has a 10" base. It has 8" (203mm) of clearance between the table and
spindle. The travels of the three axes of movement are: X=9.00" (228mm), Y =
3.00" (76mm) and Z = 6.50" (165mm). It has red anodized handwheels with laser
engraved markings.

\secrel{The Model 5400 Deluxe Mill}

An upgraded or "Deluxe" Model 5400 (5410) mill is also available which offers a
12" base. This increases "Y" Axis travel from 3.00" to 5.00" (127mm). It comes
with a headstock spacer block which increases the throat distance from 2.25"
(50mm) to 3.50" (90mm). (This spacer is available as an option on the standard
mill.) It also includes a 1/4" drill chuck, arbor and drawbolt. In addition, the
base and table have laser engraved scales cut into them which makes keeping
track of positions and movements easier. Adjustable "zero" handwheels are
standard on all three axes. (See photo on page 36.)

\bigskip
\textit{The Model 5400 mill adds 2" more Y-axis travel, laser engraved markings,
a headstock spacer block and a 1/4 " drill chuck and key.}
\bigskip

\secrel{The Model 2000 8-Motion Mill}
 
The latest addition to the model line is patterned after the movements available
on the most widely used and imitated full size machine tool in the world\ ---
the Bridgeport mill. In addition to the standard handwheel adjustable movements
of the X, Y and Z axes, and the pivoting headstock, the new mill offers four
additional movements. The new round column base allows the column to be moved in
and out and pivoted left or right. The addition of the rotary column attachment
allows the column to be rotated in a clockwise or counterclockwise direction up
to $90^{o}$ either way. A "knuckle" on the back of the rotary column
attachment allows the column to tilt forward or back. Laser engraved scales on each of
these movements make it easy to set the column to angled settings.

\bigskip
\textit{The Model 2000 mill takes tabletop machining into the year 2000 and
beyond. With eight directions of movement for the part or tool, a part can be
milled or drilled from any angle while mounted square to the table. For more on
the development of the new mill see page 193.}
\bigskip

The Model 5000 or 5400 mills can mill an angle by rotating the headstock. To
drill an angled hole, however, requires either mounting the part on an angle or
adding the optional rotary column attachment. The Model 2000 mill allows a part
to be drilled or machined from any angle while it is mounted square to the
table. These additional movements bring all the capabilities of professional
mills down to tabletop machine shop size.

\bigskip
\textit{Adjustable "zero"\ handwheels follow large machine tool practice by
allowing you to return your handwheel setting to zero at any time without moving
the leadscrew. They can be ordered on new machines or added as a retrofit to
older machines.}
\bigskip

\secrel{ADJUSTABLE "ZERO" HANDWHEELS} 

Most expensive full size machine tools allow the machinist to reset the
handwheel to "zero"\ (or any desired setting) at any time during the machining
operation. All Sherline tools now offer that option.

Operation is simple. Just release the locking nut while holding the handwheel.
Then reset the handwheel barrel to "zero" and retighten the locking nut. Now you
can dial in the amount of feed you want starting from zero without having to
calculate your stopping point. It's a great time saver and also reduces the
chance for errors.

If you own an older Sherline machine, the adjustable handwheels can be ordered
separately and swapped for the standard handwheels. The old handwheels come off
by simply releasing a set screw. The new ones go on just as easily. If you are
buying a new machine, you can save some money compared to switching later by
ordering a machine with the adjustable handwheels already installed. The Model
4400 (4410) lathe and Model 5400 (5410) mill already come with these timesaving
handwheels installed. It costs about an additional $40 on a lathe and $60 on a
mill to order them with adjustable "zero" handwheels. This saves about $20-$35
(1998) over buying them later, plus you don't have to install them. A standard
lathe with adjustable handwheels is known as Model 4500 (4530). A Model 4000A
(4100A) becomes a Model 4500A (4530A). A Model 5000 (5100) mill changes to a
Model 5500 (5510) when adding the adjustable handwheels. Naturally, the XY base
and the XYZ base can also be ordered with or without adjustable "zero"
handwheels. Rather than quote even more model numbers, it's easier just to say
that information can be found on Sherline's web site or in their price list. My
main purpose here was just to let you know they exist and that any handwheel on
any Sherline tool, new or used, can be replaced with an adjustable one to make
your life a little easier.

\secrel{Things to ask when buying a used machine} 

Like full size equipment, a well maintained used miniature machine tool can be
just as good as a new one and sometimes much cheaper. Check for obvious signs of
abuse like major cuts in the table or noticeable play in the movement of the
parts. Check for excessive backlash in the leadscrews.

Many people buy machines thinking they will use them only to find out the hobby
doesn't really appeal to them. Older tools can sometimes be found that have seen
almost no use. If you happen to find a good used machine, you can usually check
with a manufacturer to find out if that model is still available and if parts
are still offered for it. With Sherline tools this isn't a problem, but there
are used tools on the market made by companies that no longer do business in the
United States, and getting parts for them could be a problem.

\secrel{Accessories sweeten the deal} 

If you can find a good used machine for sale ask what accessories come with it,
If the sale includes a number of expensive accessories, you could be getting a
really good deal. Even if you may not think you need them now, you'll probably
be glad you have them later on as you learn how to use them.

\secrel{Check new prices to make sure it's a good deal}

Regardless of the manufacturer, before you buy a used machine, check the latest
factory product literature for current models and prices. Sometimes the
"bargain" prices asked for used machines might be only slightly lower than what
you would pay for a new one and you wouldn't be getting any warranty. Though the
product numbers and sizes listed in this book for Sherline machines are correct
at the time of printing, in the future, additional models and features will
undoubtedly become available.

\secrel{CNC versions of miniature machine tools} 

As we move into the 21st century, the computer will continue to enter more
phases of our lives. One could not hope to become a production machinist these
days without learning to program a CNC machine.

Even one-off and prototype jobs are often done on machines that offer the option
of hand cranking the dials or using computer controls for appropriate
operations. As machinists become more familiar with using computer controlled
machines in their work, they sometimes wish to extend those advantages into the
small parts they make, either at work or at home. Even those who have never
worked with computer controlled machines can learn to use these smaller versions
if they have the need or simply the desire. Just like their larger counterparts,
these machines can take the drudgery out of making the same part over and over
and can also machine three-dimensional shapes that would be difficult or
impossible by hand.

\secrel{CNC machines for education} 

Educators have found that training new machinists to use computer controlled
machines is done at much lower cost using small machines that duplicate all the
functions of the larger ones the future machinists will eventually be using.
Also, a mistake in a program that causes a tool to crash into a chuck can do
some very expensive damage to a full-size machine. It is better to make your
mistakes in miniature as you learn.

\bigskip
\textit{A CNC machining center designed for educational or industrial use has
all the features of larger shop CNC machines, but these features add to the
price.}
\bigskip

\textit{Stepper motors drive the threaded leadscrews on each of the machine's
axes to produce controlled movements based on coded instructions sent from a
computer. They produce good results at a much lower cost than the servo motors
and ball lead screws used on larger, more expensive CNC machines. Shown here are
stepper motors with handwheels mounted to the rear shafts. This provides the
operator the option of manual control when appropriate.}
\bigskip

\secrel{A number of manufacturers to choose from}

Because of the demand for small CNC machines, in addition to Sherline Products,
there are now several companies that offer both conversions for existing
Sherline machines or complete, turn-key miniature CNC machining centers based on
Sherline machines. There is a broad range of price depending on how many "bells
and whistle" you require, but the basic models are amazingly inexpensive.
Stripped-down home versions are available, while the educational models come
with housings and safety features that duplicate those found on full-size
machines. A complete CNC Sherline mill including three stepper motors, four
drivers with power supply and even a new computer with pre-installed operating
system and software can be purchased for under \$2500.00 (2004) with retrofit
kits for your own machine starting under \$1800.00. A fourth (rotary) axis can
be added for \$395.00. CNC lathes and complete CNC lathe/mill/accessory shop
packages are also available. Versions where you supply your own computer offer
additional cost savings for those who have the computer skills to install the
appropriate operating systems and software.

A list of companies dial offer CNC retrofits or complete machines can be found
in the "dealers" section of Sherline's web site at
\url{www.sherline.com/cncdlrs.htm}.

\bigskip
\textit{Any technology sufficiently advanced is indistinguishable from magic."}

\textit{---\ Arthur C. Clarke}

\bigskip
\textit{A full size Bridgeport vertical milling machine. Unlike its miniature
counterpart, this one cannot be stored on a closet shelf and requires a forklift
to move it. The mill shown combines old and new technology. The handwheels can
be cranked by hand or driven by computer numeric control (CNC). The directions
of movement of this machine served as a model for the Sherline Model 2000 mill}

\secup

\secup
